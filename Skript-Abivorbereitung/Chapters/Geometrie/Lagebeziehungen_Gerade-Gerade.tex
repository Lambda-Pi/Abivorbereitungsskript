\subsection{Gerade - Gerade}
	\todo[color=purple]{Kasten für Formel hinzufügen}
	\todo[color=green]{Zusammenfassung-Kasten hinzufügen}
	\todo[color=purple]{Signalwörter-Kasten hinzufügen}
	Beim Betrachten der Lage zwischen zwei Geraden gibt es vier unterschiedliche
	Möglichkeiten, die wir prüfen müssen. Wir empfehlen euch, zuerst die
	Richtungsvektoren zu betrachten und dann zu prüfen, ob diese linear abhängig
	sind oder nicht.\\

	\(\star\) Ist dies der Fall, so können sie entweder parallel sein oder liegen
	sogar aufeinander. Um das herauszufinden, ermittelt ihr einfach den Abstand von
	einer Geraden zu dem Stützpunkt der anderen Geraden (Beziehung \textbf{Punkt -
	Gerade}). Ist der Abstand 0, so sind die Geraden identisch, ansonsten sind sie
	parallel und haben den Abstand d.\\

	\(\star\) Sind die Richtungsvektoren nicht linear abhängig, so sind sie
	windschief oder schneiden sich an einem Punkt. Hierzu setzt man zuerst beide
	Geraden gleich und löst das LGS. Geht es auf, so schneiden sie sich in diesem
	Punkt (setzt das s oder t bitte in die entsprechende Gerade ein, um zu schauen
	an welchen Punkt sie sich schneiden und gebt diesen an).\\
	Schneiden sich die beiden Geraden nicht, so sind sie windschief. Dann ist
	wieder der kürzeste Abstand zwischen den Geraden anzugeben. Hier müsst ihr
	wieder eine Hilfsebene aufstellen. Diesmal sind die Richtungsvektoren der
	beiden Geraden die Spannvektoren der Hilfsebene (siehe \textbf{Parameterform}).
	Aus ihnen könnt ihr sofort die Normalen- oder Koordinatenform mit dem
	Normalenvektor angeben. Euer Punkt auf der Ebene ist dann einer der
	Stützvektoren. Der Abstand der Ebene zum Stützvektor (Beziehung \textbf{Punkt -
	Ebene}) der anderen Geraden ist dann der kleinste Abstand zwischen den
	windschiefen Geraden.
	
	\summary{
		Richtungsvektoren vergleichen:
		  \begin{itemize}
		    \item Sind sie linear Abhängig dann sind die Geraden parallel zueinander
		    oder liegen sogar aufeinander.
		    \begin{enumerate}
		      \item Sützvektor der einen Geraden in die andere Gerade einsetzen.
		      \item Liegt der Punkt drauf liegen die Geraden aufeinander, ansonsten
		      sind sie parallel zueinander.
		      \item Sind die Geraden parallel muss man den Abstand von der einen
		      Geraden zum Stützvektor der anderen einsetzen.
		    \end{enumerate}
		    \item Ansonsten Kreuzen sie sich oder sind liegen windschief zueinander.
		    \begin{enumerate}
		      \item Beide Geraden gleich setzen.
		      \item LGS lösen.
		      \item Geht das LGS auf, so hat man den Schnittpunkt der Geraden.
		      Ansonsten:
		      \item Hilfsebene aufstellen.
		      \begin{itemize}
		        \item Die Richtungsvektoren der beiden Ebenen sind die Spannvektoren
		        der Ebene.
		        \item Ein Stützvektor einer Gerade ist der Stützvektor der Ebene.
		      \end{itemize}
		      \item Abstand der Hilfsebene zu der Geraden finden, deren Stützvektor
		      nicht für die Ebene genommen wurde.
		    \end{enumerate} 
		  \end{itemize}
	}
