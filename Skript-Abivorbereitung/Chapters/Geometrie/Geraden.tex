\section{Geraden}
Geraden können wir mithilfe von Vektoren im Raum darstellen. Schauen wir uns zunächst noch einmal an, wie wir das in der Analysis gemacht haben, um die Parallelen aufzuzeigen: f(x)=mx+c. Mit Vektoren stellt man dies ganz ähnlich dar. Zuerst hat man einen Stützvektor, der dem c ähnlich ist und ohne Variable in der Funktion steht. Der Richtungsvektor gibt, wie schon der Name sagt, die Richtung an. Vor ihm steht eine skalare Variable (meist s oder t genannt) welche sich mit unserem x vergleichen lässt. Der Richtungsvektor entspricht also der Steigung m (diese gibt ja ebenfalls die Richtung an). Somit können wir jeden Punkt \(\vec{x}\) (als Ortsvektor dargestellt) der auf unserer Geraden ist, angeben (\(\vec{a}\) ist der Stützvektor, \(\vec{b}\) der Richtungsvektor):
\[g:\ \vec{x}=\vec{a}+s\cdot \vec{b}\]
 