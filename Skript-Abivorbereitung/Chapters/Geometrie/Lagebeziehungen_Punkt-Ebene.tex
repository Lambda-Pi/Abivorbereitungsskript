\subsection{Punkt - Ebene}
	\todo[color=green]{Kasten für Formel hinzufügen}
	\todo[inline,color=red]{Zusammenfassung-Kasten hinzufügen}
	\todo[inline,color=red]{Signalwörter-Kasten hinzufügen}
	Den Abstand zwischen einem Punkt und einer Ebene berechnet man mit der
	Hesseschen Normalform / Koordinatenform. Ihr müsst für \(\vec{x}\) einfach den
	angegebenen Punkt (dessen Abstand ihr zur Ebene ermitteln wollt) einsetzen und
	habt sofort den Abstand. Zur Wiederholung:
	\formel{\[d=(\vec{a}-\vec{x})\cdot \hat{n} \mathrm{\ bzw\ }
	d=\frac{n_1x_1+n_2x_2+n_3x_3-w}{|\vec{n}|}\]}
	Ist der Abstand 0, so liegt der Punkt natürlich auf der Ebene.
