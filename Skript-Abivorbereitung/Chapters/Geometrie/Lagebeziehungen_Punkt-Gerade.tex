\subsection{Punkt - Gerade}
	\todo[color=green]{Kasten für Formel hinzufügen}
	\todo[inline,color=red]{Zusammenfassung-Kasten hinzufügen}
	\todo[inline,color=red]{Signalwörter-Kasten hinzufügen}
	Hier wollen wir den kleinsten Abstand von der Geraden zum Punkt
	berechnen\footnote{tatsächlich gibt es ja unendlich viele Abstände,
	entscheidend ist aber in der Mathematik immer der Kürzeste.}.\\
	Zunächst setzt ihr den Punkt in die Gerade ein, um zu überprüfen, ob dieser
	nicht sogar darauf liegt. Liegt er nicht dort, so müsst ihr zuerst eine
	Hilfsebene aufstellen. Der Richtungsvektor der Geraden entspricht dann dem
	Normalenvektor der Hilfsebene. Euer Punkt, der nicht auf der Geraden liegt, ist
	der Stützvektor der Ebene (somit stellt ihr sicher, dass ein rechter Winkel
	zwischen Gerade und der Strecke zum Punkt ist). Nun müsst ihr ermitteln, in
	welchem Punkt die Gerade die Hilfsebene durchstößt. Ab hier ist es lediglich
	eine \textbf{Punkt - Punkt} - Beziehung die ihr berechnen müsst.\\
	Das ganze ist natürlich recht aufwändig und nimmt viel Zeit in Anspruch.
	Glücklicherweise gibt es aber eine einfachere Variante. Sie mag nicht so
	ersichtlich sein, auf den ersten Blick. Letztendlich müsst ihr aber lediglich
	die Formel auswendig lernen. Auf den Beweis und den Gedanken dahinter, werden
	wir im Kurs näher eingehen. Wir setzen dort dann den angegebenen Punkt
	\(\vec{x}\) ein, so wie den Stützvektor \(\vec{a}\) und den Richtungsvektor
	\(\vec{b}\) und bekommen den Abstand d (ist d=0, so liegt der Punkt natürlich
	auf der Geraden)\footnote{Vor allem durch das Kreuzprodukt scheint das ganze
	recht komplex auszusehen. Letztendlich ist aber auch das schnell eingeübt.}:
	\formel{\[d=\frac{|(\vec{a}-\vec{x})\times \vec{b}|}{|\vec{b}|}\]}
