\subsection{Ebene - Ebene}
	\todo[color=purple]{Kasten für Formel hinzufügen}
	\todo[color=purple]{Zusammenfassung-Kasten hinzufügen}
	\todo[color=purple]{Signalwörter-Kasten hinzufügen}
	Auch hier gibt es potentiell 3 Möglichkeiten. Zunächst vergleichen wir die
	Normalenvektoren und schauen, ob diese linear abhängig sind.\\

	\(\star\) Sind sie es, so sind die Ebenen natürlich parallel zueinander. Hier
	wird dann der Abstand von einem Stützvektor der Ebenen zur anderen Ebene
	berechnet (Beziehung \textbf{Punkt - Ebene}). Ist dieser Abstand 0, so sind die
	beiden Ebenen identisch. Ist der Abstand nicht 0, so haben wir 2 parallele
	Ebenen mit dem Abstand d zueinander.\\

	\(\star\) Sind die Normalen nicht linear abhängig, so haben wir die dritte
	Möglichkeit. Die Ebenen schneiden sich. Um die Schnittgerade zu ermitteln,
	setzt man die Ebenen gleich und erhält ein LGS, welches man (mit einer
	Variablen, welche unser 's' ist) löst.