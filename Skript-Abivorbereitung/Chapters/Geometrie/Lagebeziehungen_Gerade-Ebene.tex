\subsection{Gerade - Ebene}
	\todo[color=green]{Kasten für Formel hinzufügen}
	\todo[inline,color=red]{Zusammenfassung-Kasten hinzufügen}
	\todo[inline,color=red]{Signalwörter-Kasten hinzufügen}
	Bei dieser Beziehung gibt es 3 Möglichkeiten, die vorkommen können. Um
	herauszufinden, welche vorliegt, ist zu empfehlen, zunächst Richtungsvektor und
	Normalenvektor zu vergleichen. Man skalar multipliziert einfach diese beiden
	Vektoren und betrachtet das Ergebnis.\\

	\(\star\) Ist das Skalarprodukt 0, so sind die Vektoren orthogonal, sowie
	Gerade und Ebene parallel. Nun berechnet man den Abstand des Stützvektors der
	Gerade zur Ebene (Beziehung \textbf{Punkt - Ebene}). Ist der Abstand 0, so
	liegt die Gerade auf der Ebene, ansonsten kennen wir dessen Abstand d.\\

	\(\star\) Ist das Skalarprodukt nicht 0, so wird die Gerade die Ebene
	durchstoßen. Diesen Punkt kann man herausfinden, indem man in der
	Ebenengleichung \(\vec{x}\) durch die Geradengleichung ersetzt und das LGS
	löst.
	\emph{Vorsicht gilt bei der Winkelbestimmung!} Hier gilt nämlich im Gegensatz
	zu den anderen Lagebeziehungen folgende Formel ( \(\vec{n}\) ist der
	Normalenvektor, \(\vec{a}\) der Richtungsvektor):
	\formel{\[cos(\alpha)=\frac{|\vec{n}\cdot \vec{a}|}{|\vec{n}|\cdot
	|\vec{a}|}\]}
