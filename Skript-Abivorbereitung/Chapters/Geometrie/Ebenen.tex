\section{Ebenen}
	\todo[color=purple]{Kasten für Formel hinzufügen}
	\todo[color=purple]{Zusammenfassung-Kasten hinzufügen}
	\todo[color=purple]{Signalwörter-Kasten hinzufügen}
	Ebenso wie Geraden, können wir im 3D-Raum auch Ebenen darstellen. Dafür gibt es
	drei unterschiedliche Möglichkeiten, die wir verwenden.

	\subsection{Parameterform}
		\todo[color=green]{Kasten für Formel hinzufügen}
		\todo[color=purple]{Zusammenfassung-Kasten hinzufügen}
		\todo[color=purple]{Signalwörter-Kasten hinzufügen}
		Um eine Ebene anhand von drei Punkten aufzustellen, ist die Parameterform am
		einfachsten. Sie ähnelt der Geradengleichung (ist also übersichtlicher auf dem
		ersten Blick) und die anderen beiden Formen lassen sich daraus leicht
		aufstellen (mit denen leichter weiter zu rechnen ist). Deshalb werden wir
		diese Form als Erstes zeigen, auch wenn man sie seltener nutzt. Zuerst haben
		wir wieder einen Stützvektor, wie bei der Geraden. Jedoch haben wir statt
		einem gleich zwei 'Richtungsvektoren', die Spannvektoren genannt werden (sie
		Spannen die Ebene auf). Diese berechnet man, indem man zwei Vektoren zwischen
		je zwei Punkten berechnet und eine skalare Variable mit ihnen multipliziert:
		\\ \\
		\formel{\[E:\ \vec{x}=\vec{a}+s\cdot \vec{b}+t\cdot \vec{c}\]}

	\subsection{Normalenform}
		\todo[color=green]{Kasten für Formel hinzufügen}
		\todo[color=purple]{Zusammenfassung-Kasten hinzufügen}
		\todo[color=purple]{Signalwörter-Kasten hinzufügen}
		Mit der Normalenform können wir leichter rechnen, vergleichen und Abstände
		bestimmen. Die Überlegung dahinter ist ein wenig umständlicher, bei näherer
		Betrachtung aber nicht zwingend schwerer. Zuerst müssen wir den Normalenvektor
		\(\vec{n}\) finden, welcher orthogonal auf der Ebene steht. Am einfachsten und
		schnellsten geht das, indem wir das Kreuzprodukt der Spannvektoren berechnen,
		wodurch wir direkt den Normalenvektor erhalten. Eine umständlichere Variante
		wäre, das Skalarprodukt der Spannvektoren mit dem Normalenvektor 0 zu setzen
		(aufgrund der Orthogonalität) und das lineare Gleichungssystem zu lösen.\\
		Nur wie kombinieren wir jetzt die Vektoren? Die Überlegung ist folgende: Wir
		ziehen von einem Punkt auf der Ebene, den wir kennen (der Stützvektor
		\(\vec{a}\)), den Punkt den wir überprüfen wollen (\(\vec{x}\)) ab und
		bekommen so einen Vektor, von einem Punkt zum anderen. Liegt dieser auch auf
		der Ebene, so ist dieser rechtwinklig zum Normalenvektor (der ja zu jedem
		Vektor auf der Ebene orthogonal ist). Ist das nicht der Fall, liegt der Punkt
		irgendwie anders im Raum, aber nicht auf der Ebene. Das heißt, wenn wir den
		neu berechneten Vektor mit dem Normalenvektor skalar multiplizieren und der
		Punkt liegt auf der Ebene, so ist das Ergebnis 0:
		\\ \\
		\formel{\[E:\ (\vec{a}-\vec{x})\cdot\vec{n}=0\]}
		\\ \\
		Wichtig hier zu erwähnen ist auch die hessesche Normalform. Diese
		unterscheidet sich lediglich durch den normierten Normalenvektor (dargestellt
		als \(\hat n\)). Normiert heißt hier, dass wir ihn durch die Länge teilen, er
		also nur noch die Länge 1 hat. Somit sieht die Hess'sche Normalenform um den
		Abstand d zu einem Punkt zu ermitteln, wie folgt aus:
		\\ \\
		\formel{\[E:\ (\vec{a}-\vec{x})\cdot
		\frac{1}{|\vec{n}|}\vec{n}=(\vec{a}-\vec{x})\cdot \hat{n}=d\]}

	\subsection{Koordinatenform}
		\todo[color=green]{Kasten für Formel hinzufügen}
		\todo[color=purple]{Zusammenfassung-Kasten hinzufügen}
		\todo[color=purple]{Signalwörter-Kasten hinzufügen}
		An der Koordinatenform gibt es nicht viel zu verstehen. Wir müssen nur lernen,
		sie aufzustellen. Letztendlich ist sie aber eine Umformung der Normalenform.
		Hierzu brauchen wir zuerst den Normalenvektor, welchen wir mit dem erst einmal
		unbekannten Vektor \(\vec{x}\) skalar multiplizieren\footnote{Einfacher
		gesagt, setzt einfach die entsprechende Zeile des Normalenvektors zu der
		entsprechenden x-Zeile ein, usw.}. Auf der rechten Seite der Gleichung steht
		dann die Zahl w, die raus kommt, wenn wir das Skalarprodukt einmal ausrechnen,
		indem wir den Stützvektor \(\vec{a}\) für \(\vec{x}\) einsetzen:
		\\ \\
		\formel{\[n_1x_1+n_2x_2+n_3x_3=w \mathrm{\ (anders\ geschrieben\ }
		\vec{n}\cdot \vec{x}=w)\]}
		\\ \\
		Analog zur Hesseschen Normalform ist es auch möglich, die
		Koordinatenform anzugeben, indem man zuerst w abzieht und dann den gesamten
		Term durch die Länge des Normalenvektors teilt. So bekommt man den Abstand d
		der Ebene zum Punkt \(\vec{x}\):
		\\ \\
		\formel{\[\frac{n_1x_1+n_2x_2+n_3x_3-w}{|\vec{n}|}=d\]}
