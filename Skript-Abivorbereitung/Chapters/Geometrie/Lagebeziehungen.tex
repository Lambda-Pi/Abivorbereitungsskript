\section{Lagebeziehungen}
	\todo[color=purple]{Kasten für Formel hinzufügen}
	\todo[inline,color=red]{Zusammenfassung-Kasten hinzufügen}
	\todo[inline,color=red]{Signalwörter-Kasten hinzufügen}
	Zuletzt wollen wir noch die unterschiedlichen Lagebeziehungen besprechen.
	Manchmal wird explizit nach ihnen gefragt, manchmal brauchen wir sie aber auch,
	um im Wahlteil Anwendungsaufgaben beantworten zu können\footnote{wir werden
	hier nicht explizit auf die Erklärungen für Anwendungsaufgaben eingehen. Zum
	einen, weil es aus typografischen Gründen keinen Sinn macht und wir euch zum
	anderen nicht dadurch verwirren wollen. Im Kurs werden wir aber darauf
	eingehen.}. Die hier genannten Verfahren sind nicht die Einzigen, denn nicht
	nur ein Weg führt zur Lösung. Die hier besprochenen Wege sind unserer Meinung
	nach die Einfachsten. Wenn ihr mit eurer eigenen Methode aber besser zurecht
	kommt, ist es weniger sinnig, ausgerechnet unsere zu benutzen. Im Kurs werden
	wir bei Bedarf auch auf andere Wege eingehen.

	% Punkt - Punkt
	\subsection{Punkt - Punkt}
\todo[color=red]{Kasten für Formel hinzufügen}
\todo[color=red]{Zusammenfassung-Kasten hinzufügen}
\todo[color=red]{Signalwörter-Kasten hinzufügen}
Diese Beziehung dürfte klar sein. Entweder haben wir den gleichen Punkt im Raum (was sofort erkennbar wäre) oder aber sie liegen örtlich auseinander. Dann kann man angeben, wie weit die Punkte entfernt sind, indem man einen Punkt vom anderen abzieht (Ziel minus Angriff) und von diesem neuen Vektor den Betrag berechnet, um zu berechnen, um den Abstand zu bestimmen.


	% Punkt - Gerade
	\subsection{Punkt - Gerade}
	\todo[color=green]{Kasten für Formel hinzufügen}
	\todo[inline,color=red]{Zusammenfassung-Kasten hinzufügen}
	\todo[inline,color=red]{Signalwörter-Kasten hinzufügen}
	Hier wollen wir den kleinsten Abstand von der Geraden zum Punkt
	berechnen\footnote{tatsächlich gibt es ja unendlich viele Abstände,
	entscheidend ist aber in der Mathematik immer der Kürzeste.}.\\
	Zunächst setzt ihr den Punkt in die Gerade ein, um zu überprüfen, ob dieser
	nicht sogar darauf liegt. Liegt er nicht dort, so müsst ihr zuerst eine
	Hilfsebene aufstellen. Der Richtungsvektor der Geraden entspricht dann dem
	Normalenvektor der Hilfsebene. Euer Punkt, der nicht auf der Geraden liegt, ist
	der Stützvektor der Ebene (somit stellt ihr sicher, dass ein rechter Winkel
	zwischen Gerade und der Strecke zum Punkt ist). Nun müsst ihr ermitteln, in
	welchem Punkt die Gerade die Hilfsebene durchstößt. Ab hier ist es lediglich
	eine \textbf{Punkt - Punkt} - Beziehung die ihr berechnen müsst.\\
	Das ganze ist natürlich recht aufwändig und nimmt viel Zeit in Anspruch.
	Glücklicherweise gibt es aber eine einfachere Variante. Sie mag nicht so
	ersichtlich sein, auf den ersten Blick. Letztendlich müsst ihr aber lediglich
	die Formel auswendig lernen. Auf den Beweis und den Gedanken dahinter, werden
	wir im Kurs näher eingehen. Wir setzen dort dann den angegebenen Punkt
	\(\vec{x}\) ein, so wie den Stützvektor \(\vec{a}\) und den Richtungsvektor
	\(\vec{b}\) und bekommen den Abstand d (ist d=0, so liegt der Punkt natürlich
	auf der Geraden)\footnote{Vor allem durch das Kreuzprodukt scheint das ganze
	recht komplex auszusehen. Letztendlich ist aber auch das schnell eingeübt.}:
	\formel{\[d=\frac{|(\vec{a}-\vec{x})\times \vec{b}|}{|\vec{b}|}\]}


	% Punkt - Ebene
	\subsection{Punkt - Ebene}
\todo[color=red]{Kasten für Formel hinzufügen}
\todo[color=red]{Zusammenfassung-Kasten hinzufügen}
\todo[color=red]{Signalwörter-Kasten hinzufügen}
Den Abstand zwischen einem Punkt und einer Ebene berechnet man mit der Hesseschen Normalform / Koordinatenform. Ihr müsst für \(\vec{x}\) einfach den angegebenen Punkt (dessen Abstand ihr zur Ebene ermitteln wollt) einsetzen und habt sofort den Abstand. Zur Wiederholung:
\[d=(\vec{a}-\vec{x})\cdot \hat{n} \mathrm{\ bzw\ } d=\frac{n_1x_1+n_2x_2+n_3x_3-w}{|\vec{n}|}\]
Ist der Abstand 0, so liegt der Punkt natürlich auf der Ebene.


	% Gerade - Gerade
	\subsection{Gerade - Gerade}
	\todo[color=red]{Kasten für Formel hinzufügen}
	\todo[color=red]{Zusammenfassung-Kasten hinzufügen}
	\todo[color=red]{Signalwörter-Kasten hinzufügen}
	Beim Betrachten der Lage zwischen zwei Geraden gibt es vier unterschiedliche
	Möglichkeiten, die wir prüfen müssen. Wir empfehlen euch, zuerst die
	Richtungsvektoren zu betrachten und dann zu prüfen, ob diese linear abhängig
	sind oder nicht.\\

	\(\star\) Ist dies der Fall, so können sie entweder parallel sein oder liegen
	sogar aufeinander. Um das herauszufinden, ermittelt ihr einfach den Abstand von
	einer Geraden zu dem Stützpunkt der anderen Geraden (Beziehung \textbf{Punkt -
	Gerade}). Ist der Abstand 0, so sind die Geraden identisch, ansonsten sind sie
	parallel und haben den Abstand d.\\

	\(\star\) Sind die Richtungsvektoren nicht linear abhängig, so sind sie
	windschief oder schneiden sich an einem Punkt. Hierzu setzt man zuerst beide
	Geraden gleich und löst das LGS. Geht es auf, so schneiden sie sich in diesem
	Punkt (setzt das s oder t bitte in die entsprechende Gerade ein, um zu schauen
	an welchen Punkt sie sich schneiden und gebt diesen an).\\
	Schneiden sich die beiden Geraden nicht, so sind sie windschief. Dann ist
	wieder der kürzeste Abstand zwischen den Geraden anzugeben. Hier müsst ihr
	wieder eine Hilfsebene aufstellen. Diesmal sind die Richtungsvektoren der
	beiden Geraden die Spannvektoren der Hilfsebene (siehe \textbf{Parameterform}).
	Aus ihnen könnt ihr sofort die Normalen- oder Koordinatenform mit dem
	Normalenvektor angeben. Euer Punkt auf der Ebene ist dann einer der
	Stützvektoren. Der Abstand der Ebene zum Stützvektor (Beziehung \textbf{Punkt -
	Ebene}) der anderen Geraden ist dann der kleinste Abstand zwischen den
	windschiefen Geraden.


	% Gerade - Ebene
	\subsection{Gerade - Ebene}
	\todo[color=green]{Kasten für Formel hinzufügen}
	\todo[color=green]{Zusammenfassung-Kasten hinzufügen}
	\todo[color=purple]{Signalwörter-Kasten hinzufügen}
	Bei dieser Beziehung gibt es 3 Möglichkeiten, die vorkommen können. Um
	herauszufinden, welche vorliegt, ist zu empfehlen, zunächst Richtungsvektor und
	Normalenvektor zu vergleichen. Man skalar multipliziert einfach diese beiden
	Vektoren und betrachtet das Ergebnis.\\

	\(\star\) Ist das Skalarprodukt 0, so sind die Vektoren orthogonal, sowie
	Gerade und Ebene parallel. Nun berechnet man den Abstand des Stützvektors der
	Gerade zur Ebene (Beziehung \textbf{Punkt - Ebene}). Ist der Abstand 0, so
	liegt die Gerade auf der Ebene, ansonsten kennen wir dessen Abstand d.\\

	\(\star\) Ist das Skalarprodukt nicht 0, so wird die Gerade die Ebene
	durchstoßen. Diesen Punkt kann man herausfinden, indem man in der
	Ebenengleichung \(\vec{x}\) durch die Geradengleichung ersetzt und das LGS
	löst.
	\emph{Vorsicht gilt bei der Winkelbestimmung!} Hier gilt nämlich im Gegensatz
	zu den anderen Lagebeziehungen folgende Formel ( \(\vec{n}\) ist der
	Normalenvektor, \(\vec{a}\) der Richtungsvektor):
	\formel{\[cos(\alpha)=\frac{|\vec{n}\cdot \vec{a}|}{|\vec{n}|\cdot
	|\vec{a}|}\]}
	
	\summary{
		Skalarprodukt des Richtungsvektors der Geraden und des Normalenvektors der
		Ebene bilden:
		\begin{itemize}
		  \item Skalarprodukt ist 0, Gerade ist also parallel zur Ebene.
		  \begin{enumerate}
		    \item Abstand von der Ebene zum Stützvektor der Geraden ermitteln.
		    \item Die Gerade liegt mit diesem Abstand entfernt, parallel zur Ebene. 
		  \end{enumerate}
		  \item Skalarprodukt ist nicht 0, Gerade schneidet also Ebene.
		  \begin{enumerate}
		    \item Geradengleichung in Ebene einsetzen.
		    \item LGS lösen.
		    \item Das Ergebnis ist der Schnittpunkt zwischen Ebene und Gerade.
		  \end{enumerate}
		\end{itemize}
	}


	% Ebene - Ebene
	\subsection{Ebene - Ebene}
	\todo[color=red]{Kasten für Formel hinzufügen}
	\todo[color=red]{Zusammenfassung-Kasten hinzufügen}
	\todo[color=red]{Signalwörter-Kasten hinzufügen}
	Auch hier gibt es potentiell 3 Möglichkeiten. Zunächst vergleichen wir die
	Normalenvektoren und schauen, ob diese linear abhängig sind.\\

	\(\star\) Sind sie es, so sind die Ebenen natürlich parallel zueinander. Hier
	wird dann der Abstand von einem Stützvektor der Ebenen zur anderen Ebene
	berechnet (Beziehung \textbf{Punkt - Ebene}). Ist dieser Abstand 0, so sind die
	beiden Ebenen identisch. Ist der Abstand nicht 0, so haben wir 2 parallele
	Ebenen mit dem Abstand d zueinander.\\

	\(\star\) Sind die Normalen nicht linear abhängig, so haben wir die dritte
	Möglichkeit. Die Ebenen schneiden sich. Um die Schnittgerade zu ermitteln,
	setzt man die Ebenen gleich und erhält ein LGS, welches man (mit einer
	Variablen, welche unser 's' ist) löst.
