\subsection{Rechnen mit Vektoren}
Da die Vektoren erst neu dazu kamen, sollten wir zunächst noch einmal kurz auf die Grundrechenarten eingehen und wie diese bei Vektoren angewendet werden.
\subsubsection{Addition}
\todo[color=red]{Kasten für Formel hinzufügen}
\todo[color=red]{Zusammenfassung-Kasten hinzufügen}
\todo[color=red]{Signalwörter-Kasten hinzufügen}
Die Addition von Vektoren ist recht simpel. Wir addieren einfach die einzelnen Zeilen zueinander:
\[\vec{a}+\vec{b}=
\begin{pmatrix}
 a_1\\
 a_2\\
 a_3
\end{pmatrix}+\begin{pmatrix}
 b_1\\
 b_2\\
 b_3
\end{pmatrix}
=
\begin{pmatrix}
 a_1+b_1\\
 a_2+b_2\\
 a_3+b_3
\end{pmatrix} \]
Hier wird vielleicht auch ersichtlich, wie sich die Richtung eines Vektors zusammensetzt, wenn man ihn wie folgt umschreibt:
\[\vec{a}=
\begin{pmatrix}
 a_1\\
 a_2\\
 a_3
\end{pmatrix}=
\begin{pmatrix}
 a_1\\
 0\\
 0
\end{pmatrix} +
\begin{pmatrix}
 0\\
 a_2\\
 0
\end{pmatrix} +
\begin{pmatrix}
 0\\
 0\\
 a_3
\end{pmatrix}
\]
\subsubsection{Subtraktion}
\todo[color=red]{Kasten für Formel hinzufügen}
\todo[color=red]{Zusammenfassung-Kasten hinzufügen}
\todo[color=red]{Signalwörter-Kasten hinzufügen}
Entsprechend kann man auch zwei Vektoren voneinander abziehen. Die Subtraktion erfolgt analog zur Addition. Damit kann man auch den Vektor zwischen zwei Punkten berechnen (Ziel minus Angriff):
\[\vec{a}-\vec{b}=\begin{pmatrix}
 a_1-b_1\\
 a_2-b_2\\
 a_3-b_3
\end{pmatrix}\]
\subsubsection{skalare Faktoren}
\todo[color=red]{Kasten für Formel hinzufügen}
\todo[color=red]{Zusammenfassung-Kasten hinzufügen}
\todo[color=red]{Signalwörter-Kasten hinzufügen}
Mit einem Faktor kann man einen Vektor ums c-Fache vergrößern (oder Verkleinern), bzw. mit einem negativen Faktor die Richtung des Vektors auch um \(180^{\circ}\) drehen. Der Pfeil auf der Linie ist dann einfach auf der anderen Seite. Steht der skalare Faktor vor (oder auch hinter) dem Vektor, so ist das das Gleiche, wie wenn man das Skalar mit jeder Zeile multipliziert:
\[c\cdot \vec{a}=
\begin{pmatrix}
 c\cdot a_1\\
 c\cdot a_2\\
 c\cdot a_3
\end{pmatrix}
\]
\subsubsection{Betrag (Länge) eines Vektors}
\todo[color=red]{Kasten für Formel hinzufügen}
\todo[color=red]{Zusammenfassung-Kasten hinzufügen}
\todo[color=red]{Signalwörter-Kasten hinzufügen}
Will man die Länge eines Vektors ermitteln, so bekommt man ein Skalar mit eben dieser Länge. Man berechnet diesen, indem man den Satz des Pythagoras anwendet, allerdings für 3 Dimensionen. Das sieht dann wie folgt aus:
\[|\vec{a}|=\sqrt{a_1^2+a_2^2+a_3^2}\]
\subsubsection{Skalarprodukt}
\todo[color=red]{Kasten für Formel hinzufügen}
\todo[color=red]{Zusammenfassung-Kasten hinzufügen}
\todo[color=red]{Signalwörter-Kasten hinzufügen}
Eine Art des Produkts bei Vektoren ist das Skalarprodukt. Man multipliziert hier die Länge des Vektors, auf den anderen projektiert, mit der kompletten Länge des anderen Vektors. Wie man sich das genauer vorstellen kann, findet man im Internet oder hier in unserem Kurs ;). Man berechnet das Skalarprodukt wie folgt:
\[\vec{a}\cdot \vec{b}=a_1\cdot b_1+a_2\cdot b_2+a_3\cdot b_3\]
Somit haben wir aus zwei Vektoren ein Skalar (eine Zahl) gemacht. Welche Bedeutung dies hat sehen wir, wenn wir uns die Definition genauer anschauen. Mathematisch sieht diese wie folgt aus:
\[\vec{a}\cdot \vec{b}=|\vec{a}|\cdot |\vec{b}|\cdot cos(\varphi)\]
Mit dieser Relation kann man die Projektion erkennen und sieht auch, dass zwei orthogonale Vektoren das Skalarprodukt 0 haben. Mit dieser kann man jedoch auch den Winkel zwischen zwei Vektoren berechnen. Zwar ist dies recht leicht umzustellen, da dies aber eine wichtige Relation ist, werden wir sie noch einmal umgestellt aufschreiben\footnote{Beachtet bitte, dass es immer 2 Winkel gibt. Beide zusammen ergeben \(180^{\circ}\). Man gibt aber immer den kleineren Winkel an. Bekommt ihr also einen Winkel der größer ist als \(90^{\circ}\), so ist euer richtiger Winkel einfach: \(\alpha=180^{\circ}-\beta\)}:
\[cos(\varphi)=\frac{ | \vec{a}\cdot \vec{b} | }{|\vec{a}|\cdot |\vec{b}|}\]
\subsubsection{Kreuzprodukt}
\todo[color=red]{Kasten für Formel hinzufügen}
\todo[color=red]{Zusammenfassung-Kasten hinzufügen}
\todo[color=red]{Signalwörter-Kasten hinzufügen}
Das Kreuzprodukt ist in den meisten Klassen kein Unterrichtsstoff und wird auch nicht in der Prüfung abgefragt. Trotzdem wollen wir es hier ansprechen, da dies eine einfache Art ist, einen Normalenvektor (später bei den Ebenen) zu bestimmen und bei der Prüfung darf dieser auch verwendet werden. Das Kreuzprodukt ergibt nämlich immer einen Vektor, der zu den zwei anderen rechtwinklig steht, was der Definition eines Normalenvektors entspricht.\\
Ihn aufzustellen, mag bei den ersten Versuchen ein wenig kompliziert erscheinen. Das zu üben lohnt sich jedoch erheblich. In die erste Zeile des Kreuzprodukts (welcher ein Vektor ist) schreibt man \(a_2\cdot b_3-a_3\cdot b_2\). Nun schreibt man in den weiteren Zeilen jeweils den Buchstaben ab und addiert immer 1 auf die Indizes (aus einer 3 wird dann eine 1!):
\[\vec{a} \times \vec{b}=
\begin{pmatrix}
 a_2b_3-a_3b_2\\
 a_3b_1-a_1b_3\\
 a_1b_2-a_2b_1
\end{pmatrix}\]
Es gibt auch noch eine weitere Methode, die wir hier aber nicht schriftlich festhalten wollen. Diese besprechen wir dann im Kurs selbst.

\subsubsection{Lineare Abhängigkeit}
\todo[color=red]{Kasten für Formel hinzufügen}
\todo[color=red]{Zusammenfassung-Kasten hinzufügen}
\todo[color=red]{Signalwörter-Kasten hinzufügen}
Für das Abitur braucht ihr die Definition der linearen Abhängigkeit nicht mehr komplett zu kennen. Jedoch solltet ihr die Frage, ob zwei Vektoren linear abhängig sind, beantworten können (die Definition befasst sich auch mit mehreren Vektoren). Entscheidend wird dies, wenn wir Geraden oder Ebenen vergleichen wollen. Können wir einen Vektor durch einen anderen mit einem konstanten Faktor k multipliziert darstellen, sind diese linear abhängig. Bei 
\(k\cdot \begin{pmatrix}
 1\\
 -2\\
 4
\end{pmatrix}=\begin{pmatrix}
 -4\\
 8\\
 8
\end{pmatrix}\)
zum Beispiel überlegt man zunächst, wie man von 1 auf -4 kommt. Das gilt wenn k=-4 ist. Überprüft man das nun, so stimmt mit diesem k auch die zweite Zeile. Die dritte Zeile macht uns dann aber einen Strich durch die Rechnung, so dass diese Vektoren linear unabhängig sind (wäre die letzte Zeile -16 so würde es passen). Linear abhängig sind zwei Vektoren, wenn sich ein k finden lässt, sodass gilt:
\[\vec{a}=k\cdot \vec{b}\]
