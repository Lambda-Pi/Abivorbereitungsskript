\subsection{Tangenten \& Normale}
	\todo[color=green]{Kasten für Formel hinzufügen}
	\todo[color=green]{Zusammenfassung-Kasten hinzufügen}
	\todo[color=green]{Signalwörter-Kasten hinzufügen}
	Öfter kann es vorkommen, dass die Tangente, also die Gerade, die an der
	Funktion an einer Stelle anliegt, berechnet werden soll. Wie stellen wir das
	nun an? Am einfachsten und schnellsten ist die folgende Variante: Wir kennen
	die allgemeine Form einer Geraden und wie man eine Funktion verschiebt. Auch
	wissen wir, dass die erste Ableitung die Steigung der Funktion an dieser Stelle
	\(x_0\) ist. Das können wir nun anwenden, um die Tangente der Funktion am Punkt
	\(P(x_0|f(x_0))\) zu bestimmen. Wir setzen in unsere allgemeine
	Geradengleichung also die Steigung an dem Punkt ein und verschieben sie noch um
	die entsprechenden Werte in x- \& y -Richtung. Somit bekommen wir auch schon
	die Tangentengleichung:
	\formel{\[t(x)=f'(x_0)\cdot (x-x_0)+f(x_0)\]}
	\(x_0\) ist hier eine Konstante, das x die Variable.\\
	Die Normale steht senkrecht zur Tangente. Wie man die Steigung dieser
	berechnet, haben wir bereits gesehen. Die Verschiebung geht zum gleichen Punkt.
	Also ist die Normalengleichung:
	\formel{\[n(x)=-\frac{1}{f'(x_0)}\cdot (x-x_0)+f(x_0)\]}
	
	\summary{
		Um die Tangente oder Normale anzugeben muss man:
		\begin{enumerate}
		  \item die Stelle \(x_0\), den Funktionswert an der Stelle \(f(x_0)\) und die
		  erste Ableitung der Funktion an der Stelle \(f'(x_0)\) berechnen und
		  aufschreiben.
		  \item die Werte in die jeweilige Funktion einsetzen.
		  \item gegebenfalls (vor allem wenn man damit weiter rechnet) das
		  Distributivgesetz anwenden und zusammenfassen.
		\end{enumerate}
	}
	
	\tags{
		Gerade die Funktion "`berührt"', Gerade die Funktion rechtwinklig schneidet
	}