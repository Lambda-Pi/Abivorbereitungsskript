\subsection{Anwendungen des Integrals}
	\todo[color=purple]{Kasten für Formel hinzufügen}
	\todo[color=purple]{Zusammenfassung-Kasten hinzufügen}
	\todo[color=purple]{Signalwörter-Kasten hinzufügen}
	Hier möchten wir noch einmal darauf eingehen, wie man die Integrale verwenden
	kann. Dabei gehen wir auf drei Arten ein:

	\subsubsection{Flächeninhalte}
		\todo[color=purple]{Kasten für Formel hinzufügen}
		\todo[color=purple]{Zusammenfassung-Kasten hinzufügen}
		\todo[color=purple]{Signalwörter-Kasten hinzufügen}
		Da man das Integral über die Flächeninhalte definieren kann, ist es natürlich
		auch logisch, dass man damit Flächeninhalte berechnen kann. Und zwar immer
		den Flächeninhalt zwischen der Funktion und der x-Achse. Allerdings können diese Flächeninhalte Vorzeichen haben und sich somit auch gegenseitig aufheben. Das wollen wir dann allerdings nicht. 
		Das Problem löst man ganz einfach, indem man die Nullstellen findet und immer
		von Nullstelle zu Nullstelle integriert, von den einzelnen Integralen den
		Betrag nimmt, falls nötig und die Beträge anschließend aufsummiert. Ein
		Beispiel hierzu:
		\[f(x)=x^2-4\textrm{, mit den Nullstellen }x_0=\pm2\ =>\ \int\limits_0^4
		|f(x)|\ dx=|\int\limits_0^2 f(x)\ dx|+|\int\limits_2^4 f(x)\ dx|\]
		Wollen wir den Flächeninhalt zwischen zwei Funktionen, so ziehen wir einfach
		die eine Funktion von der anderen ab und Integrieren dann (also wir Integrieren dann h(x)=f(x)-g(x)).

	\subsubsection{Rekonstruierter Bestand}
		\todo[color=purple]{Kasten für Formel hinzufügen}
		\todo[color=purple]{Zusammenfassung-Kasten hinzufügen}
		\todo[color=purple]{Signalwörter-Kasten hinzufügen}
		Wie schon bei den Anwendungsaufgaben für Ableitungen gezeigt, kann man
		jegliche Form von Geschwindigkeiten als Ableitung angeben. Haben wir also nun
		eine Geschwindigkeit als Funktion angegeben und integrieren diese, so können
		wir den eigentlichen Bestand rekonstruieren. Oder zu deutsch, wir schauen,
		welche Strecke bis zu diesem Zeitpunkt zurückgelegt wurde (bzw. wie viel Liter
		eingelassen wurden oder allgemein immer der obere Teil der Einheit). Hier wird
		dann der Teil angegeben, der vom Anfangswert dazu kam oder weg ging. Hier
		müsst ihr aber beachten, dass ihr das Integral nicht aufsplitten dürft, wie
		beim Berechnen von Flächeninhalten. Der negative Teil wird dann zwar
		mitberechnet, dass macht bei näherer Überlegung aber auch Sinn. Ist die
		Geschwindigkeit negativ, so führt das Fahrzeug auch in negative Richtung.
		Genau das sagt uns der negative 'Flächeninhalt'.\\
		Hierzu pasiert es bei schwereren Aufgaben, dass man Funktionen kombinieren
		muss. Beispielsweise soll der Abstand zwischen zwei sich unterschiedlich
		bewegenden Fahrzeugen untersucht werden, deren Geschwindigkeiten wir kennen.
		Am Einfachsten ist es, die Geschwindigkeiten voneinander abzuziehen, wodurch
		wir eine neue Funktion erhalten und diese dann integrieren können. Das ist so,
		weil wir die relative Geschwindigkeit betrachten (ähnlich wird auch der
		Flächeninhalt zwischen zwei Schaubildern betrachtet).

	\subsubsection{Mittelwert}
		\todo[color=purple]{Zusammenfassung-Kasten hinzufügen}
		\todo[color=purple]{Signalwörter-Kasten hinzufügen}
		Mit dem Integral können wir auch den Mittelwert \(\overline{m}\) der
		eigentlichen Funktion f(x) in einem Intervall von a nach b berechnen. Das
		sieht dann folgendermaßen aus:
		\\ \\
		\formel{\[\overline{m}=\frac{1}{b-a}\cdot \int\limits_a^b f(x)\]}
