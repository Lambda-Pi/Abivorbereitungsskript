\subsection{Anwendung von Ableitungen}
	\todo[color=purple]{Kasten für Formel hinzufügen}
	\todo[color=green]{Zusammenfassung-Kasten hinzufügen}
	\todo[color=green]{Signalwörter-Kasten hinzufügen}
	Ableitungen sind letztendlich ein Instrument, mit dem wir Eigenschaften von
	Funktionen überprüfen können oder mit denen wir Eigenschaften beschreiben und
	in die Mathematik übersetzen können.\\
	
	\(\star\) Ein Beispiel ist die Änderungsrate. Die erste Ableitung spiegelt
	immer eine Änderungsrate der Funktion wieder, gibt also an, um wie viel sich
	die Funktion an einer bestimmten Stelle verändert.
	Machen wir uns das mal an einem Beispiel der Physik klar. Haben wir zum
	Beispiel den Ort eines Gegenstands in Abhängigkeit von der Zeit gegeben
	(s(t)=\ldots) und wir wollen wissen, wie sich der Ort mit der Zeit verändert,
	so leitet man nach der Zeit ab. Somit haben wir die Geschwindigkeit
	\(\frac{\Delta s}{\Delta t}\)(wobei \(\Delta\) gegen 0 geht), was eben angibt,
	um welche Strecke sich unser Gegenstand in einer Sekunde bewegt hat (erkennbar
	auch an den Einheiten ;) ). Genau so verhält sich das mit "Liter pro Zeit"\
	oder allem anderen, was einen Bruch als Einheit besitzt.\\

	\(\star\) Ableitungen können auch verwendet werden, um geometrische
	Informationen in die Sprache der Mathematik umzuwandeln. Folgendes Beispiel
	habt ihr vielleicht schon durchgerechnet. Man hat eine Funktion gegeben, die
	den Querschnitt eines Tals zwischen zwei Bergen beschreibt. Jetzt wissen wir an
	welchen Punkt die Sonne anfängt und der Schatten aufhört (natürlich ein Punkt
	auf der Funktion, also dem Querschnitt) und wir wollen wissen, in welchem
	Winkel die Sonne gerade zur x-Achse steht. Was wir jetzt noch (durch
	Überlegungen) wissen sollten ist, dass die Sonne den Berg tangential streift.
	Also brauchen wir die Tangente (an einer noch unbekannten Stelle). Die Steigung
	an der Stelle kann man dann noch als Winkel umrechnen (dazu kommen wir später
	noch). Wie man das ganze konkret berechnet, gehen wir im Kurs selbst durch.\\

	\(\star\) Eine weitere wichtige Möglichkeit zur Anwendung von Ableitungen ist
	die Berechnung von Extrem- \& Wendepunkten.
	
	\summary{
		Ableitungen werden verwendet um
		\begin{itemize}
		  \item Änderungsraten zu bestimmen.
		  \item geometrische Informationen in die Sprache der Mathematik umzuwandeln.
		  \item Extrem- \& Wendepunkte zu berechnen
		\end{itemize}
	}
	
	\tags{
		Ableitung, Änderungsrate, Extempunkte, Wendepunkte,
		\(\frac{\textit{Einheit}}{\textit{Andere Einheit}}\), tangiert, orthogonal /
		Normale 
	}
	

	\subsubsection{Extrempunkte}
		\todo[color=purple]{Kasten für Formel hinzufügen}
		\todo[color=green]{Zusammenfassung-Kasten hinzufügen}
		\todo[color=green]{Signalwörter-Kasten hinzufügen}
		Um diese zu bestimmen,  brauchen wir die ersten beiden Ableitungen. Zuerst
		bestimmt man die Nullstellen der ersten Ableitung, denn an Extrempunkten haben
		Funktionen immer die Steigung 0. Somit haben wir schon mal Stellen, an denen
		Extrempunkte sein \textbf{können}. Die Stellen (x-Werte) setzen wir dann in
		die zweite Ableitung ein, um zu schauen, ob es sich um einen Hoch- oder
		Tiefpunkt handelt. Haben wir in der zweiten Ableitung einen positiven Wert, so
		liegt ein Tiefpunkt vor. Bei einem negativem Wert haben wir einen Hochpunkt
		der ursprünglichen Funktion. Als Eselsbrücke kann man Smilies nehmen. Haben
		wir einen positiven Wert, so malen wir einen glücklichen Smilie, dessen Mund
		einen Tiefpunkt bildet. Bei einem negativen Wert malen wir einen traurigen
		Smilie, dessen Mund einen Hochpunkt besitzt. Weiter muss man den Punkt
		berechnen, an dem der Extremwert ist. Die Stelle \(x_0\) kennen wir ja bereits
		und müssen diese nun in die eigentliche Funktion f(x) einsetzen. Somit
		bekommen wir den Extrempunkt \(P(x_0|f(x_0))\).\\
		Ein Problem ist es, wenn die zweite Ableitung 0 ergibt. Dann setzt man einfach
		in die erste Ableitung ein x kleiner als die Stelle ein und ein x, ein
		bisschen größer als diese Stelle, um die Steigung links und rechts von dem
		Extrempunkt zu ermitteln. \\
		Ist die Steigung erst positiv, dann negativ, so haben wir einen Hochpunkt,
		umgekehrt einen Tiefpunkt. Sind beide Seiten positiv oder beide negativ, so
		haben wir einen Sattelpunkt. Diese Variante ist mit Nachdenken verbunden und
		man muss verstanden haben, wieso das so ist.\\
		Noch ein kurzes Beispiel zu Extrempunktberechnungen im allgemeinen:
		Untersuchen wir die Funktion \(f(x)=\frac{1}{4}x^4+x^2\) auf Extremstellen.
		Die ersten zwei Ableitungen sind \(f'(x)=x^3+2x,\ f''(x)=3x^2+2\). Eine der
		potentiellen Extremstellen ist 0 (denn f'(0)=0). Eingesetzt in die zweite
		Ableitung ergibt f''(0)=2>0, also haben wir einen Tiefpunkt vorliegen.
		
		\summary{
			Um einen Extrempunkt zu bestimmen muss man
			\begin{enumerate}
			  \item erste und zweite Ableitung bilden.
			  \item erste Ableitung 0 setzen und alle Lösungen für x finden.
			  \item die Stellen (Lösungen aus dem vorherhigen Punkt) in die 2-te
			  Ableitung einsetzen:
			  \begin{itemize}
			    \item \(f''(x_i)<0\): es handelt sich um einen Hochpunkt
			    \item \(f''(x_i)>0\): es handelt sich um einen Tiefpunkt
			    \item \(f''(x_i)=0\): es ist nicht sicher um was für ein Extrema es sich
			    handelt \(\Rightarrow\) Folgendes ist zu tun für jede gefundene Stelle:
			    \begin{enumerate}
			      \item eine Stelle vor und eine hinter der gefundenen Stelle in die 1-te
			      Ableitung einsetzen.
			      \item Vorzeichenwechsel betrachten:
			      \begin{itemize}
			        \item von + nach -: Hochpunkt
			        \item von - nach +: Tiefpunkt
			        \item von - nach - oder von + nach +: Sattelpunkt
			      \end{itemize}
			    \end{enumerate}
			  \end{itemize}
			  \item Die Stellen in die ursprüngliche Funktion f(x) einsetzen um die
			  Funktionswerte zu bekommen
			  \item Die Tief-, Hoch- und Sattelpunkte aufschreiben
			\end{enumerate}
		}
		
		\tags{
			Hoch- / Tiefpunkt, höchster / tiefster (Funktionswert), lokales Minima /
			Maxima
		}

	\subsubsection{Wendepunkte}
		\todo[color=purple]{Kasten für Formel hinzufügen}
		\todo[color=green]{Zusammenfassung-Kasten hinzufügen}
		\todo[color=purple]{Signalwörter-Kasten hinzufügen}
		Wendestellen / -punkte berechnet man ganz ähnlich wie Extrempunkte. Allerdings
		berechnen wir hier die Extrempunkte der ersten Ableitung. Die Nullstelle der
		zweiten Ableitung, eingesetzt in die dritte Ableitung, gibt uns die
		Richtungsänderung. Ändert sich die Kurve der Funktion von links nach rechts,
		haben wir bei der dritten Ableitung einen negativen Wert (\(f'''(x_w)<0)\),
		also der Hochpunkt der ersten Ableitung) und umgekehrt.
		
		\summary{
			Um Wendepunkte zu bestimmen muss man
			\begin{enumerate}
			  \item zweite und dritte Ableitung bilden.
			  \item zweite Ableitung 0 setzen und alle Lösungen für x finden.
			  \item die Stellen (Lösungen aus dem vorherhigen Punkt) in die 2-te
			  Ableitung einsetzen:
			  \begin{itemize}
			    \item \(f'''(x_i)<0\) oder \(f'''(x_i)>0\): es handelt sich um eine
			    Wendestelle
			    \item \(f'''(x_i)=0\) es kann nicht gesagt werden ob es sich um eine
			    Wendestelle handelt. Folgendes ist zu tun für jede gefundene Stelle:
			    \begin{enumerate}
			      \item eine Stelle vor und eine hinter der gefundenen Stelle in die 2-te
			      Ableitung einsetzen.
			      \item findet ein Vorzeichenwechsel statt: es handelt sich um eine
			      Wendestelle, findet keiner statt handelt es sich nicht um eine
			      Wendestelle.
			    \end{enumerate}
			  \end{itemize}
			  \item Die Stellen in die ursprüngliche Funktion f(x) einsetzen um die
			  Funktionswerte zu bekommen.
			  \item Die Wendepunkte aufschreiben.
			\end{enumerate}
		}
	
	\subsubsection{Winkel zwischen Funktionen und Achsen}
		\todo[color=green]{Kasten für Formel hinzufügen}
		\todo[color=green]{Zusammenfassung-Kasten hinzufügen}
		\todo[color=green]{Signalwörter-Kasten hinzufügen}
		\(\star\) Um den Winkel zwischen einer Funktion und der x-Achse zu berechnen,
		brauchen wir zunächst die erste Ableitung, um so die Steigung an der
		entsprechenden Stelle zu bekommen. Mit einer Skizze kann man sich nun
		herleiten, wie man den Winkel berechnet. Zeichnen wir eine Gerade mit
		entsprechender Steigung inklusive Steigungsdreieck (wobei wir 1 nach rechts
		und \(f'(x_0)\) nach oben, bzw. unten gehen), so haben wir ein rechtwinkliges
		Dreieck mit bekannten Katheten und können mit dem Tangens entsprechend den
		Winkel berechnen:
		\formel{\[\alpha = tan^{-1}(f'(x_0))\]}

		\(\star\) Den Winkel zwischen zwei Funktionen an einer Stelle (im Normalfall
		der Schnittpunkt der Funktionen) berechnet man, indem zuerst wie oben
		beschrieben, die Winkel der Funktionen mit der x-Achse berechnet werden (aber
		natürlich an der entsprechenden Stelle \(x_0\)). Die beiden Winkel werden dann
		voneinander abgezogen, um den Schnittwinkel zu bekommen.\\
		Beachtet auch, dass es immer zwei Schnittwinkel gibt, zumeist einen größeren
		und einen kleineren und immer ist  der kleinere der Gesuchte ! Ist euer
		berechneter Winkel größer als \(90^\circ\), so zieht ihr diesen einfach von
		\(180^\circ\) ab.
		
		\summary{
			Um den Winkel zwischen der Funktion an einer bestimmten Stelle \(x_0\) und
			der x-Achse zu bestimmen muss man:
			\begin{enumerate}
			  \item Die erste Ableitung der Funktion bilden.
			  \item die Formel von oben benutzen um den Winkel zu berechnen.
			\end{enumerate}
			Soll der Winkel zwischen zwei Funktionen an einer Stelle bestimmt werden muss
			man zusätzlich:
			\begin{enumerate}
			  \item die Schritte von oben für die zweite Funktion wiederholen um den
			  Winkel mit der x-Achse zu bestimmen.
			  \item die beiden Winkel voneinander abziehen.
			\end{enumerate}
		}
		
		\tags{
			einschließender Winkel zwischen Funktionen, Anstiegswinkel
		}