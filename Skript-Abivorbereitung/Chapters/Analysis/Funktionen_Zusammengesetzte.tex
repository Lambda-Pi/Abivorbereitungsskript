\section{Zusammengesetzte Funktionen}
\todo[color=red]{Kasten für Formel hinzufügen}
\todo[color=red]{Zusammenfassung-Kasten hinzufügen}
\todo[color=red]{Signalwörter-Kasten hinzufügen}
Unsere Grundfunktionen können wir nun beliebig kombinieren (im Folgenden sind die Grundfunktionen immer als f(x) und g(x) dargestellt, die Resultierende nennen wir h(x)). Dieses Kapitel dient zum einen dem tieferen Verständnis der Materie, aber vor allem wird es uns helfen, die Ableitungs-/ \& Integral-Regeln zu verstehen und richtig anwenden zu können\footnote{Natürlich lassen die drei Möglichkeiten sich auch noch einmal miteinander kombinieren. Das werden wir im Kurs an einigen Beispielen sehen.}. Wenn ihr zwei Funktionen kombiniert, so setzt einfach die Funktion für das entsprechende f(x) oder g(x) ein.
\subsection{Summe \& Differenzen}
\todo[color=red]{Kasten für Formel hinzufügen}
\todo[color=red]{Zusammenfassung-Kasten hinzufügen}
\todo[color=red]{Signalwörter-Kasten hinzufügen}
Die einfachste Form ist das Addieren und Subtrahieren von zwei Funktionen. Dies wird vor allem nötig sein im Wahlteil, wenn man z. B. die Fläche berechnen will, die von zwei Funktionen eingeschlossen wird (zuerst ziehen wir die Funktionen voneinander ab und integrieren dann). Die Darstellung sieht folgendermaßen aus:
\[h_1(x)=f(x)+g(x), \mathrm{\ bzw\ } h_2(x)=f(x)-g(x)\]
\subsection{Produkte \& Quotienten}
\todo[color=red]{Kasten für Formel hinzufügen}
\todo[color=red]{Zusammenfassung-Kasten hinzufügen}
\todo[color=red]{Signalwörter-Kasten hinzufügen}
Diese Regel bei einer Funktion zu erkennen, wird uns helfen, die Faktorregel beim Ableiten und Integrieren anwenden zu können:
\[h_1(x)=f(x)\cdot g(x), \mathrm{\ bzw\ } h_2(x)=\frac{f(x)}{g(x)}\]
\subsection{Verkettungen}
\todo[color=red]{Kasten für Formel hinzufügen}
\todo[color=red]{Zusammenfassung-Kasten hinzufügen}
\todo[color=red]{Signalwörter-Kasten hinzufügen}
Eine letzte Kombination stellt die Verkettung dar. Diese werden wir beim Ableiten bei der Kettenregel wieder erkennen. Allgemein beschreibt man das so:
\[h(x)=f(g(x))\]
Wir ersetzen also bei f(x) jedes x durch die Funktion in g(x) (alles in Klammern schreiben). Ein kleines Beispiel für das Verständnis: mit \(f(x)=e^x\ \&\ g(x)=2x+3\) gilt \(h(x)=f(g(x))=e^{2x+3}\).
