\subsection{Aufstellen von Funktionen}
	\todo[color=purple]{Kasten für Formel hinzufügen}
	\todo[inline,color=red]{Zusammenfassung-Kasten hinzufügen}
	\todo[inline,color=red]{Signalwörter-Kasten hinzufügen}
	Es kommt öfter vor, dass man Funktionen selbst aufstellen muss. Dazu sollte man
	sich zuerst überlegen, welche Art von Grundfunktion man dazu benötigt. Das zu
	erkennen ist Übungssache, meist steht jedoch dabei, welche ihr verwenden sollt.
	Je nach Angaben kann man einfach Verschiebungen und Streckungen an ihnen
	vollziehen. Trotzdem wollen wir das nochmals kurz durchgehen.\\

	\(\star\) Haben wir einen Vorgang, der sich in einer bestimmten Zeit
	verdoppelt, halbiert oder ähnliches, so können wir dies als e-Funktion
	darstellen.\\

	\(\star\) Bei einem sich wiederholenden Vorgang handelt es sich um eine Sinus-
	oder Kosinusfunktion. Diese sind gesondert zu beachten, wenn man diese nicht
	mit dem Taschenrechner löst, da man hier Amplitude und Periode ermittelt und
	diese dann einfach einsetzt.\\

	\(\star\) Bei den anderen Funktionen kann man dies meist auch auf folgende Art
	lösen (manchmal auch nur): Zuerst einmal stellen wir Bedingungen auf. Das
	bedeutet, wir setzen in unsere Funktion alle Daten ein, die wir kennen (z. B.
		wenn wir den Punkt (1|2) kennen, so setzen wir f(1)=2 ein und einen
	Extrempunkt an der Stelle 5 wird mit f'(5)=0 angegeben). Die Faktoren, die wir
	nicht kennen, lassen wir einfach entsprechend stehen und haben dann ein
	Gleichungssystem, dass wir lösen können. Grundsätzlich gilt, man braucht
	mindestens so viele Informationen, wie man Variablen hat. Bei einer
	ganzrationalen Funktion mit dem Grad n haben wir immer n + 1 Unbekannte. Haben
	wir zum Beispiel eine ganzrationale Funktion mit dem Grad 3, so haben wir 4
	Faktoren und um diese zu lösen, benötigen wir mindestens 4 Bedingungen.
