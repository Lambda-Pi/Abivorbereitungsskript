\subsection{Lineare Substitution ('Kettenregel')}
	\todo[color=purple]{Kasten für Formel hinzufügen}
	\todo[color=red]{Zusammenfassung-Kasten hinzufügen}
	\todo[color=red]{Signalwörter-Kasten hinzufügen}
	Verkettungen zu Integrieren ist leider um einiges komplizierter als beim
	Ableiten. Ihr habt aber Glück, dass im Abitur lediglich die lineare
	Substitution benutzt wird, sprich es gibt nur eine lineare Funktion (z. B.
	g(x)=f(3x +2)) die in einer anderen steckt. Haben wir eine solche Funktion
	vorliegen, so müssen wir lediglich zu dem Integral der äußeren Funktion die
	Zahl vor dem x (hier die 3) als Kehrwert dazu multiplizieren. Wir werden hier
	lediglich ein Beispiel zeigen, allerdings gilt das bei allen Funktionen
	genauso:
	\[f(x)=sin(3x+4)\ =>\ \int f(x)\ dx=-\frac{1}{3}\cdot cos(3x+4)\]
