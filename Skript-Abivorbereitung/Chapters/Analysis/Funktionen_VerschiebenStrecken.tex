\subsection{Verschieben \& strecken von Funktionen}
	\todo[color=green]{Kasten für Formel hinzufügen}
	\todo[color=purple]{Zusammenfassung-Kasten hinzufügen}
	\todo[color=purple]{Signalwörter-Kasten hinzufügen}
	Bevor wir die Funktionsarten auflisten und beschreiben, wollen wir erst noch
	darauf eingehen, wie man Funktionen verschiebt und streckt. Für das bessere
	Verständnis zeigen wir das an den einzelnen Funktionen selbst noch einmal. Im
	Kommenden werden wir die ursprüngliche Funktion f(x) und die veränderten h(x)
	nennen.\\

	\(\star\) Wollen wir eine Funktion nach oben oder unten verschieben, dann
	addieren wir einfach die entsprechende Zahl c hinzu (eine positive, um sie nach
	oben zu schieben und eine negative für selbiges nach unten), also
	\formel{\[h_1(x)=f(x)+c\]}

	\(\star\) Wollen wir die Funktion nach links oder rechts verschieben, so
	schreiben wir
	\formel{\[h_2(x)=f(x-b)\]}
	wobei b die Verschiebung darstellt. \textit{Am Besten setzt ihr das (x-b) in
	Klammern} dort ein, wo zuvor das x war, so vermeidet ihr Fehler. Wollen wir sie
	nach rechts verschieben, so setzen wir für b eine positive Zahl ein (das -
	bleibt also), wollen wir sie nach links verschieben, so setzen wir für b
	entsprechend eine negative Zahl ein, wodurch ein + in der Klammer steht.\\

	\(\star\) Kommen wir nun noch zur Streckung bzw. Stauchung. Haben wir eine
	Funktion und wollen sie strecken, so multiplizieren wir die Funktion einfach
	mit einer Zahl a>1. Die Stauchung erfolgt durch das multiplizieren mit 0<a<1,
	wodurch die Funktion an die x-Achse geschmiegt wird. Ist unser a negativ, dann
	wird unsere Funktion einfach an der x-Achse gespiegelt, die Streckung oder
	Stauchung bleibt aber, wie oben, von a abhängig. Dargestellt sieht das dann so
	aus:
	\formel{\[h_3(x)=a\cdot f(x)\]}
