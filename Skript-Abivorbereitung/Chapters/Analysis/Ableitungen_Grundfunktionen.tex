\subsection{Ableitungen der Grundfunktionen}
	\todo[color=green]{Kasten für Formel hinzufügen}
	\todo[color=green]{Zusammenfassung-Kasten hinzufügen}
	\todo[color=purple]{Signalwörter-Kasten hinzufügen}
	Die Herleitung für die Ableitung stammt von der mittleren Änderungsrate, welche
	die Steigung einer Sekante angibt. Dazu setzen wir einfach die zwei Punkte die
	wir haben, in folgende Funktion ein:
	\formel{\[\frac{f(x_2)-f(x_1)}{x_2-x_1}\textrm{, oder anders geschrieben }
	\frac{\Delta f(x)}{\Delta x}\]}
	Die Ableitung erfolgt dann, indem man den Abstand der Punkte
	gegen 0 laufen lässt.\\

	\(\star\) Für die ganzrationalen Funktionen, die Brüche und die
	Wurzelfunktionen gilt allgemein die Formel:
	\formel{\[f(x)=x^n\ \Rightarrow\ f'(x)=n\cdot x^{n-1}\]}
	Hier ein paar Beispiele: \(f_1(x)=x^3\ \Rightarrow\ f_1(x)=3x^2;\\
	f_2(x)=\frac{1}{x^2}=x^{-2}\ \Rightarrow\
	f_2'(x)=-2x^{-2-1}=-2x^{-3}=-\frac{2}{x^3};\\
	f_3(x)=\sqrt{x}=x^{\frac{1}{2}}\
	\Rightarrow\ f_3'(x)=\frac{1}{2}x^{-\frac{1}{2}}=\frac{1}{2\sqrt{x}}\).\\
	Konstanten ergeben abgeleitet immer 0.\\

	\(\star\) Ableitungen von exponentiellen Funktionen ergeben immer die gleiche
	Funktion wieder, jedoch mit einem Vorfaktor. Jetzt kommen wir auch endlich zum
	Sinn der eulerschen Zahl: Leitet man \(2^x\) ab, so bekommt man einen
	Vorfaktor, der kleiner ist als 1, die Ableitung von \(3^x\) einen der größer
	ist als 1. Die Ableitung von \(e^x\) hat den Vorfaktor 1, die Ableitung ist
	also gleich der eigentlichen Funktion. Wir möchten noch einmal anmerken, dass
	\(a^x=e^{ln(a)x}\) ist. Wie man das dann ableitet, wird später erklärt. Die
	Ableitung der Funktion ist jetzt also folgendermaßen:
	\formel{\[f(x)=e^x\ \Rightarrow\ f'(x)=e^x\]}

	\(\star\) Die trigonometrischen Funktionen leiten sich folgendermaßen ab:
	\formel{\[f_1(x)=sin(x)\ \Rightarrow\ f_1'(x)=cos(x)\ \&\ f_2(x)=cos(x)\
	\Rightarrow\ f_2'(x)=-sin(x)\]}
	\(\star\) Zu guter Letzt noch die Ableitung des ln:
	\formel{\[f(x)=ln(x)\ \Rightarrow\ f'(x)=\frac{1}{x}\]}
	\todo[color=green]{Zusammenfassung als Bilddatei?!}
	\summary{
		\begin{tabular}{|l|l|}
			\hline
			\textbf{f(x)} & \textbf{f'(x)} 		\\ \hline
			\(x^n\)       & \(n\cdot x^{n-1}\)  \\ \hline
			\(e^x\)       & \(e^x\)	            \\ \hline
			\(sin(x)\)    & \(cos(x)\)          \\ \hline
			\(cos(x)\)	  & \(-sin(x)\)			\\ \hline
			\(ln(x)\)	  & \(\frac{1}{x}\)		\\ \hline
		\end{tabular}
	}