\section{Funktionen}
Kommen wir kurz auf Funktionen im allgemeinen zu sprechen. Grundsätzlich werden Funktionen als f(x)=\ldots dargestellt (anstelle von f können wir den Funktionen natürlich auch andere Namen geben). Das x ist unsere Variable, das Ergebnis, dass wir dann bekommen, ist unser Wert an dieser Stelle. So ist für \(f(x)=x^2\) der Wert für die Stelle \(x=2:\ f(2)=2^2=4\).\\
Doch was genau macht eine Funktion überhaupt? Sie weist jeder Zahl \underline{eine} andere Zahl zu. Letztendlich ist sogar ein Telefonbuch eine Funktion, denn sie weist jedem Namen eine Telefonnummer zu (jeder Name darf dann aber nur einmal vorkommen und nur eine Telefonnummer drin stehen haben, damit der Vergleich zulässig ist ;) ). Eine Funktion darf also zusätzlich keine Stelle mit zwei Werten haben. So darf also f(0)=1 und f(0)=3 nicht vorkommen!

% Verschieben & strecken von Funktionen
\subsection{Verschieben \& strecken von Funktionen}
	\todo[color=green]{Kasten für Formel hinzufügen}
	\todo[inline,color=red]{Zusammenfassung-Kasten hinzufügen}
	\todo[inline,color=red]{Signalwörter-Kasten hinzufügen}
	Bevor wir die Funktionsarten auflisten und beschreiben, wollen wir erst noch
	darauf eingehen, wie man Funktionen verschiebt und streckt. Für das bessere
	Verständnis zeigen wir das an den einzelnen Funktionen selbst noch einmal. Im
	Kommenden werden wir die ursprüngliche Funktion f(x) und die veränderten h(x)
	nennen.\\

	\(\star\) Wollen wir eine Funktion nach oben oder unten verschieben, dann
	addieren wir einfach die entsprechende Zahl c hinzu (eine positive, um sie nach
	oben zu schieben und eine negative für selbiges nach unten), also
	\formel{\[h_1(x)=f(x)+c\]}

	\(\star\) Wollen wir die Funktion nach links oder rechts verschieben, so
	schreiben wir
	\formel{\[h_2(x)=f(x-b)\]}
	wobei b die Verschiebung darstellt. \textit{Am Besten setzt ihr das (x-b) in
	Klammern} dort ein, wo zuvor das x war, so vermeidet ihr Fehler. Wollen wir sie
	nach rechts verschieben, so setzen wir für b eine positive Zahl ein (das -
	bleibt also), wollen wir sie nach links verschieben, so setzen wir für b
	entsprechend eine negative Zahl ein, wodurch ein + in der Klammer steht.\\

	\(\star\) Kommen wir nun noch zur Streckung bzw. Stauchung. Haben wir eine
	Funktion und wollen sie strecken, so multiplizieren wir die Funktion einfach
	mit einer Zahl a>1. Die Stauchung erfolgt durch das multiplizieren mit 0<a<1,
	wodurch die Funktion an die x-Achse geschmiegt wird. Ist unser a negativ, dann
	wird unsere Funktion einfach an der x-Achse gespiegelt, die Streckung oder
	Stauchung bleibt aber, wie oben, von a abhängig. Dargestellt sieht das dann so
	aus:
	\formel{\[h_3(x)=a\cdot f(x)\]}


% Grundarten von Funktionen
\subsection{Grundarten von Funktionen}
\todo[color=red]{Kasten für Formel hinzufügen}
\todo[color=red]{Zusammenfassung-Kasten hinzufügen}
\todo[color=red]{Signalwörter-Kasten hinzufügen}
In den folgenden Seiten wollen wir euch die Grundfunktionen vorstellen, mit denen wir uns beschäftigen werden und letztlich auch ein paar Worte über zusammengesetzte Funktionen verlieren. Mit der Verschiebung und Zusammensetzung von Funktionen haben wir dann alle Funktionen betrachtet, welche ihr kennen sollt.

	\subsubsection{Lineare Funktionen \& deren Normale}
	\todo[color=red]{Kasten für Formel hinzufügen}
\todo[color=red]{Zusammenfassung-Kasten hinzufügen}
\todo[color=red]{Signalwörter-Kasten hinzufügen}
Eine lineare Funktion stellt die leichteste Funktionenklasse dar. Als Schaubild haben wir eine Gerade. Dargestellt wird sie allgemein als:
\[f(x)=m\cdot x+c\]
wobei c der y-Achsenabschnitt ist (dort schneidet sie diese) und m die Steigung (welche auch 0 sein darf)\footnote{Wir werden uns hier sparen, jede Funktion aufzuzeichnen. Das könnt ihr bequem mit dem Taschenrechner nachholen.}. Eine Normale bedeutet, dass sie im rechten Winkel zur eigentlichen Funktion steht. Das ist der Fall, wenn die Steigung \(m_2=-\frac{1}{m_1}\) ist.

	\subsubsection{Quadratische und andere ganzrationale Funktionen}
	\todo[color=red]{Kasten für Formel hinzufügen}
\todo[color=red]{Zusammenfassung-Kasten hinzufügen}
\todo[color=red]{Signalwörter-Kasten hinzufügen}
\(\star\) Beschäftigen wir uns zunächst nur mit quadratischen Funktionen. Mit Verschiebungen wird aus \(x^2\):
\[f(x)=a(x-b)^2+c\]
Der Faktor a gibt die Streckung an, was man unter Vorbehalt mit der Steigung der linearen Funktion vergleichen kann. Je größer der Faktor ist, desto schneller steigt die Funktion an. b  stellt die Verschiebung in x-Richtung dar und c ist die Verschiebung in y-Richtung.\\
Wie man sehen kann, befindet sich in der Funktion eine binomische Formel, welche man ausklammern kann. Dann erhält man dieselbe Funktion anders geschrieben (wir haben hier noch die Faktoren umbenannt):
\[f(x)=a_2 \cdot x^2+a_1 \cdot x+a_0\]
\(\star\) Mit dieser Schreibweise kommen wir schon allgemein zu den ganzrationalen Funktionen:
\[f(x)=a_n \cdot x^n+a_{n-1} \cdot x^{n-1}+\ldots +a_1 \cdot x+a_0\]
Wir haben also eine Funktion, bei der beliebig viele (positive und ganzzahlige) Potenzen von x, mit einem Faktor \(a_n\) (dieser darf auch negativ oder 0 sein) davor, summiert werden. Die höchste Potenz (hier n) nennt man den Grad der Funktion. Wieso das wichtig ist, werden wir später noch sehen, wenn wir das Verhalten von Funktionen im unendlichen betrachten. Eine Funktion mit dem Grad 1 ist eine lineare Funktion, eine mit dem Grad 0 entspricht einer konstanten Funktion (eine lineare, mit der Steigung m=0).


%Gebrochenrationale fliegen raus!	%\subsubsection{gebrochen rationale Funktionen}
%Diese Funktionen sind 2 (meist unterschiedliche) ganzrationale Funktionen, welche geteilt werden. Heißen unsere ganzrationalen Funktionen f(x) und g(x), so ist unsere gebrochene rationale Funktion:
%\[h(x)=\frac{f(x)}{g(x)}\]
%Hier gilt nur zu beachten, dass es einen Grad im Nenner und einen im Zähler gibt und dass sie im Gegensatz zu den ganzrationalen Funktionen eine Definitionslücke haben können und zwar immer dann, wenn im Nenner 0 heraus kommt (vergleiche \(\frac{1}{x}\)).

	\subsubsection{Exponentiale Funktionen}
	\todo[color=red]{Kasten für Formel hinzufügen}
\todo[color=red]{Zusammenfassung-Kasten hinzufügen}
\todo[color=red]{Signalwörter-Kasten hinzufügen}
	%\& logarithmische Funktionen
Diese hatten wir ja vorhin schon einmal angesprochen. Ein Beispiel für eine Exponentialfunktion (mit der Basis 2) ist \(2^x\). Allerdings ist 2 eine ungeschickte Basis beim Ableiten. Unter anderem deswegen nutzt man als Basis die eulersche Zahl e=2,71828\ldots \ . Im Pflichtteil braucht ihr den Logarithmus oder das Exponential von e nicht ausrechnen (es sei denn, es ist trivial) und könnt sie einfach so hinschreiben (z. B. \(e^3\)). Genau so geht das übrigens bei allen irrationalen Zahlen wie z. B. \(\pi\).\\
Umrechnen in eine e-Funktion lässt sich unser Beispiel leicht. Da der ln die Umkehrfunktion der e-Funktion ist gilt: \(f(x)=2^x=e^{ln(2)\cdot x}\). Grundsätzlich stellen wir e-Funktionen so dar:
\[f(x)=a \cdot e^{b(x-c)}+d\]
a ist wieder die Streckung, ähnlich wie bei den quadratischen Funktionen. Das b ist auch eine Art Streckung (sie entspricht dem ln, wenn wir eine Funktion umschreiben). c verschiebt wieder in x-Richtung und d in y-Richtung.\\
%Nun noch zu den Logarithmusfunktionen. Wie schon erwähnt, sind diese die Umkehrfunktionen der e-Funktion\footnote{Was genau das heißt, sieht man gerade bei den beiden Funktionen sehr schön, wenn man beide Zeichnen lässt. Abstrakter kann man sich das so klar machen: Haben wir die e-Funktion, so vertauschen wir x und y einfach und lösen wieder nach y=\ldots auf.}. Darstellen lassen sie sich als: \[f(x)=a\cdot ln(b(x-c))+d\]
%a ist wieder die Streckung, c und d die Verschiebung und b ist ähnlich geartet wie das b bei den e-Funktionen.

	\subsubsection{Trigonometrische Funktionen}
	\todo[color=red]{Kasten für Formel hinzufügen}
\todo[color=red]{Zusammenfassung-Kasten hinzufügen}
\todo[color=red]{Signalwörter-Kasten hinzufügen}
Die trigonometrischen Funktionen bilden einen wichtigen Teil in der Analysis. Mit ihnen werden sich wiederholende Vorgänge beschrieben. Auch sind die Ableitungen und Integrale, ähnlich wie bei den e-Funktionen, sehr einfach. Dafür erfordert es ein größeres Maß an Konzentration, um die Grunddefinitionen zu verstehen. Deshalb werden wir dieses Thema ausführlich in 3 Unterkapiteln erklären.
	\paragraph{Bogenmaß}
	\todo[color=red]{Kasten für Formel hinzufügen}
\todo[color=red]{Zusammenfassung-Kasten hinzufügen}
\todo[color=red]{Signalwörter-Kasten hinzufügen}
Das Bogenmaß ist in mathematischer Sicht eingänglicher und wird deshalb auch öfter benutzt als Gradzahlen. Beide beschreiben jedoch einen Winkel. Definiert ist das Bogenmaß über den Umfang des Einheitskreises. Schauen wir uns einmal den Einheitskreis an.\\
Dieser hat den Radius 1 (daher Einheitskreis). Der Umfang des kompletten Kreises beträgt \(2\pi\). Diese Entsprechen den \(360^\circ\). Haben wir eine \(180^\circ\) Wende, so entspricht das dem Umfang eines halben Einheitskreises im Bogenmaß, also \(\pi\) \footnote{Beide Angaben geben das gleiche an, allerdings in unterschiedlichen Einheiten. Mit der Umrechnung verhält es sich ähnlich wie mit Yards und Metern.}. Allgemein gilt immer das Verhältnis \(\frac{\beta}{2\pi}=\frac{\alpha}{360^\circ}\), wobei \(\alpha\) unser Winkel in Grad ist und \(\beta\) der Winkel in Bogenmaß. Als Umrechnungsformel könnt ihr Folgendes benutzen :
\[\beta=\frac{\pi \cdot \alpha}{180^\circ}\]
\underline{\textbf{VORSICHT:}} Achtet immer darauf, dass euer Taschenrechner richtig eingestellt ist! Rechnet ihr mit Bogenmaß, so sollte im Taschenrechner an entsprechender Stelle "Bog" stehen. Rechnet ihr mit Grad, so an gleicher Stelle "Gra" !!!
	\paragraph{Definition am Einheitskreis}
	\todo[color=red]{Kasten für Formel hinzufügen}
\todo[color=red]{Zusammenfassung-Kasten hinzufügen}
\todo[color=red]{Signalwörter-Kasten hinzufügen}
	   
   \begin{figure}[h]
   \centering
   \includegraphics[scale=0.2]{Images/Einheitskreis.jpeg}
   \caption{Sinus \& Kosinus am Einheitskreis}
   \end{figure}
  
   Das Bild soll uns die Definition der trigonometrischen Funktionen erklären. Der Radius des Kreises ist 1. Somit wird die schräge Linie dies auch immer sein, egal in welchem Winkel \(\alpha\) sie zur x-Achse steht. Letzterer ist beliebig wählbar. Die rote Linie ist nun der Kosinus in Abhängigkeit des Winkels und die blaue Linie entspricht dem Sinus. Ist der Winkel 0, so ist unser Sinus (der y-Wert, an der die Gerade den Kreis schneidet) ebenfalls 0, der Kosinus (der x-Wert) entsprechend 1. Hier wird vielleicht auch ersichtlich, wieso sich das ganze wiederholt. Wenn nicht, empfehlen wir euch folgendem Link nachzugehen und durch ein wenig Ausprobieren die Funktion besser kennen zu lernen. Ein Dankeschön an dieser Stelle an den Autor des Applets Walter Fendt, welcher uns erlaubt, seine Applets zu nutzen ): \url{http://www.walter-fendt.de/m14d/sincostan.htm}\\
Zuletzt noch der Tangens. Wie der Wert am Einheitskreis festgelegt ist (auch im Link) spielt keine große Rolle. Definiert ist er einfach als \(tan(x)=\frac{sin(x)}{cos(x)}\).\\
Wir möchten noch kurz ansprechen, wie wir die trigonometrischen Funktionen in der Mittelstufe benutzt haben. Damit könnt ihr zum Beispiel den Winkel zwischen einer Funktion (bzw. deren Steigung an dem Punkt) und der x-Achse bestimmen. So ist bei einem rechtwinkligen Dreieck die längste Seite (gegenüber vom rechten Winkel) die Hypotenuse (kurz h), die Seite am Winkel \(\alpha\) nennt man die Ankathete (kurz a) und die andere, gegenüber des Winkels ist die Gegenkathete (kurz g). Damit gilt dann:
\[sin(\alpha)=\frac{g}{h},\ cos(\alpha)=\frac{a}{h},\ tan(\alpha)=\frac{g}{a}\]
	\paragraph{sin, cos \& tan als Funktion}
	\todo[color=red]{Kasten für Formel hinzufügen}
\todo[color=red]{Zusammenfassung-Kasten hinzufügen}
\todo[color=red]{Signalwörter-Kasten hinzufügen}
Wie wir schon zuvor angekündigt hatten, ist bei diesen Funktionen viel zu beachten. Wir möchten als Beispiel den Sinus nutzen, um euch das Prinzip zu erklären. Der Kosinus funktioniert aber genau so (er ist nur verschoben, wie wir später sehen werden). Den Tangens werden wir nur kurz andeuten, da er nicht oft vorkommt.\\
Grundsätzlich stellen wir den Sinus so dar:
\[f(x)=a\cdot sin(b(x-c))+d\]	
Betrachten wir zuerst die bekannten und daher einfachen Konstanten. c verschiebt wieder nach links oder rechts und das d nach oben oder unten. Verschiebt man den Kosinus um \(\frac{\pi}{2}\) nach rechts, so erhält man den Sinus, verschiebt man ihn um die gleiche Länge nach links, hat man den -sin(x) (genauer um eine \(\frac{1}{4}\) Periode, wenn diese nicht \(2 \pi\) ist):
\[sin(x)=cos(x-\frac{\pi}{2})\ \&\ cos(x)=sin(x+\frac{\pi}{2})\]
 Das a ist letztendlich wieder eine Streckung. Hier fällt einem jedoch zusätzlich etwas auf: Ist das a nicht da, (wir nehmen jetzt mal an, dass die Funktion nicht verschoben wurde) so haben alle Hochpunkte den Wert 1 und alle Tiefpunkte den Wert -1. Mit a sind alle Werte zwischen a und -a. Selbiges bei einem verschobenen Graphen herauszufinden, ist etwas komplexer. Wir nehmen einfach den höchsten Punkt und ziehen ihn vom niedrigsten ab und teilen durch 2. Ein kleines Beispiel: Haben die Hochpunkte den Wert y=4 und die Tiefpunkte den Wert y=0, so rechnen wir \(a=\frac{4-0}{2}=2\).\\
 Nun kommen wir noch zum b. Hierzu müssen wir erst einmal wissen, was eine Periode ist. Diese gibt an, wie lange es dauert, bis die Funktion wieder am gleichen Status ist, wie zuvor auch schon (sie wiederholt sich ja periodisch). Am besten schaut man, wie der Abstand von Hochpunkt zu Hochpunkt ist. Bei b=1 wären das \(2\pi\) also gerade eine Umdrehung im Einheitskreis. Durch das b verändert sich aber die Periode p und zwar nach folgender Gleichung:
 \[p=\frac{2\pi}{b}\]
 Die Erkenntnisse für den Sinus (und somit auch für den Kosinus) gelten auch für den Tangens, mit folgender Ausnahme: Beim Tangens gibt es keine Amplitude, die Streckung erfolgt also wie bei den vorherigen Funktionen.\\
 Auch wenn wir uns hier wiederholen, aber schaut bitte unbedingt, dass euer Taschenrechner mit der richtigen Winkeleinheit rechnet! Bei den \textit{trigonometrischen Funktionen} nehmt am besten immer das Bogenmaß, abgesehen davon, dass es sowieso meistens gebraucht wird, sehen eure Kurven auf dem Taschenrechner sonst eigenartig aus und ihr lauft Gefahr, dass die Rechnungen falsch sind.

%Wurzelfunktionen fliegen raus 
% \subsubsection{Wurzelfunktionen}
%Eine letzte Funktionenklasse stellt die Wurzelfunktion dar. Allgemein hat sie folgende Form:
%\[f(x)=a\sqrt[n]{b(x-c)}+d\]
%c und d Verschieben wieder, a streckt die Funktion und das b ist wieder ähnlich dem der e-Funktion (streckt also wieder in gewisser Art).


% Zusammengesetzte Funktionen
\section{Zusammengesetzte Funktionen}
Unsere Grundfunktionen können wir nun beliebig kombinieren (im Folgenden sind die Grundfunktionen immer als f(x) und g(x) dargestellt, die Resultierende nennen wir h(x)). Dieses Kapitel dient zum einen dem tieferen Verständnis der Materie, aber vor allem wird es uns helfen, die Ableitungs-/ \& Integral-Regeln zu verstehen und richtig anwenden zu können\footnote{Natürlich lassen die drei Möglichkeiten sich auch noch einmal miteinander kombinieren. Das werden wir im Kurs an einigen Beispielen sehen.}. Wenn ihr zwei Funktionen kombiniert, so setzt einfach die Funktion für das entsprechende f(x) oder g(x) ein.
\subsection{Summe \& Differenzen}
Die einfachste Form ist das Addieren und Subtrahieren von zwei Funktionen. Dies wird vor allem nötig sein im Wahlteil, wenn man z. B. die Fläche berechnen will, die von zwei Funktionen eingeschlossen wird (zuerst ziehen wir die Funktionen voneinander ab und integrieren dann). Die Darstellung sieht folgendermaßen aus:
\[h_1(x)=f(x)+g(x), \mathrm{\ bzw\ } h_2(x)=f(x)-g(x)\]
\subsection{Produkte \& Quotienten}
Diese Regel bei einer Funktion zu erkennen, wird uns helfen, die Faktorregel beim Ableiten und Integrieren anwenden zu können:
\[h_1(x)=f(x)\cdot g(x), \mathrm{\ bzw\ } h_2(x)=\frac{f(x)}{g(x)}\]
\subsection{Verkettungen}
Eine letzte Kombination stellt die Verkettung dar. Diese werden wir beim Ableiten bei der Kettenregel wieder erkennen. Allgemein beschreibt man das so:
\[h(x)=f(g(x))\]
Wir ersetzen also bei f(x) jedes x durch die Funktion in g(x) (alles in Klammern schreiben). Ein kleines Beispiel für das Verständnis: mit \(f(x)=e^x\ \&\ g(x)=2x+3\) gilt \(h(x)=f(g(x))=e^{2x+3}\).

