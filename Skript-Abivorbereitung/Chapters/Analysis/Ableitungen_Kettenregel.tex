\subsection{Kettenregel}
	\todo[color=green]{Kasten für Formel hinzufügen}
	\todo[color=green]{Zusammenfassung-Kasten hinzufügen}
	\todo[color=purple]{Signalwörter-Kasten hinzufügen}
	Eine verkettete Funktion abzuleiten erfordert zuerst, dass wir eine Verkettung
	haben. Also, dass eine Grundfunktion in eine andere eingesetzt wurde. Dann gilt
	der Merksatz \emph{innere Ableitung mal äußere Ableitung}. Die innere Funktion
	in der abgeleiteten äußeren bleibt aber unverändert stehen.
	\formel{\[h(x)=f(g(x))\ \Rightarrow\ h'(x)=f'(g(x))\cdot g'(x)\]}
	Drei Beispiele hierzu: \(f_1(x)=e^{x^2}\ \Rightarrow\ f_1'(x)=e^{x^2}\cdot
	2x,\\
	f_2(x)=sin(ln(x))\ \Rightarrow\ f_2'(x)=cos(ln(x))\cdot \frac{1}{x}\).\\
	Als letztes Beispiel die Ableitung von \(3^x\): \(f_3(x)=3^x=e^{ln(3)\cdot x}\
	\Rightarrow\ f_3'(x)=ln(3)\cdot e^{ln(3)\cdot x}=ln(3)\cdot 3^x\).
	
	\summary{
		Um die Kettenregel anzuwenden muss man:
		\begin{enumerate}
		  \item die äußere und die innere Funktion identifizieren.
		  \item die äußere Funktion wie gewohnt ableiten, die innere Funktion bleibt
		  unverändert stehen!
		  \item die innere Funktion ableiten und mit der abgeleiteten äußeren
		  multiplizieren.
		\end{enumerate}
	}