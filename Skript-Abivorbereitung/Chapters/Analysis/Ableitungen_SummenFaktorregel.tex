\subsection{Summen- \& Faktorregel}
Diese beiden Regeln sind nicht schwer, aber sehr nützlich.\\
\(\star\) Die Summenregel besagt, wir können bei Summen einfach jeden Teil einzeln ableiten:
\[(f(x)+g(x))'=f'(x)+g'(x)\]
Dazu noch ein kurzes Beispiel: \(f'(x)=(x^2+sin(x))'=2x+cos(x)\).\\
\(\star\) Die Faktorregel sagt, wir können Zahlen, die mit der Grundfunktion multipliziert werden, einfach beim Ableiten zu ignorieren, müssen sie aber natürlich in die Ableitung mitnehmen:
\[f'(x)=(a\cdot g(x))'=a\cdot g'(x)\]
Auch hierzu ein Beispiel: \(f'(x)=(2\ sin(x))'=2\ cos(x)\)
