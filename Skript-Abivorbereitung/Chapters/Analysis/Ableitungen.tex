\section{Ableitungen (Differentialrechnung)}
	\todo[color=purple]{Kasten für Formel hinzufügen}
	\todo[color=purple]{Zusammenfassung-Kasten hinzufügen}
	Stellen wir uns zunächst einmal vor, was eine Ableitung überhaupt ist. Sie gibt
	die Steigung der Tangenten einer Funktion an jeder Stelle an. Eine Tangente
	kann man sich so vorstellen: Nehmt ihr eine Kugel und haltet ein Buch daran, so
	stellt dieses die Tangente dar (wenn auch im Dreidimensionalen, wir brauchen
	aber nur die zweidimensionalen). Das Buch berührt die Oberfläche und liegt an
	ihr an, durchstößt sie jedoch nicht.
	
	\tags{
		Steigung, Änderungsrate, Tangente / Normale, Differential, Extremas,
		Wendepunkte, Winkel zwischen Funktionen / Achsen }

	% Ableitungen der Grundfunktionen
	\subsection{Ableitungen der Grundfunktionen}
	\todo[color=green]{Kasten für Formel hinzufügen}
	\todo[color=purple]{Zusammenfassung-Kasten hinzufügen}
	\todo[color=purple]{Signalwörter-Kasten hinzufügen}
	Die Herleitung für die Ableitung stammt von der mittleren Änderungsrate, welche
	die Steigung einer Sekante angibt. Dazu setzen wir einfach die zwei Punkte die
	wir haben, in folgende Funktion ein:
	\formel{\[\frac{f(x_2)-f(x_1)}{x_2-x_1}\textrm{, oder anders geschrieben }
	\frac{\Delta f(x)}{\Delta x}\]}
	Die Ableitung erfolgt dann, indem man den Abstand der Punkte
	gegen 0 laufen lässt.\\

	\(\star\) Für die ganzrationalen Funktionen, die Brüche und die
	Wurzelfunktionen gilt allgemein die Formel:
	\formel{\[f(x)=x^n\ \Rightarrow\ f'(x)=n\cdot x^{n-1}\]}
	Hier ein paar Beispiele: \(f_1(x)=x^3\ \Rightarrow\ f_1(x)=3x^2;\\
	f_2(x)=\frac{1}{x^2}=x^{-2}\ \Rightarrow\
	f_2'(x)=-2x^{-2-1}=-2x^{-3}=-\frac{2}{x^3};\\
	f_3(x)=\sqrt{x}=x^{\frac{1}{2}}\
	\Rightarrow\ f_3'(x)=\frac{1}{2}x^{-\frac{1}{2}}=\frac{1}{2\sqrt{x}}\).\\
	Konstanten ergeben abgeleitet immer 0.\\

	\(\star\) Ableitungen von exponentiellen Funktionen ergeben immer die gleiche
	Funktion wieder, jedoch mit einem Vorfaktor. Jetzt kommen wir auch endlich zum
	Sinn der eulerschen Zahl: Leitet man \(2^x\) ab, so bekommt man einen
	Vorfaktor, der kleiner ist als 1, die Ableitung von \(3^x\) einen der größer
	ist als 1. Die Ableitung von \(e^x\) hat den Vorfaktor 1, die Ableitung ist
	also gleich der eigentlichen Funktion. Wir möchten noch einmal anmerken, dass
	\(a^x=e^{ln(a)x}\) ist. Wie man das dann ableitet, wird später erklärt. Die
	Ableitung der Funktion ist jetzt also folgendermaßen:
	\formel{\[f(x)=e^x\ \Rightarrow\ f'(x)=e^x\]}

	\(\star\) Die trigonometrischen Funktionen leiten sich folgendermaßen ab:
	\formel{\[f_1(x)=sin(x)\ \Rightarrow\ f_1'(x)=cos(x)\ \&\ f_2(x)=cos(x)\
	\Rightarrow\ f_2'(x)=-sin(x)\]}
	\(\star\) Zu guter Letzt noch die Ableitung des ln:
	\formel{\[f(x)=ln(x)\ \Rightarrow\ f'(x)=\frac{1}{x}\]}
	\todo[inline,color=red]{Zusammenfassung als Bilddatei?!}

	% Summen- & Faktorregel
	\subsection{Summen- \& Faktorregel}
Diese beiden Regeln sind nicht schwer, aber sehr nützlich.\\
\(\star\) Die Summenregel besagt, wir können bei Summen einfach jeden Teil einzeln ableiten:
\[(f(x)+g(x))'=f'(x)+g'(x)\]
Dazu noch ein kurzes Beispiel: \(f'(x)=(x^2+sin(x))'=2x+cos(x)\).\\
\(\star\) Die Faktorregel sagt, wir können Zahlen, die mit der Grundfunktion multipliziert werden, einfach beim Ableiten zu ignorieren, müssen sie aber natürlich in die Ableitung mitnehmen:
\[f'(x)=(a\cdot g(x))'=a\cdot g'(x)\]
Auch hierzu ein Beispiel: \(f'(x)=(2\ sin(x))'=2\ cos(x)\)


	% Produktregel
	\subsection{Produktregel}
	\todo[color=green]{Kasten für Formel hinzufügen}
	\todo[inline,color=red]{Zusammenfassung-Kasten hinzufügen}
	\todo[inline,color=red]{Signalwörter-Kasten hinzufügen}
	Haben wir ein Produkt aus zwei unserer Grundfunktionen, so leiten wir zunächst
	die eine Funktion ab und multiplizieren sie mit der anderen (unveränderten) und
	addieren dieses Produkt zu dem Produkt der einen (unveränderten) Funktion mit
	der Ableitung der anderen. Mathematisch geschrieben\footnote{Oftmals werden die
	Funktionen auch mit v(x) und u(x) betitelt, dies tut aber nichts zur Sache, da
	wir die Funktionen ja nennen dürfen, wie wir wollen.}:
	\formel{\[h'(x)=(f(x)\cdot g(x))' = f'(x)\cdot g(x)+f(x)\cdot g'(x)\]}
	Wieder ein Beispiel für das bessere Verständnis: \(f(x)=x^2\cdot sin(x)\ =>\
	f'(x)=2x\cdot sin(x)+x^2\cdot cos(x)\)


	% Kettenregel
	\subsection{Kettenregel}
	\todo[color=green]{Kasten für Formel hinzufügen}
	\todo[inline,color=red]{Zusammenfassung-Kasten hinzufügen}
	\todo[inline,color=red]{Signalwörter-Kasten hinzufügen}
	Eine verkettete Funktion abzuleiten erfordert zuerst, dass wir eine Verkettung
	haben. Also, dass eine Grundfunktion in eine andere eingesetzt wurde. Dann gilt
	der Merksatz \emph{innere Ableitung mal äußere Ableitung}. Die innere Funktion
	in der abgeleiteten äußeren bleibt aber unverändert stehen.
	\formel{\[h(x)=f(g(x))\ \Rightarrow\ h'(x)=f'(g(x))\cdot g'(x)\]}
	Drei Beispiele hierzu: \(f_1(x)=e^{x^2}\ \Rightarrow\ f_1'(x)=e^{x^2}\cdot
	2x,\\
	f_2(x)=sin(ln(x))\ \Rightarrow\ f_2'(x)=cos(ln(x))\cdot \frac{1}{x}\).\\
	Als letztes Beispiel die Ableitung von \(3^x\): \(f_3(x)=3^x=e^{ln(3)\cdot x}\
	\Rightarrow\ f_3'(x)=ln(3)\cdot e^{ln(3)\cdot x}=ln(3)\cdot 3^x\).


	% Tangenten & Normale
	\subsection{Tangenten \& Normale}
Öfter kann es vorkommen, dass die Tangente, also die Gerade, die an der Funktion an einer Stelle anliegt, berechnet werden soll. Wie stellen wir das nun an? Am einfachsten und schnellsten ist die folgende Variante: Wir kennen die allgemeine Form einer Geraden und wie man eine Funktion verschiebt. Auch wissen wir, dass die erste Ableitung die Steigung der Funktion an dieser Stelle \(x_0\) ist. Das können wir nun anwenden, um die Tangente der Funktion am Punkt \(P(x_0|f(x_0))\) zu bestimmen. Wir setzen in unsere allgemeine Geradengleichung also die Steigung an dem Punkt ein und verschieben sie noch um die entsprechenden Werte in x- \& y -Richtung. Somit bekommen wir auch schon die Tangentengleichung:
\[t(x)=f'(x_0)\cdot (x-x_0)+f(x_0)\]
\(x_0\) ist hier eine Konstante, das x die Variable.\\
Die Normale steht senkrecht zur Tangente. Wie man die Steigung dieser berechnet, haben wir bereits gesehen. Die Verschiebung geht zum gleichen Punkt. Also ist die Normalengleichung:
\[n(x)=-\frac{1}{f'(x_0)}\cdot (x-x_0)+f(x_0)\]


	% Anwendung von Ableitungen
	\subsection{Anwendung von Ableitungen}
Ableitungen sind letztendlich ein Instrument, mit dem wir Eigenschaften von Funktionen überprüfen können oder mit denen wir Eigenschaften beschreiben und in die Mathematik übersetzen können.\\
\(\star\) Ein Beispiel ist die Änderungsrate. Die erste Ableitung spiegelt immer eine Änderungsrate der Funktion wieder, gibt also an, um wie viel sich die Funktion an einer bestimmten Stelle verändert. 
Machen wir uns das mal an einem Beispiel der Physik klar. Haben wir zum Beispiel den Ort eines Gegenstands in Abhängigkeit von der Zeit gegeben (s(t)=\ldots) und wir wollen wissen, wie sich der Ort mit der Zeit verändert, so leitet man nach der Zeit ab. Somit haben wir die Geschwindigkeit \(\frac{\Delta s}{\Delta t}\)(wobei \(\Delta\) gegen 0 geht), was eben angibt, um welche Strecke sich unser Gegenstand in einer Sekunde bewegt hat (erkennbar auch an den Einheiten ;) ). Genau so verhält sich das mit "Liter pro Zeit"\ oder allem anderen, was einen Bruch als Einheit besitzt.\\
\(\star\) Ableitungen können auch verwendet werden, um geometrische Informationen in die Sprache der Mathematik umzuwandeln. Folgendes Beispiel habt ihr vielleicht schon durchgerechnet. Man hat eine Funktion gegeben, die den Querschnitt eines Tals zwischen zwei Bergen beschreibt. Jetzt wissen wir an welchen Punkt die Sonne anfängt und der Schatten aufhört (natürlich ein Punkt auf der Funktion, also dem Querschnitt) und wir wollen wissen, in welchem Winkel die Sonne gerade zur x-Achse steht. Was wir jetzt noch (durch Überlegungen) wissen sollten ist, dass die Sonne den Berg tangential streift. Also brauchen wir die Tangente (an einer noch unbekannten Stelle). Die Steigung an der Stelle kann man dann noch als Winkel umrechnen (dazu kommen wir später noch). Wie man das ganze konkret berechnet, gehen wir im Kurs selbst durch.\\
\(\star\) Eine weitere wichtige Möglichkeit zur Anwendung von Ableitungen ist die Berechnung von Extrem- \& Wendepunkten.
\subsubsection{Extrempunkte}
Um diese zu bestimmen,  brauchen wir die ersten beiden Ableitungen. Zuerst bestimmt man die Nullstellen der ersten Ableitung, denn an Extrempunkten haben Funktionen immer die Steigung 0. Somit haben wir schon mal Stellen, an denen Extrempunkte sein \textbf{können}. Die Stellen (x-Werte) setzen wir dann in die zweite Ableitung ein, um zu schauen, ob es sich um einen Hoch- oder Tiefpunkt handelt. Haben wir in der zweiten Ableitung einen positiven Wert, so liegt ein Tiefpunkt vor. Bei einem negativem Wert haben wir einen Hochpunkt der ursprünglichen Funktion. Als Eselsbrücke kann man Smilies nehmen. Haben wir einen positiven Wert, so malen wir einen glücklichen Smilie, dessen Mund einen Tiefpunkt bildet. Bei einem negativen Wert malen wir einen traurigen Smilie, dessen Mund einen Hochpunkt besitzt. Weiter muss man den Punkt berechnen, an dem der Extremwert ist. Die Stelle \(x_0\) kennen wir ja bereits und müssen diese nun in die eigentliche Funktion f(x) einsetzen. Somit bekommen wir den Extrempunkt \(P(x_0|f(x_0))\).\\
Ein Problem ist es, wenn die zweite Ableitung 0 ergibt. Dann setzt man einfach in die erste Ableitung ein x kleiner als die Stelle ein und ein x, ein bisschen größer als diese Stelle, um die Steigung links und rechts von dem Extrempunkt zu ermitteln. \\
Ist die Steigung erst positiv, dann negativ, so haben wir einen Hochpunkt, umgekehrt einen Tiefpunkt. Sind beide Seiten positiv oder beide negativ, so haben wir einen Sattelpunkt. Diese Variante ist mit Nachdenken verbunden und man muss verstanden haben, wieso das so ist.\\
Noch ein kurzes Beispiel zu Extrempunktberechnungen im allgemeinen: Untersuchen wir die Funktion \(f(x)=\frac{1}{4}x^4+x^2\) auf Extremstellen. Die ersten zwei Ableitungen sind \(f'(x)=x^3+2x,\ f''(x)=3x^2+2\). Eine der potentiellen Extremstellen ist 0 (denn f'(0)=0). Eingesetzt in die zweite Ableitung ergibt f''(0)=2>0, also haben wir einen Tiefpunkt vorliegen.
\subsubsection{Wendepunkte}
Wendestellen / -punkte berechnet man ganz ähnlich wie Extrempunkte. Allerdings berechnen wir hier die Extrempunkte der ersten Ableitung. Die Nullstelle der zweiten Ableitung, eingesetzt in die dritte Ableitung, gibt uns die Richtungsänderung. Ändert sich die Kurve der Funktion von links nach rechts, haben wir bei der dritten Ableitung einen negativen Wert (\(f'''(x_w)<0)\), also der Hochpunkt der ersten Ableitung) und umgekehrt.
\subsubsection{Winkel zwischen Funktionen und Achsen}
\(\star\) Um den Winkel zwischen einer Funktion und der x-Achse zu berechnen, brauchen wir zunächst die erste Ableitung, um so die Steigung an der entsprechenden Stelle zu bekommen. Mit einer Skizze kann man sich nun herleiten, wie man den Winkel berechnet. Zeichnen wir eine Gerade mit entsprechender Steigung inklusive Steigungsdreieck (wobei wir 1 nach rechts und \(f'(x_0)\) nach oben, bzw. unten gehen), so haben wir ein rechtwinkliges Dreieck mit bekannten Katheten und können mit dem Tangens entsprechend den Winkel berechnen:
\[\alpha = tan^{-1}(f'(x_0))\]
\(\star\) Den Winkel zwischen zwei Funktionen an einer Stelle (im Normalfall der Schnittpunkt der Funktionen) berechnet man, indem zuerst wie oben beschrieben, die Winkel der Funktionen mit der x-Achse berechnet werden (aber natürlich an der entsprechenden Stelle \(x_0\)). Die beiden Winkel werden dann voneinander abgezogen, um den Schnittwinkel zu bekommen.\\
Beachtet auch, dass es immer zwei Schnittwinkel gibt, zumeist einen größeren und einen kleineren und immer ist  der kleinere der Gesuchte ! Ist euer berechneter Winkel größer als \(90^\circ\), so zieht ihr diesen einfach von \(180^\circ\) ab.

