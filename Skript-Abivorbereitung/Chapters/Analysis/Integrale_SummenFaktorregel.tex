\subsection{Summen- \& Faktorregel}
	\todo[color=green]{Kasten für Formel hinzufügen}
	\todo[color=purple]{Zusammenfassung-Kasten hinzufügen}
	\todo[color=purple]{Signalwörter-Kasten hinzufügen}
	Ganz synchron zu der Summenregel beim Ableiten dürfen wir die einzelnen Summen
	einzeln integrieren:
	\formel{\[\int (f(x)+g(x))\ dx=\int f(x)\ dx+\int g(x)\ dx\]}
	Auch die Faktorregel ist gleich. Eine konstante (z. B. eine Zahl) als Faktor
	vor der Funktion wird einfach bei der Stammfunktion dazugeschrieben, ohne sie
	weiter zu beachten:
	\formel{\[\int c\cdot f(x)\ dx=c\cdot \int f(x)\ dx\]}
