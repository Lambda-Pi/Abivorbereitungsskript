\subsection{Produktregel}
	\todo[color=purple]{Signalwörter-Kasten hinzufügen}
	Haben wir ein Produkt aus zwei unserer Grundfunktionen, so leiten wir zunächst
	die eine Funktion ab, multiplizieren sie mit der anderen (unveränderten) und
	addieren dieses Produkt zu dem Produkt der einen (unveränderten) Funktion mit
	der Ableitung der anderen. Mathematisch geschrieben\footnote{Oftmals werden die
	Funktionen auch mit v(x) und u(x) betitelt, dies tut aber nichts zur Sache, da
	wir die Funktionen ja nennen dürfen, wie wir wollen.}:
	\\ \\
	\formel{\[h'(x)=(f(x)\cdot g(x))' = f'(x)\cdot g(x)+f(x)\cdot g'(x)\]}
	\\ \\
	Wieder ein Beispiel für das bessere Verständnis: \(f(x)=x^2\cdot sin(x)\ =>\
	f'(x)=2x\cdot sin(x)+x^2\cdot cos(x)\)
	
	\summary{
		Um die Produktregel anzuwenden muss man:
		\begin{enumerate}
		  \item die multiplizierten Funktionen identifizieren.
		  \item die erste Funktion ableiten und mit der zweiten, unveränderten,
		  multiplizieren.
		  \item dazu addieren: die erste Funktion, unverändert, multipliziert mit der
		  Zweiten, abgeleitet.
		\end{enumerate}
	}