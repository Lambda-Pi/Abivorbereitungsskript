\subsection{Produktregel}
Haben wir ein Produkt aus zwei unserer Grundfunktionen, so leiten wir zunächst die eine Funktion ab und multiplizieren sie mit der anderen (unveränderten) und addieren dieses Produkt zu dem Produkt der einen (unveränderten) Funktion mit der Ableitung der anderen. Mathematisch geschrieben\footnote{Oftmals werden die Funktionen auch mit v(x) und u(x) betitelt, dies tut aber nichts zur Sache, da wir die Funktionen ja nennen dürfen, wie wir wollen.}:
\[h'(x)=(f(x)\cdot g(x))' = f'(x)\cdot g(x)+f(x)\cdot g'(x)\]
Wieder ein Beispiel für das bessere Verständnis: \(f(x)=x^2\cdot sin(x)\ =>\ f'(x)=2x\cdot sin(x)+x^2\cdot cos(x)\)
