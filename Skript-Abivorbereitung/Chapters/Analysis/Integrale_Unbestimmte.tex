\subsection{Unbestimmte Integrale/Stammfunktionen der Grundfunktionen}
	\todo[color=green]{Kasten für Formel hinzufügen}
	\todo[color=purple]{Zusammenfassung-Kasten hinzufügen}
	\todo[color=green]{Signalwörter-Kasten hinzufügen}
	Hier können wir eigentlich die Tabelle der Ableitungen nehmen, nur dass wir sie
	von rechts nach links lesen (also wir schauen, was f'(x) ist und das f(x) ist
	dann unser Integral). Deshalb werden wir uns sparen, hier alle noch einmal
	aufzulisten.\\
	Bei den ganzrationalen, Wurzel- \& Bruchfunktionen mag das vielleicht nicht
	ganz so ersichtlich sein:
	\formel{\[f(x)=x^n\ \Rightarrow\ F(x)=\int f(x)\ dx=\frac{1}{n+1}\cdot
	     x^{n+1}+c\]}
	 Dazu zwei Beispiele:\(f_1(x)=x^4\ \Rightarrow\ F_1(x)=\int f_1(x)\
	 dx=\frac{1}{5}\cdot x^5+c;\\
	 f_2(x)=\sqrt{x}=x^\frac{1}{2}\ \Rightarrow\ F_2(x)=\int f_2(x)\
	 dx=\frac{2}{3}\cdot x^{\frac{3}{2}}+c=\frac{2}{3}\cdot \sqrt{x^3}+c\).\\ \\
	Wichtig ist noch eine Ausnahme:
	\formel{\[f(x)=\frac{1}{x}\ \Rightarrow\ \int f(x)\ dx=ln(x)+c\]}
	
	\tags{Stammfunktion, "`Aufleitung"'}
