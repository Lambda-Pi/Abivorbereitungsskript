\section{Integralrechnung}
\todo[color=red]{Kasten für Formel hinzufügen}
\todo[color=red]{Zusammenfassung-Kasten hinzufügen}
\todo[color=red]{Signalwörter-Kasten hinzufügen}
Grundsätzlich sind Integrale die Umkehrung der Ableitung. Integrieren wir also eine Ableitung, (oder leiten ein Integral ab) so haben wir wieder unsere ursprüngliche Funktion. Mathematisch lässt sich das folgendermaßen darstellen\footnote{Diese Erkenntnis können wir auch nutzen, um unsere Integrale zu überprüfen. Wir leiten das Integral einfach ab und schauen dann, ob wieder die ursprüngliche Funktion raus kommt.}:
\[\int f'(x)\ dx=f(x)\]
Mit Integralen macht man hauptsächlich zwei Dinge. Haben wir eine bestimmte Funktion, so berechnet man mit dem Integral den Flächeninhalt zwischen ihrem Schaubild und der x-Achse. Das kann man natürlich einfach so nutzen, um die Fläche dazwischen explizit auszurechnen, auch zwischen zwei Funktionen. Die Relation, die wir oben gesehen haben, können wir auch nutzen. Haben wir eine Änderungsrate (z. B. eine Geschwindigkeit) so gibt das Integral die tatsächliche Änderung (oft auch der Bestand genannt, z. B. die zurückgelegte Strecke) an.

% Unbestimmte Integrale/Stammfunktionen der Grundfunktionen
\subsection{Unbestimmte Integrale/Stammfunktionen der Grundfunktionen}
	\todo[color=red]{Kasten für Formel hinzufügen}
	\todo[color=red]{Zusammenfassung-Kasten hinzufügen}
	\todo[color=red]{Signalwörter-Kasten hinzufügen}
	Hier können wir eigentlich die Tabelle der Ableitungen nehmen, nur dass wir sie
	von rechts nach links lesen (also wir schauen, was f'(x) ist und das f(x) ist
	dann unser Integral). Deshalb werden wir uns sparen, hier alle noch einmal
	aufzulisten.\\
	Bei den ganzrationalen, Wurzel- \& Bruchfunktionen mag das vielleicht nicht
	ganz so ersichtlich sein:
	\[f(x)=x^n\ \Rightarrow\ F(x)=\int f(x)\ dx=\frac{1}{n+1}\cdot x^{n+1}+c\]
	Dazu zwei Beispiele:\(f_1(x)=x^4\ \Rightarrow\ F_1(x)=\int f_1(x)\
	dx=\frac{1}{5}\cdot x^5+c;\\
	 f_2(x)=\sqrt{x}=x^\frac{1}{2}\ \Rightarrow\ F_2(x)=\int f_2(x)\
	 dx=\frac{2}{3}\cdot x^{\frac{3}{2}}+c=\frac{2}{3}\cdot \sqrt{x^3}+c\).\\ \\
	Wichtig ist noch eine Ausnahme:
	\[f(x)=\frac{1}{x}\ \Rightarrow\ \int f(x)\ dx=ln(x)+c\]


% Bestimmte Integrale
\subsection{Bestimmte Integrale}
	\todo[color=purple]{Zusammenfassung-Kasten hinzufügen}
	Bestimmte Integrale geben explizit die Fläche zwischen einer Funktion und der
	x-Achse in einem Intervall an. Man erhält also eine Zahl, wenn man dieses
	berechnet. Zuerst sucht man die Stammfunktion F(x) und setzt in diese die obere
	Grenze ein. Anschließend zieht man von diesem Wert den Wert der Stammfunktion,
	mit der unteren Grenze eingesetzt, ab. Der neue Wert ist unser Ergebnis. Das
	ganze sieht dann allgemein wie folgt aus:
	\formel{\[\int\limits_a^b f(x) \ dx=F(b)-F(a)\]}
	
	\tags{Flächeninhalt zwischen x-Achse und Funktion, Bestand}

% Summen- & Faktorregel
\subsection{Summen- \& Faktorregel}
Ganz synchron zu der Summenregel beim Ableiten dürfen wir die einzelnen Summen einzeln integrieren:
\[\int (f(x)+g(x))\ dx=\int f(x)\ dx+\int g(x)\ dx\]
Auch die Faktorregel ist gleich. Eine konstante (z. B. eine Zahl) als Faktor vor der Funktion wird einfach bei der Stammfunktion dazugeschrieben, ohne sie weiter zu beachten:
\[\int c\cdot f(x)\ dx=c\cdot \int f(x)\ dx\]


% Lineare Substitution ('Kettenregel')
\subsection{Lineare Substitution ('Kettenregel')}
	\todo[color=purple]{Kasten für Formel hinzufügen}
	\todo[color=purple]{Zusammenfassung-Kasten hinzufügen}
	\todo[color=purple]{Signalwörter-Kasten hinzufügen}
	Verkettungen zu integrieren ist leider um einiges komplizierter, als beim
	Ableiten. Ihr habt aber Glück, dass im Abitur lediglich die lineare
	Substitution benutzt wird, sprich es gibt nur eine lineare Funktion (z. B.
	g(x)=f(3x +2)), die in einer anderen steckt. Haben wir eine solche Funktion
	vorliegen, so müssen wir lediglich zu dem Integral der äußeren Funktion, die
	Zahl vor dem x (hier die 3) als Kehrwert dazu multiplizieren. Wir werden hier
	lediglich ein Beispiel zeigen, allerdings gilt das bei allen Funktionen
	genauso:
	\[f(x)=sin(3x+4)\ =>\ \int f(x)\ dx=-\frac{1}{3}\cdot cos(3x+4)\]


% Anwendungen des Integrals
\subsection{Anwendungen des Integrals}
	\todo[color=red]{Kasten für Formel hinzufügen}
	\todo[color=red]{Zusammenfassung-Kasten hinzufügen}
	\todo[color=red]{Signalwörter-Kasten hinzufügen}
	Hier möchten wir noch einmal darauf eingehen, wie man die Integrale verwenden
	kann. Dabei gehen wir auf drei Arten ein:

	\subsubsection{Flächeninhalte}
		\todo[color=red]{Kasten für Formel hinzufügen}
		\todo[color=red]{Zusammenfassung-Kasten hinzufügen}
		\todo[color=red]{Signalwörter-Kasten hinzufügen}
		Da man das Integral über die Flächeninhalte definieren kann, ist es natürlich
		auch logisch, dass man damit Flächeninhalte berechnen kann. Und zwar immer
		den Flächeninhalt zwischen der Funktion und der x-Achse. Allerdings können diese Flächeninhalte Vorzeichen haben und sich somit auch gegenseitig aufheben. Das wollen wir dann allerdings nicht. Das Problem löst man ganz einfach, indem man die Nullstellen findet und immer von Nullstelle zu Nullstelle integriert und von den einzelnen Integralen den Betrag nimmt, also wenn nötig, das Vorzeichen ändert. Ein Beispiel hierzu:
		\[f(x)=x^2-4\textrm{, mit den Nullstellen }x_0=\pm2\ =>\ \int\limits_0^4
		|f(x)|\ dx=|\int\limits_0^2 f(x)\ dx|+|\int\limits_2^4 f(x)\ dx|\]
		Wollen wir den Flächeninhalt zwischen zwei Funktionen, so ziehen wir einfach
		die eine Funktion von der anderen ab und Integrieren dann (also wir Integrieren dann h(x)=f(x)-g(x)).

	\subsubsection{Rekonstruierter Bestand}
		\todo[color=red]{Kasten für Formel hinzufügen}
		\todo[color=red]{Zusammenfassung-Kasten hinzufügen}
		\todo[color=red]{Signalwörter-Kasten hinzufügen}
		Wie schon bei den Anwendungsaufgaben für Ableitungen gezeigt, kann man
		jegliche Form von Geschwindigkeiten als Ableitung angeben. Haben wir also nun
		eine Geschwindigkeit als Funktion angegeben und Integrieren diese, so können
		wir den eigentlichen Bestand rekonstruieren. Oder zu deutsch, wir schauen,
		welche Strecke bis zu diesem Zeitpunkt zurückgelegt wurde (bzw. wie viel Liter
		eingelassen wurden oder allgemein immer der obere Teil der Einheit). Hier wird
		dann der Teil angegeben, der vom Anfangswert dazu kam oder weg ging. Hier
		müsst ihr aber beachten, dass ihr das Integral nicht aufsplitten dürft, wie
		beim Berechnen von Flächeninhalten. Der negative Teil wird dann zwar
		mitberechnet, dass macht bei näherer Überlegung aber auch Sinn. Ist die
		Geschwindigkeit negativ, so führt das Fahrzeug auch in negative Richtung.
		Genau das sagt uns der negative 'Flächeninhalt'.\\
		Hierzu passiert es bei schwereren Aufgaben, dass man Funktionen kombinieren
		muss. Beispielsweise soll der Abstand zwischen zwei sich unterschiedlich
		bewegenden Fahrzeugen untersucht werden, deren Geschwindigkeiten wir kennen.
		Am Einfachsten ist es, die Geschwindigkeiten voneinander abzuziehen, wodurch
		wir eine neue Funktion erhalten und diese dann integrieren können. Das ist so,
		weil wir die relative Geschwindigkeit betrachten (ähnlich wird auch der
		Flächeninhalt zwischen zwei Schaubildern betrachtet).

	\subsubsection{Mittelwert}
		\todo[color=red]{Kasten für Formel hinzufügen}
		\todo[color=red]{Zusammenfassung-Kasten hinzufügen}
		\todo[color=red]{Signalwörter-Kasten hinzufügen}
		Mit dem Integral können wir auch den Mittelwert \(\overline{m}\) der
		eigentlichen Funktion f(x) in einem Intervall von a nach b berechnen. Das
		sieht dann folgendermaßen aus:
		\[\overline{m}=\frac{1}{b-a}\cdot \int\limits_a^b f(x)\]

