\subsection{Bestimmte Integrale}
	\todo[color=red]{Kasten für Formel hinzufügen}
	\todo[color=red]{Zusammenfassung-Kasten hinzufügen}
	\todo[color=red]{Signalwörter-Kasten hinzufügen}
	Bestimmte Integrale geben Explizit die Fläche zwischen einer Funktion und der
	x-Achse in einem Intervall an. Man erhält also eine Zahl, wenn man dieses
	berechnet. Zuerst sucht man die Stammfunktion F(x) und setzt in diese die obere
	Grenze ein. Anschließend zieht man von diesem Wert den Wert der Stammfunktion,
	mit der unteren Grenze eingesetzt ab. Der neue Wert ist unser Ergebnis. Das
	ganze sieht dann allgemein wie folgt aus:
	\[\int\limits_a^b f(x) \ dx=F(b)-F(a)\]
