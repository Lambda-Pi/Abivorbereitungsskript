\chapter{Strukturiertes Lösen von Aufgaben}
	Die Mathematik bietet uns viele Algorithmen, also 'Vorschriften' wie ein
	Problem zu lösen ist. Letztendlich sind alle Themen, die wir bis hier hin
	durchgegangen sind, solche Handlungsvorschriften. Die Kunst ist nun, vor allem
	bei komplizierteren Aufgaben, zu erkennen, wann welcher Lösungsalgorithmus
	benutzt werden kann und wie eventuell mehrere Algorithmen zu kombinieren sind.
	
	\section{Allgemeine Tipps}
		Zunächst sollte man immer das Problem genauer betrachten und sich folgende
		Fragen stellen:
		\begin{itemize}
		  \item Zu welchem Thema gehört das Problem? (Analysis, Geometrie, Stochastik)
		  \item Was ist gegeben?
		  \item Was wird gesucht?
		  \item Gibt es spezielle Signalwörter?
		\end{itemize}
		Anschließend muss man das 'Handwerkszeug', das man im Unterricht beigebracht
		bekommen hat, durchgehen und sich überlegen, welcher Algorithmus nun zu diesem
		Problem passt. Soll der Bestand bestimmt werden und wir haben die Änderungsrate als
		Funktion gegeben, so muss integriert werden. Soll eine Fläche im
		dreidimensionalen Raum berechnet werden, so kann man sich überlegen, was für
		eine Figur dargestellt ist und ob man sie einfach in Dreiecke zerlegen kann und diese mit
		dem Kreuzprodukt berechnet und dann summiert. Alternativ kann man schauen,
		welche Art von Figur gegeben ist (z.B. ein Trapez oder ein Rechteck) und
		mit der Formelsammlung entsprechend die Fläche berechnen.\\
		Manchmal muss man aber auch mehrere Schritte anwenden. Dann empfiehlt es sich,
		von oben nach unten zu arbeiten. Zur Erklärung ein einfaches Beispiel: Wir
		haben die vier Punkte einer dreiseitigen Pyramide gegeben und sollen die Höhe
		dieser Pyramide berechnen (der Punkt in der Spitze der Pyramide heiße S). Man soll einen
		Abstand bestimmen, und zwar den zwischen dem Boden der Pyramide (was eine
		Ebene ist) und der Spitze (also einem Punkt). Hat man sich das Problem auf
		diese Weise klar gemacht, z.B. durch eine Skizze, so kommt man schnell darauf,
		dass es sich um die Lagebeziehung zwischen einem Punkt und einer Ebene
		handelt. Dieses Problem ist leicht zu lösen, da wir hierfür eine (bzw. sogar
		zwei) Formeln haben. Unser Problem ist nur, dass wir keine Ebene gegeben
		haben, wir kennen aber die drei Punkte dieser Ebene. Mit diesen können wir natürlich
		die Ebene aufstellen und letztendlich das eigentliche Problem lösen.\\
		Diese Art von Aufgaben rangzugehen, empfiehlt sich bei jeder Aufgabe! Und
		das nicht nur in Mathe, sondern auch in Physik, Chemie, usw.
		
		\summary{
			Wie man bei komplizierteren Aufgaben vorgehen sollte:
			\begin{enumerate}
			  \item Was wird gesucht?
			  \item Was ist gegeben?
			  \item Welche Signalwörter gibt es?
			  \item Skizzen machen um sich das Problem klar zu machen
			  \item Mit welcher Formel würde man zum Ziel kommen?
			  \item Was fehlt noch um diese Formel anwenden zu können?
			  \item Wie kann ich die fehlenden Informationen aus dem gegebenen bekommen?
			  \item Rechnen
			  \item Lösung aufschreiben
			\end{enumerate}
		}
	
	\section{Was soll ich tun, wenn ich bei einer Aufgabe nicht weiter komme?}
		Kommt man bei einer Teilaufgabe in der Klausur oder im Abitur nicht weiter,
		sollte man diese erstmal überspringen. Oft sieht man vor lauter Bäumen den
		Wald nicht mehr und verschwendet so nur Zeit und damit Punkte. Hat man die
		anderen Aufgaben fertig und widmet sich der Aufgabe erneut, hat man oft einen
		anderen Blick darauf und eher eine Idee, wie man die Aufgabe lösen kann.
		Außerdem hat man die Gewissheit, dass man die 'einfachen' Punkte bereits
		alle gesammelt hat und keine Punkte nur durch fehlende Zeit verschenken
		kann.
		Hat man noch Zeit und grübelt über eine Aufgabe und kommt auch mit den Tricks
		von oben nicht weiter, helfen oft Skizzen. Zeichnet euch auf, wie ihr denkt,
		dass das Problem aussehen könnte, bei dem Beispiel von oben, einfach die
		Pyramide. Haben wir die Zulauf- \& Ablaufrate eines Beckens mit Wasser, dann
		zeichnet euch das Becken und ein Rohr für den Einlauf und eins für den Abfluss. 
		So kann man sich leicht das Problem klar machen und bekommt oftmals die
		entscheidende Idee, was zu tun ist.
		Sollte auch das nicht weiterhelfen, dann ratet. Schreibt Formeln hin, bei
		denen ihr denkt, sie könnten richtig sein, beschreibt in kurzen Worten, wie
		ihr das Problem lösen wollt und versucht so weit zu rechnen, wie ihr könnt.
		Entweder ihr bekommt dann den entscheidenden Geistesblitz oder wenn nicht,
		gibt es mit etwas Glück Teilpunkte auf die Aufgabe.
