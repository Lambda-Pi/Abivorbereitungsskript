\subsection{Auswertung des Ergebnisses}
\todo[color=red]{Kasten für Formel hinzufügen}
\todo[color=red]{Zusammenfassung-Kasten hinzufügen}
\todo[color=red]{Signalwörter-Kasten hinzufügen}
Im Abitur wird meistens \(\alpha\) angegeben. Bei den einseitigen Signifikanztests können wir also genau betrachten, wann die Wahrscheinlichkeit für die Gegenhypothese diesen Wert überschreitet oder unterschreitet. Daher gibt die Ungleichung direkt an, welcher Ablehnungsbereich $\bar{A}$ und welcher Annahmebereich $A$ für unsere Nullhypothese $H_0$ gilt. Je nach dem, wie wichtig es ist, ob man \(\alpha\) vertraut, würde man selbiges verändern. Ein großer Wert für $\alpha$ bedeutet, dass wir uns mit einer größeren Wahrscheinlichkeit irren, was aber dazu führt, dass unser Annahmebereich größer ist. Aufgrund dessen können wir also eine eindeutige Antwort darauf finden, ob die Behauptung zutrifft oder nicht.\\
Bei unserem Beispiel schließen wir daraus, dass $H_1$ für mindestens 62 kaputte Teile zutrifft, also $H_0$ verworfen werden muss.
Es ergeben sich für $H_0$ ein Annahmebereich $A=\{0,\cdots,\ 61\}$, sowie ein Ablehnungsbereich $\bar{A}=\{62,\cdots,\ 500\}$
