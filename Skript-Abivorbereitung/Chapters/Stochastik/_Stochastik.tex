\chapter{Stochastik}
	In der Stochastik geht es darum, Wahrscheinlichkeiten zu berechnen und zu
	verstehen, was es mit ihnen auf sich hat. Es geht also um statistische
	Verteilungen, bei der man eine Wahrscheinlichkeit angeben kann, dass ein
	Ereignis eintritt. Die Idee dahinter ist nicht allzu schwer und die
	Aufgabenstellungen lassen sich meist auf euch bekannte Situationen, wie z.B.
	Würfel oder Ziehen aus einer Urnen, herunterbrechen. Das größte Problem wird
	sein, festzustellen, welches Ereignis wir überhaupt betrachten müssen.

	% Section Begriffe
	\section{Begriffe}
	In der Stochastik tauchen einige neue Begriffe auf, die wir zunächst betrachten
	wollen, bevor wir diese dann anwenden.

	\subsection{Ereignis}
		\todo[color=red]{Kasten für Formel hinzufügen}
		\todo[color=red]{Zusammenfassung-Kasten hinzufügen}
		\todo[color=red]{Signalwörter-Kasten hinzufügen}
		Zunächst sollten wir klären, was ein Ereignis in der Stochastik ist.
		Betrachten wir, wie etwas ausgehen kann, so gibt es meistens mehrere
		Möglichkeiten, die eintreten können. Diese werden in diesem Zusammenhang auch
		Ereignisse genannt. Jedes mögliche Ereignis bekommt dann einen großen
		Buchstaben (manchmal mit Indizes) zugewiesen\footnote{Das liegt daran, dass
		man hier Mengen angibt und keine Zahlen oder Sonstiges. Falls ihr damit
		Probleme habt schaut euch nochmal das dazugehörige Kapitel im Vorgeplänkel
		an}. Grundsätzlich können in einem Ereignis mehrere Einzelereignisse stecken.
		Ist unser Ereignis, dass eine gerade Zahl gewürfelt wird, so schreibt man dies
		als
		\[A=\left\lbrace 2;\ 4;\ 6\right\rbrace \]
		Da unser Ereignis \(A\) eine Menge ist lässt sich das Gegenereignis, welches
		immer dann eintritt, falls \(A\) nicht eintritt, als \(\bar{A}\) angeben.

	\subsection{Ereignismenge}
		\todo[color=red]{Kasten für Formel hinzufügen}
		\todo[color=red]{Zusammenfassung-Kasten hinzufügen}
		\todo[color=red]{Signalwörter-Kasten hinzufügen}
		Die Ereignismenge ist die Menge aller Ereignisse\footnote{Also eine Menge
		aller möglichen Mengen.}. Nehmen wir an, dass wir Würfeln. Es gibt also sechs
		mögliche Ereignisse. So würde man die Ereignismenge wie folgt darstellen:
		\[\Omega=\left\lbrace A_1;\ A_2;\ A_3;\ A_4;\ A_5;\ A_6\right\rbrace \]
		Wird gefragt, wie viele Ereignisse möglich sind, gebt niemals
		\(\Omega=\ldots\) an, denn die Ereignismenge ist eine Menge und allein deshalb
		schon keine Zahl. Was ihr dann angebt müsst ist die Anzahl der Elemente, also
		die Kardinalität, der Ereignismenge:
		\[\#\Omega=6\]

	\subsection{Wahrscheinlichkeit}
		\todo[color=red]{Kasten für Formel hinzufügen}
		\todo[color=red]{Zusammenfassung-Kasten hinzufügen}
		\todo[color=red]{Signalwörter-Kasten hinzufügen}
		Unser Ziel wird zumeist sein, die Wahrscheinlichkeit eines Ereignisses
		anzugeben. Das Ergebnis kann dann als Bruch (meist am bequemsten), Dezimalzahl
		oder als Prozentzahl angeben werden. Die Wahrscheinlichkeit P, dass das
		Ereignis A eintritt, ist dann (als Beispiel):
		\[P(A)=\frac{1}{2}=0,5=50\%\]
		An dieser Stelle möchten wir noch auf zwei Sätze eingehen, die uns zum Teil
		viel Arbeit ersparen können. Zuerst sollten wir die Nichtnegativität
		betrachten. Das bedeutet einfach, dass wir voraussetzen, dass alle
		Wahrscheinlichkeiten positive Zahlen sind (was auch Sinn macht, keine
		negativen Wahrscheinlichkeiten zu haben). So gilt:
		\[P(A)\geq 0 \mathrm{,\ für\ alle\ A}\]
		Weiter ist logisch nachvollziehbar, dass eines aller möglichen Ereignisse
		eintreten muss. Deshalb gilt auch folgende Relation:
		\[P(\Omega)=P(A_1)+P(A_2)+ \ldots+ P(A_n)=1\]
		Aus der gleichen Logik entspringt auch der folgende Ausdruck, welcher uns viel
		Arbeit ersparen wird. Denn die Wahrscheinlichkeit, dass ein Ereignis eintrifft
		\emph{und} dass es nicht eintrifft sind \(100\%\)(=1) (also die
		Wahrscheinlichkeiten summiert). Oder umgeschrieben:
		\[P(A)=1-P(\bar{A})\]

	\subsection{Zufallsgröße X}
		\todo[color=red]{Kasten für Formel hinzufügen}
		\todo[color=red]{Zusammenfassung-Kasten hinzufügen}
		\todo[color=red]{Signalwörter-Kasten hinzufügen}
		Haben wir Situationen, in denen mehrfach 'gezogen' wird, so gibt unsere
		Zufallsgröße an, wie oft das gewählte Ereignis eintreten soll, z. B. welche
		k's in die Bernoulli-Formel eingesetzt werden. Dies können dann alle k's höher
		oder niedriger oder genau ein Wert sein. Darauf gehen wir später genauer ein.

	\subsection{mit / ohne Zurücklegen}
		\todo[color=red]{Kasten für Formel hinzufügen}
		\todo[color=red]{Zusammenfassung-Kasten hinzufügen}
		\todo[color=red]{Signalwörter-Kasten hinzufügen}
		Macht man mehr als eine Probe, so gibt es unterschiedliche Möglichkeiten, wie
		die Wahrscheinlichkeiten im nächsten Zug sind.\\
		
		\(\star\) Ein System, in dem die Wahrscheinlichkeiten immer gleich sind,
		nennen wir ein System mit Zurücklegen. Aus dem Urnenmodell ist das leicht
		nachvollziehbar. Ziehe ich eine bestimmte Farbe mit der Wahrscheinlichkeit
		P(A), lege die Kugel zurück, so haben wir die gleiche Situation wie zuvor und
		somit auch die gleiche Wahrscheinlichkeit. Am Beispiel des Würfels ist dies
		leicht klar zu machen\footnote{Das 'Zurücklegen' ist gerade in diesem
		Vergleich ein schlechter Begriff, da es nicht unbedingt mit Ziehen oder Legen
		zusammenhängen muss. Da man jedoch oft das Urnenmodell als Vergleich hat,
		behalten wir das trotzdem bei.}.\\
		
		\(\star\) Systeme ohne Zurücklegen haben genau diese Eigenschaft nicht. Nach
		jedem Ziehen liegt eine neue Situation vor. Deshalb muss für jeden neuen Zug
		die Wahrscheinlichkeit neu berechnet werden.

	\subsection{Geordnet / Ungeordnet}
		\todo[color=red]{Kasten für Formel hinzufügen}
		\todo[color=red]{Zusammenfassung-Kasten hinzufügen}
		\todo[color=red]{Signalwörter-Kasten hinzufügen}
		Ungeachtet der Wahrscheinlichkeiten in einem Zug ist entscheidend, ob unser
		System geordnet oder ungeordnet ist, um zu entscheiden, welche und wie viele
		Pfade zum Erfolg führen (worauf später genauer eingegangen wird). Ein
		geordnetes System ist zum Beispiel eine PIN-Nummer. Selbst wenn man die
		Ziffern kennt und dann in der falschen Reihenfolge eingetippt, kommt man nicht
		ans Ziel. In diesem Beispiel gibt es also nur einen Pfad für das Eintreten des
		Ereignisses.\\
		Dagegen ist das Lotto spielen ein ungeordnetes System. Entscheidend ist nicht,
		welche Kugel als erstes gezogen wird, sondern lediglich welche Zahlen.


	% Section Mathematische Grundlagen
	\section{Mathematische Grundlagen}
Hier werden noch kurz mathematische Grundlagen angesprochen, auf die wir nachher zurückgreifen. Abgesehen davon, solltet ihr euch vor allem noch einmal mit den Regeln des Bruchrechnens und den Potenzgesetzen vertraut machen.

\subsection{Fakultät}
Die Fakultät wird in der Mathematik als Abkürzung verwendet. Immer dann, wenn man viele aufeinanderfolgende natürliche Zahlen multipliziert, verwendet man sie. Definiert ist sie wie folgt:
\[n!=1\cdot 2\cdot \ldots \cdot (n-1)\cdot n\ \&\ 0!=1\]

\subsection{Binomialkoeffizient}
Später werden wir seinen Nutzen noch genauer sehen, jetzt brauchen wir erstmal nur dessen Definition
\[\binom{n}{k}=\frac{n!}{(n-k)!\cdot k!}\mathrm{\ für\ alle\ }k,n\in \mathbb{N}\]
Weiter gilt folgendes:
\[\binom{n}{0}=\binom{n}{n}=1,\ \binom{n}{k}=\binom{n}{n-k}\]
Für kleine $n$ könnt ihr den Binomialkoeffizienten schnell mit Hilfe eines Pascal'schen Dreiecks bestimmen.\\
$
\begin{array}{ccccccccccccccccccccc}
n=0 &  &  &  &  &  & 1 &  &  &  &  &  &  &  &  &  & \binom{0}{0} &  &  &  & \\ 
n=1 &  &  &  &  & 1 &  & 1 &  &  &  &  &  &  &  & \binom{1}{0} &  & \binom{1}{1} &  &  & \\ 
n=2 &  &  &  & 1 &  & 2 &  & 1 &  &  & \Leftrightarrow &  &  & \binom{2}{0} &  & \binom{2}{1} &  & \binom{2}{2} &  & \\ 
n=3 &  &  & 1 &  & 3 &  & 3 &  & 1 &  &  &  & \binom{3}{0} &  & \binom{3}{1} &  & \binom{3}{2} &  & \binom{3}{3} & \\ 
n=4 &  & 1 &  & 4 &  & 6 &  & 4 &  & 1 &  & \binom{4}{0} &  & \binom{4}{1} &  & \binom{4}{2} &  & \binom{4}{3} &  & \binom{4}{4}\\
 & && && && && && && && && && &\\
k= & &0& &1& &2& &3& &4& &0& &1& &2& &3& &4   
\end{array} 
$


	% Section Einmaliges Ziehen
	\section{Einmaliges Ziehen}
	\subsection{Bei gleicher Wahrscheinlichkeit}
		\todo[color=green]{Kasten für Formel hinzufügen}
		\todo[color=purple]{Zusammenfassung-Kasten hinzufügen}
		\todo[color=purple]{Signalwörter-Kasten hinzufügen}
		Die einfachste Variante wäre, dass wir lediglich einmal 'ziehen', bzw. eine
		einzige Stichprobe nehmen. Letztendlich lässt sich auch das mehrmalige
		Wiederholen als immer neue einzelne Stichprobe ansehen (wobei dann auch die
		Kombination zu beachten ist). Doch wie können wir die Wahrscheinlichkeit für
		einen einzelnen Versuch nun berechnen?\\
		Zuerst definiert man die unterschiedlichen Ereignisse, die eintreten können.
		Nun kommt die Anzahl der Ergebnisse in den Nenner und die Anzahl derer, die zu
		dem gesuchten Ereignis gehören, in den Zähler und schon hat man die
		Wahrscheinlichkeit:
		\formel{\[P(A)=\frac{Anzahl\ der\ Ereignisse\ von\ A}{Anzahl\ der\ gesamten\
		Ereignisse}=\frac{\#A}{\#\Omega}\]}
		Leichter verständlich wird das an einem	Beispiel. Haben wir zum Beispiel einen
		Würfel, ist unser Ereignis, dessen Wahrscheinlichkeit wir berechnen wollen,
		nehmen wir nun an: A: es wird eine gerade Zahl gewürfelt. Nun haben wir sechs
		mögliche Zahlen, welche gewürfelt werden können, wobei jedes gleich
		wahrscheinlich ist. Damit Ereignis A eintritt, muss eine 2, 4 oder 6 gewürfelt
		werden. Wir haben also drei 'Mini-Ereignisse' für A. Somit gilt
		\(P(A)=\frac{3}{6}=\frac{1}{2}=50\%\).
	
	\subsection{Bei unterschiedlichen Wahrscheinlichkeiten}
		\todo[color=green]{Kasten für Formel hinzufügen}
		\todo[color=purple]{Zusammenfassung-Kasten hinzufügen}
		\todo[color=purple]{Signalwörter-Kasten hinzufügen}
		Bei unterschiedlichen Wahrscheinlichkeiten für einzelne Ereignisse müssen
		diese in der Aufgabenstellung gegeben sein. Es ist somit uninteressant die
		Wahrscheinlichkeit für ein einzelnes Ereignis \(A_1\) zu berechnen. Meist ist
		dann die Frage mit welcher Wahrscheinlichkeit eine Kombination von Ereignissen
		eintritt. Dafür addiert man die Einzel-Wahrscheinlichkeiten zusammen:
		\formel{\[P(A)=P(A_1)+P(A_2)+\cdots+P(A_n)\]}
		Bei einer Kombination aus sehr vielen Ereignissen ist es oft leichter die
		Wahrscheinlichkeit mit Hilfe des Gegenereignisses zu bestimmen:
		\formel{\[P(A)=1-P(\bar{A})=1-(P(\bar{A}_1)+P(\bar{A}_2)+\cdots+P(\bar{A}_n))\]}


	% Section Mehrmaliges Ziehen
	\section{Mehrmaliges Ziehen}

\subsection{Pfade}
\todo[color=red]{Kasten für Formel hinzufügen}
\todo[color=red]{Zusammenfassung-Kasten hinzufügen}
\todo[color=red]{Signalwörter-Kasten hinzufügen}
Zieht man nun häufiger, so ist das Bestimmen der Wahrscheinlichkeit nicht auf den ersten Blick sichtbar. Um sich die Arbeit zu vereinfachen, zeichnet man ein Pfad-Diagramm. So beginnt man von einem Punkt und zeichnet dann davon so viele Linien, wie es mögliche Ereignisse gibt (an deren Ende kennzeichnet man, um welches es sich handelt; zur Übersichtlichkeit alle Ereignisse untereinander). Auf oder neben die Linien kommt dann die Wahrscheinlichkeit für die jeweiligen Ereignisse. Nun wird an jedem Ereignis das Gleiche wiederholt. Oft wird ein Pfad-Diagramm, wegen seines Aussehens, auch als 'Baum'-Diagramm und dessen Pfade als 'Äste' bezeichnet\\
Liegt ein System mit Zurücklegen vor, so bleibt die Wahrscheinlichkeit für die Ereignisse immer gleich (wie beim Würfeln). Ohne Zurücklegen wird das schwieriger. Haben wir 5 Kugeln, 2 weiße und 3 schwarze, so ist die Wahrscheinlichkeit eine Schwarze zu ziehen \(\frac{3}{5}\). Legen wir diese nicht zurück, so ist beim nächsten Schritt eine schwarze Kugel weniger, die man ziehen kann, also auch insgesamt eine weniger. Somit ist dann die Wahrscheinlichkeit \(\frac{2}{4}=\frac{1}{2}\).\\
Man muss im Übrigen nicht alle möglichen Pfade aufzeichnen. Beobachtet man, wie hoch die Wahrscheinlichkeit ist, dass man nur schwarze Kugeln zieht, so ist es nicht nötig, einen Pfad mit einer weißen Kugel weiter zu führen.

\subsection{Ereignisse bei mehrmaligem Ziehen}
\todo[color=red]{Kasten für Formel hinzufügen}
\todo[color=red]{Zusammenfassung-Kasten hinzufügen}
\todo[color=red]{Signalwörter-Kasten hinzufügen}
Beim mehrmaligen Ziehen besteht ein 'Mini-Ereignis' aus mehreren Ereignissen hintereinander, immer pro Zug einem. So ist ein Element (ein Gesamtereignis) aus mehreren Objekten zusammengesetzt. Also ist bei zweimaligem Münzwurf (1 für Kopf, 0 für Zahl) das Ereignis, dass einmal Kopf und einmal Zahl (ungeordnet) geworfen wird:
\[A=\lbrace (Zahl;Kopf),\ (Kopf;Zahl)\rbrace\]

\subsection{Pfadregeln}
\todo[color=red]{Kasten für Formel hinzufügen}
\todo[color=red]{Zusammenfassung-Kasten hinzufügen}
\todo[color=red]{Signalwörter-Kasten hinzufügen}
\(\star\) Nun haben wir mehrere Pfade, die uns zur Verfügung stehen. Zunächst legen wir ein Augenmerk darauf, wie hoch die Wahrscheinlichkeit für \underline{einen} gesamten Pfad ist. Nehmen wir an, wir ziehen 2 mal und die Wahrscheinlichkeit für beide Ereignisse, die pro Ziehen stattfinden können, beträgt \(\frac{1}{2}\). Für die Wahrscheinlichkeit eines Pfades (zum Beispiel man würfelt 2-mal hintereinander eine gerade Zahl) gilt dann:
\[P(A)=P(A_1) \cdot P(A_2) \cdot \ldots\]
\(A_1\) ist das erste Mal 'iehen', \(A_2\) das zweite Mal, usw.. In unserem Beispiel wäre dann die Wahrscheinlichkeit für einen Pfad mit dem Ereignis A: \(P(A)=\frac{1}{2} \cdot \frac{1}{2}=\frac{1}{4}=25\%\). Da alle Pfade gleich wahrscheinlich sind, wir 4 Pfade haben und eines der Ereignisse immer eintreten muss, ist das Ergebnis auch schlüssig.\\
\(\star\) Nun kann es vorkommen (vor allem bei ungeordneten Systemen!), dass mehrere Pfade zum Erfolg führen. Wollen wir zum Beispiel wissen, wie hoch die Wahrscheinlichkeit ist, dass wir nur gerade oder nur ungerade Zahlen Würfeln, so liegen zwei mögliche Ereignisse vor. Die Wahrscheinlichkeit, dass das Ereignis (der Pfad) A oder B vorliegt, ist dann:
\[P(A\cup B)=P(A)+P(B)\]
In unserem Fall also \(P(C)=\frac{1}{4}+\frac{1}{4}=\frac{1}{2}=50\%\). Da alle Pfade gleichwertig sind und 2 von 4 auf das Ereignis zutreffen, ist auch das ersichtlich.

\subsection{Reduktion von möglichen Ereignissen}
\todo[color=red]{Kasten für Formel hinzufügen}
\todo[color=red]{Zusammenfassung-Kasten hinzufügen}
\todo[color=red]{Signalwörter-Kasten hinzufügen}
Um sich Arbeit zu sparen, lohnt es sich, Ereignisse zusammenzufassen. So will man in den meisten Fällen wissen, ob ein Ergebnis eintritt oder nicht. 
Nehmen wir wieder das Beispiel mit dem Würfel. So wäre das Eintreten des Ereignisses A hier, dass wir eine gerade Zahl würfeln. Das Gegenereignis \(\overline{A}\) ist dann das Würfeln einer ungeraden Zahl (wäre aber - auch umgekehrt - genau so richtig). Anstatt hier also sechs einzelne Ereignisse zu haben, können wir das auf zwei Ereignisse reduzieren (jeweils pro Ziehen).\\
Das ganze nennt sich auch \textbf{Bernoulli-Versuch}, welcher in den meisten der Aufgaben realisierbar ist.


	% Section häufiges Ziehen
	\section{häufiges Ziehen}
	In den meisten Fällen wird in den Aufgabenstellungen relativ häufig 'gezogen'.
	So häufig, dass das 'Pfade aufzeichnen' viel zu unübersichtlich wird. Jedoch
	gibt es hier einige Tricks, die wir hier noch einmal zusammenfassen wollen, für
	die vier möglichen Situationen, die es in der diskreten Statistik gibt.

	\subsection{mit Zurücklegen - geordnet}
		\todo[color=red]{Kasten für Formel hinzufügen}
		\todo[color=red]{Zusammenfassung-Kasten hinzufügen}
		\todo[color=red]{Signalwörter-Kasten hinzufügen}
		Hier lassen sich Wahrscheinlichkeiten mit etwas Überlegung einfach bestimmen,
		da die Wahrscheinlichkeit pro Versuch sich nicht verändert. Dank des
		Kommutativgesetzes (= des Vertauschungsgesetzes) kann man die
		Wahrscheinlichkeiten P einzeln multiplizieren (also hoch k nehmen) und
		multipliziert das dann mit den Gegenwahrscheinlichkeiten (hoch dem Rest n-k,
		also so oft das Ergebnis nicht eintritt). Soll bei n Versuchen k mal der
		Erfolg eintreten (deshalb P(X=k)), gilt in dem Fall
		\[P(E)=P(A)^k\cdot P(\overline{A})^{n-k} \Rightarrow\] \[P(X=k)=p^k\cdot
		(1-p)^{n-k}\]
		da ja die Wahrscheinlichkeit für das Nicht-Eintreten pro Pfad mitberechnet
		werden muss. Ob alle Ereignisse nun immer zuerst und danach alle
		Gegenereignisse eintreten oder ob das ganze gemischt auftritt, ist für die
		Berechnung aufgrund des Kommutativgesetzes, irrelevant. Die Wahrscheinlichkeit
		bleibt gleich, da wir durch das rein mathematische Umordnen den Pfad nicht
		verändern oder verlassen.

	\subsection{mit Zurücklegen - ungeordnet (Bernoulli-Formel)}
		\todo[color=red]{Kasten für Formel hinzufügen}
		\todo[color=red]{Zusammenfassung-Kasten hinzufügen}
		\todo[color=red]{Signalwörter-Kasten hinzufügen}
		\(\star\) Damit ein Bernoulli-Versuch vorliegt, muss folgendes gelten: Man
		kann den Versuch auf Eintreten \& Nicht-Eintreten reduzieren, da die
		Wahrscheinlichkeiten immer gleich bleiben (mit Zurücklegen) und die
		Reihenfolge irrelevant ist (ungeordnet).\\
		
		\(\star\) Haben wir einen Bernoulli-Versuch, so ist die Wahrscheinlichkeit für
		einen Pfad ebenso zu berechnen, wie im geordneten Fall. Jedoch gibt es mehrere
		Pfade, welche für ein Ereignis E geltend sind. Die Reihenfolge ist wieder
		unbedeutend, wie beim Lottospielen. Man addiert diese dann. Da alle Ereignisse
		die gleiche Wahrscheinlichkeit haben, kann man die Anzahl der dazugehörigen
		Pfade auch multiplizieren. Doch wie viele sind das? Nun, hier ist wieder der
		Binomialkoeffizient gefragt. Somit ergibt sich bei n Versuchen mit k Treffern
		\[P(X=k)=B(k,p,n)=\binom{n}{k}\cdot p^k\cdot (1-p)^{n-k}\] wobei 1-p die
		Gegenwahrscheinlichkeit vom Ereignis ist.


	% Section mindestens / höchstens
	\section{mindestens / höchstens}
	\todo[color=green]{Kasten für Formel hinzufügen}
	\todo[inline,color=red]{Zusammenfassung-Kasten hinzufügen}
	\todo[inline,color=red]{Signalwörter-Kasten hinzufügen}
	In einigen Aufgaben wird danach gefragt, wie hoch die Wahrscheinlichkeit ist,
	dass bei n mal ziehen mindestens / höchstens k mal das Ereignis eintrifft.
	Sollen Beispielsweise höchstens 5 Bauteile von 50 defekt sein, so müssen wir
	alle Pfade, in denen 1, 2, 3, 4 und 5 Bauteile defekt sind, berechnen und
	addieren. Bei mindestens folgt das selbe analog, nur eben dass alle mit 5 oder
	höher, addiert werden. So berechnet sich das Ganze wie folgt:
	\formel{\[\mathrm{\underline{Höchstens}: }P(X\leq k)=P(X=0)+P(X=1)+\ldots
	+P(X=k)\]
	\[\mathrm{\underline{Mindestens}: }P(X\geq k)=P(X=k)+P(X=k+1)+\ldots+P(X=n)\]}
	Ist die Wahrscheinlichkeit für 'weniger als' oder 'mehr als' gefragt, so
	schreibt man dies als \(P(X<k)\mathrm{\ bzw.\ }P(X>k)\)\footnote{Beachtet bei
	der Taschenrechner Eingabe, dass dort immer $P(X\leq k)$ oder $P(X\geq k)$
	gefragt ist!}.\\
	Ihr könnt euch auch oft viel Arbeit ersparen, wenn ihr auch hier die Formel
	\(P(A)=1-P(\overline{A})\) beachtet. Denn sollte in unserem vorherigen Beispiel
	die Wahrscheinlichkeit berechnet werden, dass mindestens 45 Bauteile
	funktionieren sollen, so ist dies das gleiche, wie wenn man die
	Wahrscheinlichkeit für höchstens 4 Bauteile berechnet und diese von 100\%
	abzieht (denn das ist die Gegenwahrscheinlichkeit dazu).


	% Section einseitiger Signifikanztest
	\section{einseitiger Signifikanztest}
	\todo[color=purple]{Kasten für Formel hinzufügen}
	\todo[color=purple]{Zusammenfassung-Kasten hinzufügen}
	\todo[color=purple]{Signalwörter-Kasten hinzufügen}
	Signifikanztests dienen der Überprüfung einer Behauptung, der Nullhypothese,
	und fallen in die Gruppe der „statisitschen Tests“.\\
	Stellt man die eine sog. Nullhypothese auf, so behauptet man, der Sachverhalt
	sei mit einer exakten Wahrscheinlichkeit verteilt, dies ist wie vorgehend
	erwähnt, lediglich eine Behauptung und spiegelt nicht die Realität wieder.\\
	Da bei einer Überprüfung lediglich eine Stichprobe entnommen wird und nicht die
	Gesamtheit aller Ereignisse betrachtet werden kann, ist es nicht möglich, die
	Behauptung vollständig zu be- oder widerlegen.\\
	Es ist lediglich möglich, die Wahrscheinlichkeit einer bestimmten Lösungsmenge
	für das Scheitern von \(H_1\) zu berechnen.\\
	Als Beispiel dient uns eine Lieferung an Elektronikartikeln. Der Hersteller
	gibt an, dass höchstens 10\% der 500 Teile defekt sind. Man betrachtet hier den
	Grenzfall. Das Ergebnis ist binomial verteilt.\\
	Für das Abitur in Baden-Württemberg sind nur die einseitigen Signifikanztests
	relevant.
	Hierbei schaut man sich an, wie sehr man sich irrt, wenn man annimmt, dass mit
	einer bestimmten Wahrscheinlichkeit höchstens, bzw. mindestens, eine bestimmte
	Anzahl an Nicht-Treffern auftritt. Ist diese entsprechend klein, kann man dann
	getrost von der eigentlichen Hypothese ausgehen.

	% Begriffe
	\subsection{Begriffe}
Zunächst sollten wir uns noch einmal die Begriffe klar machen, um zu wissen, worüber wir nachher reden.\\
\(\star\) Die angenommene (Einzel)Wahrscheinlichkeit \(p_0\) wird nachher - wie gewohnt - in die Bernoulliformel eingesetzt.\\
\(\star\) \(P(X\leq k) \mathrm{\ bzw.\ } P(X\geq k)\) sind unsere berechneten (Gesamt)Wahrscheinlichkeiten (das ist die Wahrscheinlichkeit, dass diese (Gegen)Hypothese stimmt).\\
\(\star\) k ist unser Grenzwert bei der Stichprobe, ab dem unsere aufgestellte Bedingung gilt, bzw. auf der anderen Seite, nicht mehr gilt.\\
\(\star\) \(H_0\) werden wir Nullhypothese nennen. Das ist die, welche wir annehmen.\\
\(\star\) \(H_1\) ist entsprechend die Gegenhypothese, die eintritt, falls wir falsch liegen (sie ist also das Gegenteil von \(H_0\)).\\
\(\star\) \(\alpha\) ist unsere Irrtumswahrscheinlichkeit. Diese wird meistens vorher festgelegt und gilt als Grenze. Liegt die berechnete Wahrscheinlichkeit darüber muss $H_0$ verworfen werden.\\
\(\star\) n ist die Anzahl an Stichproben, welche wir beim Versuch nehmen. k ist dann die Anzahl von Stichproben, bei denen das Ereignis eingetroffen ist (wie das bei den Wahrscheinlichkeiten selbst schon bekannt ist).


	% Die Idee
	\subsection{Die Idee}
	\todo[color=red]{Kasten für Formel hinzufügen}
	\todo[color=red]{Zusammenfassung-Kasten hinzufügen}
	\todo[color=red]{Signalwörter-Kasten hinzufügen}
	Die Idee dahinter ist auf den ersten Blick etwas abstrakt. Für das Abitur muss
	man sie nicht unbedingt verstanden haben, wir werden aber trotzdem versuchen,
	sie klar zu machen. Man versucht zu ermitteln, wie wahrscheinlich die
	Gegenhypothese ist, also wie sehr es sein kann, dass wir richtig liegen, wenn
	wir die Gegenhypothese \(H_1\) annehmen. Ist diese gering, so ist die
	Nullhypothese \(H_0\) entsprechend wahrscheinlich. Genau betrachten wir
	übrigens, wie wahrscheinlich \(p_0\) ist.\\
	Die Rechnung an sich verhält sich analog zu Bernoulli Versuchen mit höchstens
	oder mindestens mit der Ausnahme, dass beim Signifikantest die Grenze \(k\)
	gesucht und nicht gegeben ist. Das schwerste an den Signifikantests ist die
	Entscheidung, ob es sich um einen links- oder rechtsseitigen Test handelt.


	% Rechnung
	\subsection{Rechnung}
\subsubsection{Die Hypothesen \& links- oder rechtsseitiger Test}
Als erstes müssen wir aus dem Aufgabentext herauslesen, wie unsere Nullhypothese $H_0$ aussieht, falls sie nicht gegeben ist.\\
In unserem Beispiel lautet $H_0:\ p_0\leq 0.10$. Um nun unsere alternative Hypothese aufzustellen müssen wir von einer schlechteren Wahrscheinlichkeit ausgehen. Es gilt also: $H_1:\ p>0.10$.\\
Der nächste Schritt ist die Entscheidung, ob es sich um eine links- oder rechtsseitigen Test handelt. Dabei gibt es eine klare Regel\footnote{Verschiebt sich der Graph der Verteilung nach links ist es ein linksseitiger Test, verschiebt er sich nach rechts ist es ein rechtsseitiger Test}\\
$\star$ Gilt $p>p_0$ handelt es sich um einen rechtsseitigen Test.\\
$\star$ Gilt $p<p_0$ handelt es sich um einen linksseitigen Test.\\
\subsubsection{Die Gleichungen}
Bei einem linksseitigem Test werden alle Wahrscheinlichkeiten bis zu unserem Grenzwert $k$ addiert und hier kommt das einzig mathematisch Neue die Ungleichung:
\[P(X\geq k)=B(n,p_0,0)+B(n,p_0,1)+\ldots+B(n,p_0,k) \leq \alpha\]
Bei einem Rechtsseitigen Test entsprechen umgekehrt:
\[P(X\leq k)=B(n,p_0,k)+B(n,p_0,k+1)+\ldots+B(n,p_0,n) \leq \alpha\]
Nun muss die passende Gleichung nur noch nach $k$ aufgelöst werden. Dabei gibt es zwei Möglichkeiten\footnote{Für Maple Schüler bietet sich der Summen-Weg an, da dieser von Maple übersichtlicher dargestellt wird. GTR Schüler müssen den Weg über die Wertetabelle nehmen. CAS Schüler haben die Auswahl, wir empfehlen allerdings den Weg über die Wertetabelle, da es sein kann, dass der CAS bei der Summe zu lange rechnet.}:\\
$\star$ Das Umschreiben der Gleichung als Summe:
\[\text{Für den linksseitigen Test:}\sum_{i=0}^{k}B(n,p_0,i)\leq \alpha,\ \text{Für den rechtsseitigen Test:}\sum_{i=k}^{n}B(n,p_0,i)\leq \alpha\]
$\star$ Oder das Lösen mit Hilfe einer Wertetabelle.
Bei unserem Beispiel handelt es sich um einen rechtsseitigen Test da kein $\alpha$ in der Aufgabestellung gegeben war. Wir haben mit $\alpha=0.05$\footnote{Standardwerte für $\alpha$ sind $0.01,\ 0.05$ und $0.10$} gerechnet. Die Berechnungen ergeben:
\[P(X\geq 61)=0.0612,\ P(X\geq 62)=0.0465\] 


	% Auswertung des Ergebnisses
	\subsection{Auswertung des Ergebnisses}
Im Abitur wird meistens \(\alpha\) angegeben. Bei den einseitigen Signifikanztests können wir also genau betrachten, wann die Wahrscheinlichkeit für die Gegenhypothese diesen Wert überschreitet oder unterschreitet. Daher gibt die Ungleichung direkt an, welcher Ablehnungsbereich $\bar{A}$ und welcher Annahmebereich $A$ für unsere Nullhypothese $H_0$ gilt. Je nach dem, wie wichtig es ist, ob man \(\alpha\) vertraut, würde man selbiges verändern. Ein großer Wert für $\alpha$ bedeutet, dass wir uns mit einer größeren Wahrscheinlichkeit irren, was aber dazu führt, dass unser Annahmebereich größer ist. Aufgrund dessen können wir also eine eindeutige Antwort darauf finden, ob die Behauptung zutrifft oder nicht.\\
Bei unserem Beispiel schließen wir daraus, dass $H_1$ für mindestens 62 kaputte Teile zutrifft, also $H_0$ verworfen werden muss.
Es ergeben sich für $H_0$ ein Annahmebereich $A=\{0,\cdots,\ 61\}$, sowie ein Ablehnungsbereich $\bar{A}=\{62,\cdots,\ 500\}$
 