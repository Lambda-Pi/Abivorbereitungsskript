\section{einseitiger Signifikanztest}
Signifikanztests dienen der Überprüfung einer Behauptung, der Nullhypothese, und fallen in die Gruppe der „statisitschen Tests“.\\
Stellt man die eine sog. Nullhypothese auf, so behauptet man, der Sachverhalt sei mit einer exakten Wahrscheinlichkeit verteilt, dies ist wie vorgehend erwähnt, lediglich eine Behauptung und spiegelt nicht die Realität wieder.\\
Da bei einer Überprüfung lediglich eine Stichprobe entnommen wird und nicht die Gesamtheit aller Ereignisse betrachtet werden kann, ist es nicht möglich die Behauptung vollständig zu be- oder widerlegen.\\
Es ist lediglich möglich die Wahrscheinlichkeit einer bestimmten Lösungsmenge für das Scheitern von $H_1$ zu berechnen.\\
Als Beispiel dient uns eine Lieferung an Elektronikartikeln. Der Hersteller gibt an, dass höchstens 10\% der 500 Teile defekt sind. Man betrachtet hier den Grenzfall. Das Ergebnis ist binomial verteilt.\\
Für das Abitur in Baden-Württemberg sind nur die einseitigen Signifikanztests relevant.
Hierbei schaut man sich an, wie sehr man sich irrt, wenn man annimmt, dass mit einer bestimmten Wahrscheinlichkeit höchstens, bzw. mindestens, eine bestimmte Anzahl an Nicht-Treffern auftritt. Ist diese entsprechend klein, kann man dann getrost von der eigentlichen Hypothese ausgehen.

% Begriffe
\subsection{Begriffe}
Zunächst sollten wir uns noch einmal die Begriffe klar machen, um zu wissen, worüber wir nachher reden.\\
\(\star\) Die angenommene (Einzel)Wahrscheinlichkeit \(p_0\) wird nachher - wie gewohnt - in die Bernoulliformel eingesetzt.\\
\(\star\) \(P(X\leq k) \mathrm{\ bzw.\ } P(X\geq k)\) sind unsere berechneten (Gesamt)Wahrscheinlichkeiten (das ist die Wahrscheinlichkeit, dass diese (Gegen)Hypothese stimmt).\\
\(\star\) k ist unser Grenzwert bei der Stichprobe, ab dem unsere aufgestellte Bedingung gilt, bzw. auf der anderen Seite, nicht mehr gilt.\\
\(\star\) \(H_0\) werden wir Nullhypothese nennen. Das ist die, welche wir annehmen.\\
\(\star\) \(H_1\) ist entsprechend die Gegenhypothese, die eintritt, falls wir falsch liegen (sie ist also das Gegenteil von \(H_0\)).\\
\(\star\) \(\alpha\) ist unsere Irrtumswahrscheinlichkeit. Diese wird meistens vorher festgelegt und gilt als Grenze. Liegt die berechnete Wahrscheinlichkeit darüber muss $H_0$ verworfen werden.\\
\(\star\) n ist die Anzahl an Stichproben, welche wir beim Versuch nehmen. k ist dann die Anzahl von Stichproben, bei denen das Ereignis eingetroffen ist (wie das bei den Wahrscheinlichkeiten selbst schon bekannt ist).


% Die Idee
\subsection{Die Idee}
	\todo[color=red]{Kasten für Formel hinzufügen}
	\todo[color=red]{Zusammenfassung-Kasten hinzufügen}
	\todo[color=red]{Signalwörter-Kasten hinzufügen}
	Die Idee dahinter ist auf den ersten Blick etwas abstrakt. Für das Abitur muss
	man sie nicht unbedingt verstanden haben, wir werden aber trotzdem versuchen,
	sie klar zu machen. Man versucht zu ermitteln, wie wahrscheinlich die
	Gegenhypothese ist, also wie sehr es sein kann, dass wir richtig liegen, wenn
	wir die Gegenhypothese \(H_1\) annehmen. Ist diese gering, so ist die
	Nullhypothese \(H_0\) entsprechend wahrscheinlich. Genau betrachten wir
	übrigens, wie wahrscheinlich \(p_0\) ist.\\
	Die Rechnung an sich verhält sich analog zu Bernoulli Versuchen mit höchstens
	oder mindestens mit der Ausnahme, dass beim Signifikantest die Grenze \(k\)
	gesucht und nicht gegeben ist. Das schwerste an den Signifikantests ist die
	Entscheidung, ob es sich um einen links- oder rechtsseitigen Test handelt.


% Rechnung
\subsection{Rechnung}
\subsubsection{Die Hypothesen \& links- oder rechtsseitiger Test}
Als erstes müssen wir aus dem Aufgabentext herauslesen, wie unsere Nullhypothese $H_0$ aussieht, falls sie nicht gegeben ist.\\
In unserem Beispiel lautet $H_0:\ p_0\leq 0.10$. Um nun unsere alternative Hypothese aufzustellen müssen wir von einer schlechteren Wahrscheinlichkeit ausgehen. Es gilt also: $H_1:\ p>0.10$.\\
Der nächste Schritt ist die Entscheidung, ob es sich um eine links- oder rechtsseitigen Test handelt. Dabei gibt es eine klare Regel\footnote{Verschiebt sich der Graph der Verteilung nach links ist es ein linksseitiger Test, verschiebt er sich nach rechts ist es ein rechtsseitiger Test}\\
$\star$ Gilt $p>p_0$ handelt es sich um einen rechtsseitigen Test.\\
$\star$ Gilt $p<p_0$ handelt es sich um einen linksseitigen Test.\\
\subsubsection{Die Gleichungen}
Bei einem linksseitigem Test werden alle Wahrscheinlichkeiten bis zu unserem Grenzwert $k$ addiert und hier kommt das einzig mathematisch Neue die Ungleichung:
\[P(X\geq k)=B(n,p_0,0)+B(n,p_0,1)+\ldots+B(n,p_0,k) \leq \alpha\]
Bei einem Rechtsseitigen Test entsprechen umgekehrt:
\[P(X\leq k)=B(n,p_0,k)+B(n,p_0,k+1)+\ldots+B(n,p_0,n) \leq \alpha\]
Nun muss die passende Gleichung nur noch nach $k$ aufgelöst werden. Dabei gibt es zwei Möglichkeiten\footnote{Für Maple Schüler bietet sich der Summen-Weg an, da dieser von Maple übersichtlicher dargestellt wird. GTR Schüler müssen den Weg über die Wertetabelle nehmen. CAS Schüler haben die Auswahl, wir empfehlen allerdings den Weg über die Wertetabelle, da es sein kann, dass der CAS bei der Summe zu lange rechnet.}:\\
$\star$ Das Umschreiben der Gleichung als Summe:
\[\text{Für den linksseitigen Test:}\sum_{i=0}^{k}B(n,p_0,i)\leq \alpha,\ \text{Für den rechtsseitigen Test:}\sum_{i=k}^{n}B(n,p_0,i)\leq \alpha\]
$\star$ Oder das Lösen mit Hilfe einer Wertetabelle.
Bei unserem Beispiel handelt es sich um einen rechtsseitigen Test da kein $\alpha$ in der Aufgabestellung gegeben war. Wir haben mit $\alpha=0.05$\footnote{Standardwerte für $\alpha$ sind $0.01,\ 0.05$ und $0.10$} gerechnet. Die Berechnungen ergeben:
\[P(X\geq 61)=0.0612,\ P(X\geq 62)=0.0465\] 


% Auswertung des Ergebnisses
\subsection{Auswertung des Ergebnisses}
Im Abitur wird meistens \(\alpha\) angegeben. Bei den einseitigen Signifikanztests können wir also genau betrachten, wann die Wahrscheinlichkeit für die Gegenhypothese diesen Wert überschreitet oder unterschreitet. Daher gibt die Ungleichung direkt an, welcher Ablehnungsbereich $\bar{A}$ und welcher Annahmebereich $A$ für unsere Nullhypothese $H_0$ gilt. Je nach dem, wie wichtig es ist, ob man \(\alpha\) vertraut, würde man selbiges verändern. Ein großer Wert für $\alpha$ bedeutet, dass wir uns mit einer größeren Wahrscheinlichkeit irren, was aber dazu führt, dass unser Annahmebereich größer ist. Aufgrund dessen können wir also eine eindeutige Antwort darauf finden, ob die Behauptung zutrifft oder nicht.\\
Bei unserem Beispiel schließen wir daraus, dass $H_1$ für mindestens 62 kaputte Teile zutrifft, also $H_0$ verworfen werden muss.
Es ergeben sich für $H_0$ ein Annahmebereich $A=\{0,\cdots,\ 61\}$, sowie ein Ablehnungsbereich $\bar{A}=\{62,\cdots,\ 500\}$
 