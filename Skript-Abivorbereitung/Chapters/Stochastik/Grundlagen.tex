\section{Mathematische Grundlagen}
	Hier werden noch kurz mathematische Grundlagen angesprochen, auf die wir
	nachher zurückgreifen. Abgesehen davon, solltet ihr euch vor allem noch einmal
	mit den Regeln des Bruchrechnens und den Potenzgesetzen vertraut machen.

	\subsection{Fakultät}
		\todo[color=green]{Kasten für Formel hinzufügen}
		\todo[color=purple]{Zusammenfassung-Kasten hinzufügen}
		\todo[color=purple]{Signalwörter-Kasten hinzufügen}
		Die Fakultät wird in der Mathematik als Abkürzung verwendet. Immer dann, wenn
		man viele aufeinanderfolgende natürliche Zahlen multipliziert, verwendet man
		sie. Definiert ist sie wie folgt:
		\formel{\[n!=1\cdot 2\cdot \ldots \cdot (n-1)\cdot n\ \&\ 0!=1\]}

	\subsection{Binomialkoeffizient}
		\todo[color=purple]{Kasten für Formel hinzufügen}
		\todo[color=purple]{Zusammenfassung-Kasten hinzufügen}
		\todo[color=purple]{Signalwörter-Kasten hinzufügen}
		Später werden wir seinen Nutzen noch genauer sehen, jetzt brauchen wir erstmal
		nur dessen Definition
		\[\binom{n}{k}=\frac{n!}{(n-k)!\cdot k!}\mathrm{\ für\ alle\ }k,n\in
		\mathbb{N}\]
%		Weiter gilt folgendes:
%		\[\binom{n}{0}=\binom{n}{n}=1,\ \binom{n}{k}=\binom{n}{n-k}\]
%		Für kleine \(n\) könnt ihr den Binomialkoeffizienten schnell mit Hilfe eines
%		Pascal'schen Dreiecks bestimmen.\\
%		\(
%			\begin{array}{ccccccccccccccccccccc}
%				n=0 &  &  &  &  &  & 1 &  &  &  &  &  &  &  &  &  & \binom{0}{0} &  &  &  &
%				% \\
%				n=1 &  &  &  &  & 1 &  & 1 &  &  &  &  &  &  &  & \binom{1}{0} &  &
%				\binom{1}{1} &  &  & \\
%				n=2 &  &  &  & 1 &  & 2 &  & 1 &  &  & \Leftrightarrow &  &  & \binom{2}{0}
%				% & & \binom{2}{1} &  & \binom{2}{2} &  & \\
%				n=3 &  &  & 1 &  & 3 &  & 3 &  & 1 &  &  &  & \binom{3}{0} &  &
%				% \binom{3}{1} & & \binom{3}{2} &  & \binom{3}{3} & \\
%				n=4 &  & 1 &  & 4 &  & 6 &  & 4 &  & 1 &  & \binom{4}{0} &  & \binom{4}{1}
%				% & & \binom{4}{2} &  & \binom{4}{3} &  & \binom{4}{4}\\
%				 & && && && && && && && && && &\\
%				k= & &0& &1& &2& &3& &4& &0& &1& &2& &3& &4   
%			\end{array} 
%		\)
