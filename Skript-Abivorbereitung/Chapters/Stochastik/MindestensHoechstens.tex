\section{mindestens / höchstens}
	\todo[color=green]{Kasten für Formel hinzufügen}
	\todo[inline,color=red]{Zusammenfassung-Kasten hinzufügen}
	\todo[inline,color=red]{Signalwörter-Kasten hinzufügen}
	In einigen Aufgaben wird danach gefragt, wie hoch die Wahrscheinlichkeit ist,
	dass bei n mal ziehen mindestens / höchstens k mal das Ereignis eintrifft.
	Sollen Beispielsweise höchstens 5 Bauteile von 50 defekt sein, so müssen wir
	alle Pfade, in denen 1, 2, 3, 4 und 5 Bauteile defekt sind, berechnen und
	addieren. Bei mindestens folgt das selbe analog, nur eben dass alle mit 5 oder
	höher, addiert werden. So berechnet sich das Ganze wie folgt:
	\formel{\[\mathrm{\underline{Höchstens}: }P(X\leq k)=P(X=0)+P(X=1)+\ldots
	+P(X=k)\]
	\[\mathrm{\underline{Mindestens}: }P(X\geq k)=P(X=k)+P(X=k+1)+\ldots+P(X=n)\]}
	Ist die Wahrscheinlichkeit für 'weniger als' oder 'mehr als' gefragt, so
	schreibt man dies als \(P(X<k)\mathrm{\ bzw.\ }P(X>k)\)\footnote{Beachtet bei
	der Taschenrechner Eingabe, dass dort immer $P(X\leq k)$ oder $P(X\geq k)$
	gefragt ist!}.\\
	Ihr könnt euch auch oft viel Arbeit ersparen, wenn ihr auch hier die Formel
	\(P(A)=1-P(\overline{A})\) beachtet. Denn sollte in unserem vorherigen Beispiel
	die Wahrscheinlichkeit berechnet werden, dass mindestens 45 Bauteile
	funktionieren sollen, so ist dies das gleiche, wie wenn man die
	Wahrscheinlichkeit für höchstens 4 Bauteile berechnet und diese von 100\%
	abzieht (denn das ist die Gegenwahrscheinlichkeit dazu).
