\subsection{Die Idee}
	\todo[color=purple]{Kasten für Formel hinzufügen}
	\todo[color=purple]{Zusammenfassung-Kasten hinzufügen}
	\todo[color=purple]{Signalwörter-Kasten hinzufügen}
	Die Idee dahinter ist auf den ersten Blick etwas abstrakt. Für das Abitur muss
	man sie nicht unbedingt verstanden haben, wir werden aber trotzdem versuchen,
	sie klar zu machen. Man versucht zu ermitteln, wie wahrscheinlich die
	Gegenhypothese ist, also wie sehr es sein kann, dass wir richtig liegen, wenn
	wir die Gegenhypothese \(H_1\) annehmen. Ist diese gering, so ist die
	Nullhypothese \(H_0\) entsprechend wahrscheinlich. Genau betrachten wir
	übrigens, wie wahrscheinlich \(p_0\) ist.\\
	Die Rechnung an sich verhält sich analog zu Bernoulli Versuchen mit höchstens
	oder mindestens mit der Ausnahme, dass beim Signifikantest die Grenze \(k\)
	gesucht und nicht gegeben ist. Das schwerste an den Signifikantests ist die
	Entscheidung, ob es sich um einen links- oder rechtsseitigen Test handelt.
