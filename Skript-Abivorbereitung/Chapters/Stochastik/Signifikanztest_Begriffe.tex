\subsection{Begriffe}
	\todo[color=red]{Kasten für Formel hinzufügen}
	\todo[color=red]{Zusammenfassung-Kasten hinzufügen}
	\todo[color=red]{Signalwörter-Kasten hinzufügen}
	Zunächst sollten wir uns noch einmal die Begriffe klar machen, um zu wissen,
	worüber wir nachher reden.\\
	
	\(\star\) Die angenommene (Einzel)Wahrscheinlichkeit \(p_0\) wird nachher - wie
	gewohnt - in die Bernoulliformel eingesetzt.\\
	
	\(\star\) \(P(X\leq k) \mathrm{\ bzw.\ } P(X\geq k)\) sind unsere berechneten
	(Gesamt)Wahrscheinlichkeiten (das ist die Wahrscheinlichkeit, dass diese
	(Gegen)Hypothese stimmt).\\
	
	\(\star\) k ist unser Grenzwert bei der Stichprobe, ab dem unsere aufgestellte
	Bedingung gilt, bzw. auf der anderen Seite, nicht mehr gilt.\\
	
	\(\star\) \(H_0\) werden wir Nullhypothese nennen. Das ist die, welche wir
	annehmen.\\
	
	\(\star\) \(H_1\) ist entsprechend die Gegenhypothese, die eintritt, falls wir
	falsch liegen (sie ist also das Gegenteil von \(H_0\)).\\
	
	\(\star\) \(\alpha\) ist unsere Irrtumswahrscheinlichkeit. Diese wird meistens
	vorher festgelegt und gilt als Grenze. Liegt die berechnete Wahrscheinlichkeit
	darüber muss $H_0$ verworfen werden.\\
	
	\(\star\) n ist die Anzahl an Stichproben, welche wir beim Versuch nehmen. k
	ist dann die Anzahl von Stichproben, bei denen das Ereignis eingetroffen ist
	(wie das bei den Wahrscheinlichkeiten selbst schon bekannt ist).
