\subsection{Rechnung}
	\subsubsection{Die Hypothesen \& links- oder rechtsseitiger Test}
		\todo[color=red]{Kasten für Formel hinzufügen}
		\todo[color=red]{Zusammenfassung-Kasten hinzufügen}
		\todo[color=red]{Signalwörter-Kasten hinzufügen}
		Als erstes müssen wir aus dem Aufgabentext herauslesen, wie unsere
		Nullhypothese \(H_0\) aussieht, falls sie nicht gegeben ist.\\
		In unserem Beispiel lautet \(H_0:\ p_0\leq 0.10\). Um nun unsere alternative
		Hypothese aufzustellen müssen wir von einer schlechteren Wahrscheinlichkeit
		ausgehen. Es gilt also: \(H_1:\ p>0.10\).\\
		Der nächste Schritt ist die Entscheidung, ob es sich um eine links- oder
		rechtsseitigen Test handelt. Dabei gibt es eine klare
		Regel\footnote{Verschiebt sich der Graph der Verteilung nach links ist es ein
		linksseitiger Test, verschiebt er sich nach rechts ist es ein rechtsseitiger
		Test}\\
		\(\star\) Gilt \(p>p_0\) handelt es sich um einen rechtsseitigen Test.\\
		\(\star\) Gilt \(p<p_0\) handelt es sich um einen linksseitigen Test.\\
	
	\subsubsection{Die Gleichungen}
		\todo[color=red]{Kasten für Formel hinzufügen}
		\todo[color=red]{Zusammenfassung-Kasten hinzufügen}
		\todo[color=red]{Signalwörter-Kasten hinzufügen}
		Bei einem linksseitigem Test werden alle Wahrscheinlichkeiten bis zu unserem
		Grenzwert \(k\) addiert und hier kommt das einzig mathematisch Neue die
		Ungleichung:
		\[P(X\geq k)=B(n,p_0,0)+B(n,p_0,1)+\ldots+B(n,p_0,k) \leq \alpha\]
		Bei einem Rechtsseitigen Test entsprechen umgekehrt:
		\[P(X\leq k)=B(n,p_0,k)+B(n,p_0,k+1)+\ldots+B(n,p_0,n) \leq \alpha\]
		Nun muss die passende Gleichung nur noch nach \(k\) aufgelöst werden. Dabei
		gibt es zwei Möglichkeiten\footnote{Für Maple Schüler bietet sich der Summen-Weg
		an, da dieser von Maple übersichtlicher dargestellt wird. GTR Schüler müssen
		den Weg über die Wertetabelle nehmen. CAS Schüler haben die Auswahl, wir
		empfehlen allerdings den Weg über die Wertetabelle, da es sein kann, dass der
		CAS bei der Summe zu lange rechnet.}:\\

		\(\star\) Das Umschreiben der Gleichung als Summe:
		\[\text{Für den linksseitigen Test:}\sum_{i=0}^{k}B(n,p_0,i)\leq \alpha,\
		\text{Für den rechtsseitigen Test:}\sum_{i=k}^{n}B(n,p_0,i)\leq \alpha\]
		
		\(\star\) Oder das Lösen mit Hilfe einer Wertetabelle.
		Bei unserem Beispiel handelt es sich um einen rechtsseitigen Test da kein
		\(\alpha\) in der Aufgabestellung gegeben war. Wir haben mit
		\(\alpha=0.05\)\footnote{Standardwerte für \(\alpha\) sind \(0.01,\ 0.05\) und
		\(0.10\)} gerechnet. Die Berechnungen ergeben:
		\[P(X\geq 61)=0.0612,\ P(X\geq 62)=0.0465\] 
