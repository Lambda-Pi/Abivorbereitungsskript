\section{Begriffe}
	In der Stochastik tauchen einige neue Begriffe auf, die wir zunächst betrachten
	wollen, bevor wir diese dann anwenden.

	\subsection{Ereignis}
		\todo[color=purple]{Kasten für Formel hinzufügen}
		\todo[inline,color=red]{Zusammenfassung-Kasten hinzufügen}
		\todo[inline,color=red]{Signalwörter-Kasten hinzufügen}
		Zunächst sollten wir klären, was ein Ereignis in der Stochastik ist.
		Betrachten wir, wie etwas ausgehen kann, so gibt es meistens mehrere
		Möglichkeiten, die eintreten können. Diese werden in diesem Zusammenhang auch
		Ereignisse genannt. Jedes mögliche Ereignis bekommt dann einen großen
		Buchstaben (manchmal mit Indizes) zugewiesen\footnote{Das liegt daran, dass
		man hier Mengen angibt und keine Zahlen oder Sonstiges. Falls ihr damit
		Probleme habt schaut euch nochmal das dazugehörige Kapitel im Vorgeplänkel
		an}. Grundsätzlich können in einem Ereignis mehrere Einzelereignisse stecken.
		Ist unser Ereignis, dass eine gerade Zahl gewürfelt wird, so schreibt man dies
		als
		\[A=\left\lbrace 2;\ 4;\ 6\right\rbrace \]
		Da unser Ereignis \(A\) eine Menge ist lässt sich das Gegenereignis, welches
		immer dann eintritt, falls \(A\) nicht eintritt, als \(\bar{A}\) angeben.

	\subsection{Ereignismenge}
		\todo[color=purple]{Kasten für Formel hinzufügen}
		\todo[inline,color=red]{Zusammenfassung-Kasten hinzufügen}
		\todo[inline,color=red]{Signalwörter-Kasten hinzufügen}
		Die Ereignismenge ist die Menge aller Ereignisse\footnote{Also eine Menge
		aller möglichen Mengen.}. Nehmen wir an, dass wir Würfeln. Es gibt also sechs
		mögliche Ereignisse. So würde man die Ereignismenge wie folgt darstellen:
		\[\Omega=\left\lbrace A_1;\ A_2;\ A_3;\ A_4;\ A_5;\ A_6\right\rbrace \]
		Wird gefragt, wie viele Ereignisse möglich sind, gebt niemals
		\(\Omega=\ldots\) an, denn die Ereignismenge ist eine Menge und allein deshalb
		schon keine Zahl. Was ihr dann angebt müsst ist die Anzahl der Elemente, also
		die Kardinalität, der Ereignismenge:
		\[\#\Omega=6\]

	\subsection{Wahrscheinlichkeit}
		\todo[color=green]{Kasten für Formel hinzufügen}
		\todo[inline,color=red]{Zusammenfassung-Kasten hinzufügen}
		\todo[inline,color=red]{Signalwörter-Kasten hinzufügen}
		Unser Ziel wird zumeist sein, die Wahrscheinlichkeit eines Ereignisses
		anzugeben. Das Ergebnis kann dann als Bruch (meist am bequemsten), Dezimalzahl
		oder als Prozentzahl angeben werden. Die Wahrscheinlichkeit P, dass das
		Ereignis A eintritt, ist dann (als Beispiel):
		\[P(A)=\frac{1}{2}=0,5=50\%\]
		An dieser Stelle möchten wir noch auf zwei Sätze eingehen, die uns zum Teil
		viel Arbeit ersparen können. Zuerst sollten wir die Nichtnegativität
		betrachten. Das bedeutet einfach, dass wir voraussetzen, dass alle
		Wahrscheinlichkeiten positive Zahlen sind (was auch Sinn macht, keine
		negativen Wahrscheinlichkeiten zu haben). So gilt:
		\[P(A)\geq 0 \mathrm{,\ für\ alle\ A}\]
		Weiter ist logisch nachvollziehbar, dass eines aller möglichen Ereignisse
		eintreten muss. Deshalb gilt auch folgende Relation:
		\formel{\[P(\Omega)=P(A_1)+P(A_2)+ \ldots+ P(A_n)=1\]}
		Aus der gleichen Logik entspringt auch der folgende Ausdruck, welcher uns viel
		Arbeit ersparen wird. Denn die Wahrscheinlichkeit, dass ein Ereignis eintrifft
		\emph{und} dass es nicht eintrifft sind \(100\%\)(=1) (also die
		Wahrscheinlichkeiten summiert). Oder umgeschrieben:
		\formel{\[P(A)=1-P(\bar{A})\]}

	\subsection{Zufallsgröße X}
		\todo[color=purple]{Kasten für Formel hinzufügen}
		\todo[inline,color=red]{Zusammenfassung-Kasten hinzufügen}
		\todo[inline,color=red]{Signalwörter-Kasten hinzufügen}
		Haben wir Situationen, in denen mehrfach 'gezogen' wird, so gibt unsere
		Zufallsgröße an, wie oft das gewählte Ereignis eintreten soll, z. B. welche
		k's in die Bernoulli-Formel eingesetzt werden. Dies können dann alle k's höher
		oder niedriger oder genau ein Wert sein. Darauf gehen wir später genauer ein.

	\subsection{mit / ohne Zurücklegen}
		\todo[color=purple]{Kasten für Formel hinzufügen}
		\todo[inline,color=red]{Zusammenfassung-Kasten hinzufügen}
		\todo[inline,color=red]{Signalwörter-Kasten hinzufügen}
		Macht man mehr als eine Probe, so gibt es unterschiedliche Möglichkeiten, wie
		die Wahrscheinlichkeiten im nächsten Zug sind.\\
		
		\(\star\) Ein System, in dem die Wahrscheinlichkeiten immer gleich sind,
		nennen wir ein System mit Zurücklegen. Aus dem Urnenmodell ist das leicht
		nachvollziehbar. Ziehe ich eine bestimmte Farbe mit der Wahrscheinlichkeit
		P(A), lege die Kugel zurück, so haben wir die gleiche Situation wie zuvor und
		somit auch die gleiche Wahrscheinlichkeit. Am Beispiel des Würfels ist dies
		leicht klar zu machen\footnote{Das 'Zurücklegen' ist gerade in diesem
		Vergleich ein schlechter Begriff, da es nicht unbedingt mit Ziehen oder Legen
		zusammenhängen muss. Da man jedoch oft das Urnenmodell als Vergleich hat,
		behalten wir das trotzdem bei.}.\\
		
		\(\star\) Systeme ohne Zurücklegen haben genau diese Eigenschaft nicht. Nach
		jedem Ziehen liegt eine neue Situation vor. Deshalb muss für jeden neuen Zug
		die Wahrscheinlichkeit neu berechnet werden.

	\subsection{Geordnet / Ungeordnet}
		\todo[color=purple]{Kasten für Formel hinzufügen}
		\todo[inline,color=red]{Zusammenfassung-Kasten hinzufügen}
		\todo[inline,color=red]{Signalwörter-Kasten hinzufügen}
		Ungeachtet der Wahrscheinlichkeiten in einem Zug ist entscheidend, ob unser
		System geordnet oder ungeordnet ist, um zu entscheiden, welche und wie viele
		Pfade zum Erfolg führen (worauf später genauer eingegangen wird). Ein
		geordnetes System ist zum Beispiel eine PIN-Nummer. Selbst wenn man die
		Ziffern kennt und dann in der falschen Reihenfolge eingetippt, kommt man nicht
		ans Ziel. In diesem Beispiel gibt es also nur einen Pfad für das Eintreten des
		Ereignisses.\\
		Dagegen ist das Lotto spielen ein ungeordnetes System. Entscheidend ist nicht,
		welche Kugel als erstes gezogen wird, sondern lediglich welche Zahlen.
