\section{häufiges Ziehen}
In den meisten Fällen wird in den Aufgabenstellungen relativ häufig 'gezogen'. So häufig, dass das 'Pfade aufzeichnen' viel zu unübersichtlich wird. Jedoch gibt es hier einige Tricks, die wir hier noch einmal zusammenfassen wollen, für die vier möglichen Situationen, die es in der diskreten Statistik gibt.

\subsection{mit Zurücklegen - geordnet}
\todo[color=red]{Kasten für Formel hinzufügen}
\todo[color=red]{Zusammenfassung-Kasten hinzufügen}
\todo[color=red]{Signalwörter-Kasten hinzufügen}
Hier lassen sich Wahrscheinlichkeiten mit etwas Überlegung einfach bestimmen, da die Wahrscheinlichkeit pro Versuch sich nicht verändert. Dank des Kommutativgesetzes (= des Vertauschungsgesetzes) kann man die Wahrscheinlichkeiten P einzeln multiplizieren (also hoch k nehmen) und multipliziert das dann mit den Gegenwahrscheinlichkeiten (hoch dem Rest n-k, also so oft das Ergebnis nicht eintritt). Soll bei n Versuchen k mal der Erfolg eintreten (deshalb P(X=k)), gilt in dem Fall
\[P(E)=P(A)^k\cdot P(\overline{A})^{n-k} \Rightarrow\]
\[P(X=k)=p^k\cdot (1-p)^{n-k}\]
da ja die Wahrscheinlichkeit für das Nicht-Eintreten pro Pfad mitberechnet werden muss. Ob alle Ereignisse nun immer zuerst und danach alle Gegenereignisse eintreten oder ob das ganze gemischt auftritt, ist für die Berechnung aufgrund des Kommutativgesetzes, irrelevant. Die Wahrscheinlichkeit bleibt gleich, da wir durch das rein mathematische Umordnen den Pfad nicht verändern oder verlassen.

\subsection{mit Zurücklegen - ungeordnet (Bernoulli-Formel)}
\todo[color=red]{Kasten für Formel hinzufügen}
\todo[color=red]{Zusammenfassung-Kasten hinzufügen}
\todo[color=red]{Signalwörter-Kasten hinzufügen}
\(\star\) Damit ein Bernoulli-Versuch vorliegt, muss folgendes gelten: Man kann den Versuch auf Eintreten \& Nicht-Eintreten reduzieren, da die Wahrscheinlichkeiten immer gleich bleiben (mit Zurücklegen) und die Reihenfolge irrelevant ist (ungeordnet).\\
\(\star\) Haben wir einen Bernoulli-Versuch, so ist die Wahrscheinlichkeit für einen Pfad ebenso zu berechnen, wie im geordneten Fall. Jedoch gibt es mehrere Pfade, welche für ein Ereignis E geltend sind. Die Reihenfolge ist wieder unbedeutend, wie beim Lottospielen. Man addiert diese dann. Da alle Ereignisse die gleiche Wahrscheinlichkeit haben, kann man die Anzahl der dazugehörigen Pfade auch multiplizieren. Doch wie viele sind das? Nun, hier ist wieder der Binomialkoeffizient gefragt. Somit ergibt sich bei n Versuchen mit k Treffern
\[P(X=k)=B(k,p,n)=\binom{n}{k}\cdot p^k\cdot (1-p)^{n-k}\]
wobei 1-p die Gegenwahrscheinlichkeit vom Ereignis ist.
