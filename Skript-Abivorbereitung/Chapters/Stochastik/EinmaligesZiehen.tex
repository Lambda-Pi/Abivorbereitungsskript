\section{Einmaliges Ziehen}
\subsection{Bei gleicher Wahrscheinlichkeit}
Die einfachste Variante wäre, dass wir lediglich einmal 'ziehen', bzw. eine einzige Stichprobe nehmen. Letztendlich lässt sich auch das mehrmalige Wiederholen als immer neue einzelne Stichprobe ansehen (wobei dann auch die Kombination zu beachten ist). Doch wie können wir die Wahrscheinlichkeit für einen einzelnen Versuch nun berechnen?\\
Zuerst definiert man die unterschiedlichen Ereignisse, die eintreten können. Nun kommt die Anzahl der Ergebnisse in den Nenner und die Anzahl derer, die zu dem gesuchten Ereignis gehören, in den Zähler und schon hat man die Wahrscheinlichkeit:
\[P(A)=\frac{Anzahl\ der\ Ereignisse\ von\ A}{Anzahl\ der\ gesamten\ Ereignisse}=\frac{\#A}{\#\Omega}\]
Leichter verständlich wird das an einem Beispiel. Haben wir zum Beispiel einen Würfel, ist unser Ereignis, dessen Wahrscheinlichkeit wir berechnen wollen, nehmen wir nun an: A: es wird eine gerade Zahl gewürfelt. Nun haben wir sechs mögliche Zahlen, welche gewürfelt werden können, wobei jedes gleich wahrscheinlich ist. Damit Ereignis A eintritt, muss eine 2, 4 oder 6 gewürfelt werden. Wir haben also drei 'Mini-Ereignisse' für A. Somit gilt \(P(A)=\frac{3}{6}=\frac{1}{2}=50\%\).
\subsection{Bei unterschiedlichen Wahrscheinlichkeiten}
Bei unterschiedlichen Wahrscheinlichkeiten für einzelne Ereignisse müssen diese in der Aufgabenstellung gegeben sein. Es ist somit uninteressant die Wahrscheinlichkeit für ein einzelnes Ereignis $A_1$ zu berechnen. Meist ist dann die Frage mit welcher Wahrscheinlichkeit eine Kombination von Ereignissen eintritt. Dafür addiert man die Einzel-Wahrscheinlichkeiten zusammen:
\[P(A)=P(A_1)+P(A_2)+\cdots+P(A_n)\]
Bei einer Kombination aus sehr vielen Ereignissen ist es oft leichter die Wahrscheinlichkeit mit Hilfe des Gegenereignisses zu bestimmen:
\[P(A)=1-P(\bar{A})=1-(P(\bar{A}_1)+P(\bar{A}_2)+\cdots+P(\bar{A}_n))\]
