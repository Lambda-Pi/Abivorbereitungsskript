\section{p-q-Formel (auch Mitternachtsformel)}
\todo[color=red]{Kasten für Formel hinzufügen}
\todo[color=red]{Zusammenfassung-Kasten hinzufügen}
\todo[color=red]{Signalwörter-Kasten hinzufügen}
Die p-q-Formel wird immer dann benötigt, wenn wir eine quadratische Gleichung der Form \(x^2 + p x + q = 0\) \footnote{Vor dem \(x^2\) darf keine Zahl stehen und auf der rechten Seite nur eine 0!} auflösen wollen, wobei p \& q bekannt sind. p \& q werden dann einfach in folgende Gleichung eingesetzt\footnote{Beachtet bitte immer die Vorzeichen! Wenn z. B. q=-4 ist, so steht in der Wurzel ein +!}:
\[x_{1,2}=-\frac{p}{2}\pm \sqrt{\left ( \frac{p}{2} \right)^2-q}\]