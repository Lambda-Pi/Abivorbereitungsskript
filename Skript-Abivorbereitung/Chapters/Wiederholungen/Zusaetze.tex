\section{Zusätze}
Hier werden der Vollständigkeit halber noch 2 Punkte angesprochen, die oben keinen Platz gefunden haben.
\subsection{Definitionsbereich von Funktionen}
Der Definitionsbereich gibt an, welche Zahlen man für x einsetzen darf. Für gewöhnlich sind das alle aus \(\mathbb{R}\). Allerdings gibt es 3 Ausnahmen: Brüche, Logarithmen (auf welche wir gleich noch einmal zurück kommen werden) und die Wurzeln. Es gilt, dass in geraden Wurzeln (also die Quadratwurzel, die 4. Wurzel, usw.) keine negativen Zahlen stehen dürfen\footnote{Das stimmt so nicht ganz, da man das mit komplexen Zahlen berechnen kann. Aber das ist Uni-Stoff und nicht abi-relevant.} Also ist der Definitionsbereich von
\[f(x)=\sqrt{x}\ :\ x\geq 0\]
Definitionsbereich vom Logarithmus darf zusätzlich nicht die 0 enthalten!
\[f(x)=ln(x)\ :\ x>0\]
Zu guter Letzt noch die Brüche. Unter dem Bruchstrich darf keine 0 stehen. So ist der Definitionsbereich von
\[f(x)=\frac{1}{x}\ :\ x\in\mathbb{R}\setminus\{0\}\]
Natürlich müssen wir auch beachten, dass sich der Definitionsbereich, je nach Inhalt der Funktionen, verschieben kann. So wird aus:
\[f(x)=\frac{1}{x-1}\ :\ x\in\mathbb{R}\setminus\{1\}\]
\subsection{Logarithmus}
Der Logarithmus ist die Umkehrfunktion von einer Exponentialfunktion, ähnlich wie die Wurzel die Umkehrfunktion einer quadratischen Funktion darstellt. Ist n die Basis, so wird aus \(log_n\ n^x=x\). Somit 'eliminieren' wir die Basis (hier n). Meist wird die Basis die eulersche Zahl e sein. Der Logarithmus mit selbiger Basis wird mit ln (natürlicher Logarithmus genannt) bezeichnet. Welche Vorteile das hat, wird im Kapitel über Ableitungen klarer werden.
