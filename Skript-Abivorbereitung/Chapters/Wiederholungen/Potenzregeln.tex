\section{Potenzregeln}
	\todo[color=green]{Kasten für Formel hinzufügen}
	\todo[color=purple]{Zusammenfassung-Kasten hinzufügen}
	\todo[color=purple]{Signalwörter-Kasten hinzufügen}
	Auch wichtig sind die Potenzregeln. Grundsätzlich gilt:
	\formel{\[\underbrace{a\cdot a\cdot a\cdot\ldots\cdot a}_{n-Mal}=a^n\]}
	Des weiteren ist definiert, dass eine negative Hochzahl den Kehrwert angibt:
	\formel{\[\frac{1}{x^n}=x^{-n}\]}
	
	\(\star\) Kommen wir nun zu den eigentlichen Regeln. Multipliziert man zwei
	Potenzen mit gleicher Basis, so kann man die Exponenten addieren:
	\formel{\[a^m\cdot a^n=a^{m+n}\]}
	
	\(\star\) Schauen wir uns einmal die letzten zwei Regeln an und kombinieren
	sie. \(a^0=a^{n-n}=a^n\cdot a^{-n}=\frac{a^n}{a^n}=1\). Es gilt also:
	\formel{\[a^0=1,\text{ für }a\neq 0\]}
	
	\(\star\) Potenzieren wir eine schon potenzierte
	Zahl, so können wir die Potenzen einfach multiplizieren:
	\formel{\[\left ( a^n \right )^m=a^{n\cdot m}\]}
	
	\(\star\) Aus \(\sqrt{x}^2=\left ( x^{\frac{1}{2}}\right
	)^2=x^{\frac{2}{2}}=x\) können wir uns auch herleiten, wie man eine Wurzel als
	Potenz darstellen kann. Es gilt dann also:
	\formel{\[\sqrt[n]{x}=x^{\frac{1}{n}}\]}
	
	\(\star\) Haben wir die gleichen Exponenten, aber verschiedene Basen, so können
	wir zuerst die Basen multiplizieren und dann weiter rechnen:
	\formel{\[a^n\cdot b^n=(a\cdot b)^n, \textrm{\ bzw\ } \frac{a^n}{b^n}=\left (
	\frac{a}{b}\right ) ^n\]}
