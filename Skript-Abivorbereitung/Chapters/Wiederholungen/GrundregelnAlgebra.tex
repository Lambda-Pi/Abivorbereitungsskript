\section{Grundlegende Regeln der Algebra}
	Zwar sind die kommenden Regeln den meisten (wenn auch nicht nach dem Namen)
	bekannt, der Vollständigkeit halber werden wir sie aber trotzdem kurz erwähnen.
	
	\subsection{Distributivgesetz}
 		\todo[color=purple]{Zusammenfassung-Kasten hinzufügen}
 		\todo[color=purple]{Signalwörter-Kasten hinzufügen}
		\( \) % Bugfix, nur dann geht \formel{}
		\formel{\[a \cdot (b+c) = a \cdot b + a \cdot c\]}
		\\ \\
		Diese Regel gilt auch bei Brüchen!
		\[\frac{a+b}{c}=\frac{1}{c}\cdot (a+b)=\frac{a}{c}+\frac{b}{c}\]
	
	\subsection{Kommutativgesetz}
 		\todo[color=purple]{Zusammenfassung-Kasten hinzufügen}
 		\todo[color=purple]{Signalwörter-Kasten hinzufügen}
		Vertauschung von links \& rechts:
		\\ \\
		\formel{\[a+b=b+a,\ bzw\ a\cdot b=b\cdot a\]}
	
	\subsection{Assoziativgesetz}
 		\todo[color=purple]{Zusammenfassung-Kasten hinzufügen}
		\todo[color=purple]{Signalwörter-Kasten hinzufügen}
		Vernachlässigung der Reihenfolge:
		\\ \\
		\formel{\[(a+b)+c=a+(b+c),\ bzw\ (a\cdot b)\cdot c=a\cdot (b\cdot c)\]}