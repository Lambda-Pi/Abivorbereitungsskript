\section{Grundlegende Regeln der Algebra}
	Zwar sind die kommenden Regeln den meisten (wenn auch nicht nach dem Namen)
	bekannt, der Vollständigkeit halber werden wir sie aber trotzdem kurz erwähnen.
	
	\subsection{Distributivgesetz}
		\todo[color=red]{Kasten für Formel hinzufügen}
		\todo[color=red]{Zusammenfassung-Kasten hinzufügen}
		\todo[color=red]{Signalwörter-Kasten hinzufügen}
		\[a\cdot (b+c)=a\cdot b+a\cdot c\]
		Diese Regel gilt auch bei Brüchen!
		\[\frac{a+b}{c}=\frac{1}{c}\cdot (a+b)=\frac{a}{c}+\frac{b}{c}\]
	
	\subsection{Kommutativgesetz}
		\todo[color=red]{Kasten für Formel hinzufügen}
		\todo[color=red]{Zusammenfassung-Kasten hinzufügen}
		\todo[color=red]{Signalwörter-Kasten hinzufügen}
		Vertauschung von links \& rechts:
		\[a+b=b+a,\ bzw\ a\cdot b=b\cdot a\]
	
	\subsection{Assoziativgesetz}
		\todo[color=red]{Kasten für Formel hinzufügen}
		\todo[color=red]{Zusammenfassung-Kasten hinzufügen}
		\todo[color=red]{Signalwörter-Kasten hinzufügen}
		Vernachlässigung der Reihenfolge:
		\[(a+b)+c=a+(b+c),\ bzw\ (a\cdot b)\cdot c=a\cdot (b\cdot c)\]