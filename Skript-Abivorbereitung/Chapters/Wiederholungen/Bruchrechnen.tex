\section{Bruchrechnen}
	\todo[color=green]{Kasten für Formel hinzufügen}
	\todo[color=green]{Zusammenfassung-Kasten hinzufügen}
	\todo[color=purple]{Signalwörter-Kasten hinzufügen}
	\todo[color=green]{Neu schreiben!}
	Wir haben oftmals die Erfahrung gemacht, dass einige Schüler Probleme mit dem
	Bruchrechnen haben. Kommt man damit jedoch klar, so kann man sich viel Zeit,
	vor allem im Pflichtteil, ersparen und einige Rechnungen stark vereinfachen.
	Deshalb soll hier nochmal auf die verschiedenen Regeln eingegangen werden.\\
	Bevor wir hier auf die einzelnen Rechenregeln zu sprechen kommen möchten wir
	nochmal auf die Vorteile zu sprechen kommen, vor allem weil das Bruchrechnen
	bei vielen Schülern das Hassthema schlecht hin ist ;). Nehmen wir als Beispiel
	eine Aufgabe in der wir als Zwischenergebnis \(\frac{1}{3}\approx 0,333\)
	ist. Wird die Zahl später mit 3 multipliziert erhalten wir 0,999 als Ergebnis,
	was ein äußerst unschönes Ergebnis für weitere Rechnungen ist. Als Bruch
	geschrieben könnte man die 3 einfach wegkürzen und erhält 1 als Ergebnis.\\
	Zudem bekommt man auch oft einfachere Ergebnisse beim auflösen von Gleichungen.
	Und übrigens ist im Abitur \(\frac{1}{3}\) ein richtiges, und sogar deutlich
	genaueres, Ergebnis.\\
	
	\(\star\) \textbf{Addition \& Subtraktion: } Zwei Brüche dürfen nur addiert
	bzw. subtrahiert werden, wenn sie den gleichen Nenner besitzen. Gegebenfalls
	muss man vorher die Brüche erweitern, indem man Zähler und Nenner mit der
	gleichen Zahl multipliziert. Für die Addition gilt dann:
	\formel{\[\frac{a}{c}\pm\frac{b}{c}=\frac{a\pm b}{c}\]}
	
	\(\star\) \textbf{Multiplikation: } Die Multiplikation ist wohl die intuitivste
	Rechenregel bei Brüchen. Es werden einfach die Zähler und Nenner
	jeweils miteinander multipliziert:
	\formel{\[ \frac{a}{b} \cdot \frac{c}{d} = \frac{a\cdot c}{b\cdot c} \]}
	
	\(\star\) \textbf{Division: } Beim Teilen durch einen Bruch kann man einen
	einfachen Trick anwenden. Multipliziert man einen Bruch mit seinem Kehrwert so
	kürzen sich Zähler und Nenner jeweils und es bleibt 1 übrig (genau das
	passiert beim Teilen letztendlich immer). Beispielsweise ist
	\(\frac{2}{3}\cdot\frac{3}{2}=\frac{1\cdot 1}{1\cdot 1}=1\). Statt durch einen
	Bruch zu teilen kann man also auch seinen Kehrwert
	multiplizieren\footnote{Natürlich funktioniert das auch mit ganzen Zahlen.
	Statt durch 2 zu teilen kann man den Term auch mit \(\frac{1}{2}\)
	multiplizieren.}:
	\( \) % Ohne die Zeile kommt n Fehler, wieso auch immer
	\formel{ \[ \frac{a}{b} : \frac{c}{d} = \frac{a}{b} \cdot \frac{d}{c} \] }
	
	\summary{
		\begin{itemize}
		  \item Werden zwei Brüche addiert oder subtrahiert müssen die Nenner zunächst
		  auf den gleichen Nenner gebracht werden. Anschließend werden die Zähler
		  addiert bzw. subtrahiert, der Nenner bleibt unverändert stehen.
		  \item Multiplikation mit zwei Brüchen ist am einfachsten: Man multipliziert
		  Zähler und Zähler miteinander und ebenso Nenner und Nenner.
		  \item Bei der Division multipliziert man mit dem Kehrwert des Bruchs.
		\end{itemize}
	}