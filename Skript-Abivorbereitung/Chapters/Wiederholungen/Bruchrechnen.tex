\section{Bruchrechnen}
	\todo[color=red]{Kasten für Formel hinzufügen}
	\todo[color=red]{Zusammenfassung-Kasten hinzufügen}
	\todo[color=red]{Signalwörter-Kasten hinzufügen}
	Wir haben oftmals die Erfahrung gemacht, dass einige Schüler Probleme mit dem
	Bruchrechnen haben. Kommt man damit jedoch klar, so kann man sich viel Zeit im
	Pflichtteil ersparen und einige Rechnungen stark vereinfachen. \\
	
	\(\star\) Wie wir schon beim Distributivgesetz angesprochen hatten, gilt
	\[\frac{a+b}{c}=\frac{a}{c}+\frac{b}{c}\]
	
	\(\star\) Auch ein wichtiger Punkt, den wir ansprechen wollen, betrifft das
	Teilen durch einen Bruch. Haben wir zum Beispiel die Gleichung
	
	\(\frac{2}{4}\cdot x=4\) und wollen nach x auflösen, so müssen wir durch den
	Bruch teilen. \emph{Das ist aber das Gleiche wie das Produkt mit dem Kehrwert
	zu wählen!} Es gilt dann nämlich \(\frac{4}{2}\cdot \frac{2}{4}\cdot
	x=\frac{4}{2}\cdot 4\) und der Bruch kürzt sich einfach weg, auf der linken
	Seite.\\
	
	\(\star\) Noch kurz zum Vorteil der Schreibweise von Brüchen. Es gilt \(2\cdot
	\frac{1}{3}=\frac{2}{3}\). Der Vorteil, wenn man Zahlen als Bruch schreibt,
	anstatt als Dezimalzahl, liegt unter anderem daran, dass man leicht kürzen
	kann. Einzusehen dass \(6\cdot \frac{1}{3}=2\) ist deutlich einfacher als
	\(6\cdot 0,\widetilde{3}=2\).\\
	
	\(\star\) Zu guter Letzt kommen wir noch kurz auf das Addieren von Brüchen zu
	sprechen. Zuerst muss man den kleinsten gemeinsamen Nenner finden (in dem als
	Produkt aller Primzahlen dargestellt, sich beide Nenner finden lassen) und die
	Brüche dann auf diesen erweitern. Die Zähler werden dann einfach wie gewohnt
	addiert, während der Nenner gleich bleibt (siehe auch das Distributivgesetz bei
	Brüchen).
