\section{kurzer Ausblick in die Zahlentheorie}
	\todo[color=purple]{Kasten für Formel hinzufügen}
	\todo[color=green]{Zusammenfassung-Kasten hinzufügen}
	\todo[color=purple]{Signalwörter-Kasten hinzufügen}
	Zuerst möchten wir noch einmal auf die unterschiedlichen Zahlenmengen eingehen:\\
	
	\(\star\) Die natürlichen Zahlen: \(\mathbb{N}=\{1,2,3,\cdots\}\) \\
	
	\(\star\) Oft wird auch \(\mathbb{N}_0\) als Zahlenbereich angegeben:
	\(\mathbb{N}_0=\mathbb{N}\cup\{0\}\)\\
	
	\(\star\) Bei den ganzen Zahlen kommen noch negative Zahlen dazu:
	\(\mathbb{Z}=\{0,\mp 1,\mp 2,\mp 3,\cdots\}\) \\
	
	\(\star\) Die rationalen Zahlen lassen sich als Bruch einer natürlichen Zahl
	und einer ganzen Zahl darstellen, mit \(a\in\mathbb{N}\) und \(b\in\mathbb{Z}\)
	gilt: \(\frac{a}{b}\):
	\(\mathbb{Q}=\{-2,-\frac{1}{3},0,\frac{1}{2},1,\cdots\}\) \\
	
	$\star$ Man könnte meinen damit ist jeder Platz auf dem Zahlenstrahl belegt,
	aber es finden sich immer noch Plätze für Zahlen wie \(\pi\), \(\sqrt{2}\) und
	\(e\).
	Diese Zahlen heißen irrational, weil sie nicht zu den rationalen Zahlen
	\(\mathbb{Q}\) gehören. Mit den irrationalen Zahlen allein wird so gut wie nie
	gerechnet. Dargestellt werden die irrationalen Zahlen deshalb auch selten
	allein, man nimmt dafür \(\mathbb{I}\).\\

	\(\star\) Zuletzt noch die reellen Zahlen, welche alle Zahlenmengen davor
	beinhaltet. Also rein formell
	\(\mathbb{R}=\mathbb{N}\cup\mathbb{N}_0\cup\mathbb{Z}\cup\mathbb{Q}\cup\mathbb{I}\).
	Da aber \(\mathbb{Q}=\mathbb{N}\cup\mathbb{N}_0\cup\mathbb{Z}\) usw. gilt,
	schreibt man \(\mathbb{R}=\mathbb{Q}\cup\mathbb{I}\). \\ \\
	
	Grundsätzlich dürfen wir jede dieser Zahlenmengen nutzen. Meistens nutzen wir
	jedoch \(\mathbb{R}\), welche alle Zahlen einschließt, die ihr kennt. \\
	Wie können wir die Zahlen nun einsetzen? Dazu gibt es grundsätzlich zwei
	Möglichkeiten. Als Variablen oder Faktoren(/Konstanten\footnote{Der Unterschied
	ist hier nicht so wichtig, wir werden daher diese Begriffe als gleichwertig
	betrachten.}).\\
	
	\(\star\) Variablen kennen wir von Funktionen wie f(x). Bei f(x) stellt x die
	Variable der Funktion dar. Diese läuft alle Zahlen durch und nimmt jede
	mögliche Zahl an. Daraus ergibt sich dann jeweils ein Funktionswert f(x). Haben
	wir alle Funktionswerte zu jedem Wert der Variablen, so haben wir die Funktion
	komplett gezeichnet (sofern man die Werte einzeichnet). \\
	
	\(\star\) Was hat es nun mit den Faktoren auf sich? Diese stehen als Zahl
	alleine oder mit einer Variablen zusammen. Sie bilden einfach eine Zahl ab, die
	multipliziert, addiert oder sonst was wird. Entscheidend ist, dass dies auch so
	bleibt, wenn wir die Zahl nicht kennen und ihr deshalb einen Namen, wie zum
	Beispiel ein a oder c, geben.\\
	Variablen sind also Elemente, welche alle möglichen Werte annehmen können.
	Faktoren dagegen stehen als eine eindeutige Zahl da, unabhängig davon, wie sie
	dargestellt ist.
	Gleichzeitig dürfen wir Zahlen, auch ohne Buchstaben zu nutzen, verschieden
	darstellen. Ob wir \(2+2+2\text{ oder }3\cdot 2\) schreiben \textit{oder} 6
	\textit{oder} a (wenn wir die Zahl so nennen wollen) spielt also keine Rolle.
	Alle Ausdrücke sind unterschiedliche Darstellungen für die gleiche Zahl. Diese
	Tatsache kann helfen, Terme anders zu schreiben und sie damit zu
	vereinfachen.\\
	\summary{\(\star\) Variablen (Veränderliche) - innerhalb einer Funkion, nehmen
	alle möglichen Werte innerhalb einer Funktion an\\
	\(\star\) Faktoren - (Manchmal) Unbekannte, die jedoch einen festen Wert
	besitzen}
