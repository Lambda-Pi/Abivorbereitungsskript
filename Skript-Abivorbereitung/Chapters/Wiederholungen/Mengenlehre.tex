\section{Mengenlehre}
	\todo[color=green]{Kürzen / evt ganz weglassen da im Abi nicht benötigt}
	\todo[color=purple]{Kasten für Formel hinzufügen}
	\todo[color=purple]{Zusammenfassung-Kasten hinzufügen}
	\todo[color=purple]{Signalwörter-Kasten hinzufügen}
	Dieser Teil wird zwar im Abitur so nicht explizit abgefragt, aber kann einige
	Probleme, die viele unserer Nachhilfeschüler hatten, auflösen.\\
	Gerade um Aufgaben in der Wahrscheinlichkeitsrechnung besser verstehen zu
	können, ist es von Vorteil ein Verständnis von Mengen zu haben.

	\subsection{Definition}
		\todo[color=purple]{Kasten für Formel hinzufügen}
		\todo[color=purple]{Zusammenfassung-Kasten hinzufügen}
		\todo[color=purple]{Signalwörter-Kasten hinzufügen}
		Mengen sind keine Zahlen oder Symbole, aber eine Menge kann diese als Elemente
		beinhalten. Letztendlich werden Mengen benutzt um bestimmte Dinge
		(Elemente, Zahlen, die verschiedenen Schüler,\ldots) zu gruppieren und
		zusammenzufassen. Ihr kennt das bereits aus der 7. Klasse. Dort musste man
		beim Lösen von Gleichungen immer die Lösungsmenge \(L\) angeben. Diese sah
		dann möglicherweise wie folgt aus: \(L:=\{2,\ 3,\ 5\}\). \(2,\ 3,\ 5\) heißen
		dann auch Elemente der Lösungsmenge \(L\).\\
		Noch anzumerken ist, dass auch \(\textit{rot}\), \(\textit{schwarz}\),
		\(\textit{Kopf}\), \(\textit{Zahl}\),\(\cdots\) Elemente einer Menge sein
		können.\\

	\subsection{Wichtige Eigenschaften von Mengen}
		\todo[color=purple]{Kasten für Formel hinzufügen}
		\todo[color=purple]{Zusammenfassung-Kasten hinzufügen}
		\todo[color=purple]{Signalwörter-Kasten hinzufügen}
			Auch für Mengen gibt es Rechenregeln. Die wichtigsten, zumindest die die wir
			brauchen, sind hier kurz zusammengefasst.
			\subsubsection{Komplement - in der Statistik oft Gegenmenge}
				Das Komplement einer Menge sind alle Elemente, die nicht in dieser Menge
				enthalten sind. Hat man zum Beispiel die geraden Zahlen des Würfels
				\(A=\{2,\ 4,\ 6\}\), so ist das Komplement die Menge mit allen ungeraden
				Zahlen \(\bar{A}=\{1,\ 3,\ 5\}\).
			\subsubsection{Mächtigkeit}
				Später ist es interessant zu wissen wie viele Elemente eine Menge enthält.
				Dafür müssen wir "den Betrag" der Menge betrachten. Anders als bei Zahlen
				heißt dieser hier Mächtigkeit.\\
				Die Mächtigkeit der Menge der geraden Zahlen beim Würfel wäre dann
				\(|A|=3\), da die Menge aus 3 Elementen besteht.
