\section{Mengenlehre}
\todo[color=red]{Kürzen / evt ganz weglassen da im Abi nicht benötigt}
Dieser Teil wird zwar im Abitur so nicht explizit abgefragt, aber kann einige Probleme, die viele unserer Nachhilfeschüler hatten, auflösen.\\
Gerade um Aufgaben in der Wahrscheinlichkeitsrechnung besser verstehen zu können, ist es von Vorteil ein rudimentäres Verständnis davon zu haben, was Mengen sind und bedeuten.\\
Weshalb wir euch Mengen etwas näher bringen wollen, obwohl sie nicht im Unterricht behandelt werden, liegt daran, dass eigentlich alles Mathematische, was wir gelernt haben, auf Mengen basiert. Außerdem lassen sich Mengen wunderschön graphisch darstellen und dadurch vieles Abstraktes etwas begreifbarer wird.

\subsection{Definition}
Mengen sind keine Zahlen oder Symbole, aber eine Menge kann diese als Elemente beinhalten. Ihr kennt das bereits aus der 7. Klasse. Dort musste man beim Lösen von Gleichungen immer die Lösungsmenge $L$ angeben. Diese sah dann möglicherweise wie folgt aus: $L:=\{2,\ 3,\ 5\}$. $2,\ 3,\ 5$ heißen dann auch Elemente der Lösungsmenge $L$. Um die Unterscheidung zu vereinfachen, werden Mengen meist mit Großbuchstaben und deren Elemente mit Kleinbuchstaben bezeichnet. Da Mengen aber auch weiter Mengen enthalten können, lässt sich dies nicht abschließen so festlegen.\\
Noch anzumerken ist, dass auch $\textit{rot}$, $\textit{schwarz}$, $\textit{Kopf}$, $\textit{Zahl}$,$\cdots$ Elemente einer Menge sein können.\\

\subsection{Wichtige Eigenschaften von Mengen}
\subsubsection{Rechenregeln}$ $\\
Auch für Mengen gibt es Rechenregeln. Die wichtigsten dieser Rechenregeln sind hier kurz zusammengefasst.\\
Betrachten wir die Mengen $A:=\{1,\ 2,\cdots,\ 5\},\ B:=\{6,\ 7,\cdots,\ 10\}$ und $C:=\{5,\ 6\}$.
\paragraph{Vereinigung und Schnittmenge}$ $\\
Die Vereinigung entspricht in etwa der Addition. So gilt: $D:=A\cup B=\{1,\ 2,\cdots,\ 5,\ 6,\ 7,\cdots,\ 10\}$\\
Der Schnitt, oder auch Schnittmenge, ist etwas komplizierter. Es handelt sich um die Menge der Elemente, die in jeder der geschnittenen Mengen liegen, so gilt: $E:=A\cap C=\{5\}$\\
\paragraph{Differenz}$ $\\
Die Differenz zweier Mengen ist die Menge der Elemente welcher in der Einen Menge aber nicht in der anderen Menge enthalten sind, es gilt: $F:=A\setminus C=\{1,\ 2,\ 3,\ 4\}$\\
\paragraph{Komplement - Ausgeschlossenes 3.}$ $\\
Die wohl nützlichste Eigenschaft ist das Komplement einer Menge. Betrachten wir $G:=A\cup B\cup C=\{1,\ 2,\cdots,\ 10\}$ als Menge aller Element, so gilt: $G\setminus A=:\bar{A}$ bzw. $A\cup\bar{A}:=G$ und $A\cap\bar{A}=\emptyset$.\\
Es gibt also kein einziges Element aus $G$, welches nicht in $A$ oder $\bar{A}$ enthalten ist.\\ 

\paragraph{Kardinalität oder Mächtigkeit}$ $\\
Später ist es interessant zu wissen wie viele Elemente eine Menge enthält. Dafür müssen wir "den Betrag"  der Menge betrachten. Anders als bei Zahlen heißt dieser Kardinalität oder Mächtigkeit.\\
Die Kardinalität einer Menge ist immer die Anzahl der Elemente, so gilt: $\vert G\vert=\# G=10$, und $\vert\{A,\ B,\ C\}\vert=\# \{A,\ B,\ C\}=3$.\\
