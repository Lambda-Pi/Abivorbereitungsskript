\section{Gleichungen auflösen}
\todo[color=red]{Kasten für Formel hinzufügen}
\todo[color=red]{Zusammenfassung-Kasten hinzufügen}
\todo[color=red]{Signalwörter-Kasten hinzufügen}
Dieses Kapitel ist Fundamental für das komplette Mathe-Abitur. Egal ob wir Funktionen aufstellen, ein lineares Gleichungssystem (kurz LGS) im Geometrieteil lösen oder auch bei einigen anderen Aufgaben im Pflichtteil (ansonsten Übernimmt unser CAS-Taschenrechner diese Aufgaben) muss man Gleichungen auflösen. Die Palette an Anwendungen sollte also Motivation genug sein, sich hier noch einmal ausgiebig damit zu beschäftigen.\\
\(\star\)Das Wichtigste beim Auflösen einer jeden Gleichung ist, dass bei jedem Rechenschritt auf beiden Seiten das Gleiche gemacht werden muss. Grund hierfür ist die Grundüberlegung einer Gleichung. Nehmen wir an, wir haben eine Waage. Steht die Waage im Gleichgewicht, so muss die Masse links genau so groß sein, wie rechts. Wollen wir das Gleichgewicht beibehalten, so müssen wir auch auf beiden Seiten das Gleiche machen. Nehmen wir 5 kg von der einen Seite, so müssen wir das auf der anderen Seite auch tun. Genauso verhält sich das bei Gleichungen. \\
\(\star\)Ein weiterer wichtiger Punkt ist, dass bei allen Gleichungen die wir Teilen / miteinander Mal nehmen / den Logarithmus nehmen / hoch e nehmen / quadrieren / usw., der Rechenschritt mit dem kompletten Term gemacht werden muss. Ein kurzes Beispiel:
\[ln(x+2)=4+b\ |e^{...}\Rightarrow\ x+2=e^{4+b}\]
\(\star\) Um eine (lineare) Gleichung aufzulösen, gehen wir immer von außen nach innen vor\footnote{Leider ist eine komplette Anleitung hier schwer umzusetzen. Wir beschränken uns hier also auf einige wenige Beispiele. Im Kurs selbst werden wir jedoch alle Fälle durchgehen.} Zuerst wird versucht, alle Summanden mit unserer gesuchten Zahl (zum Beispiel x)zuerst auf eine Seite des =, dann auf den Zähler zu bekommen oder aus dem \(e\) oder sonstigen raus zu nehmen (durch die jeweilige Gegenfunktion). Danach haben wir die Gleichung meistens schon so gut wie gelöst. \emph{Haben wir quadratische Gleichungen, so ist natürlich die p-q-Formel zu benutzen.}\\
\(\star\) Schwerer scheint es, wenn e oder trigonometrische Funktionen im Spiel sind, die quadriert werden. Dann substituieren wir einfach, das heißt, wir ersetzen (z. B.) jedes \(e^x\) durch einen Buchstaben (geben ihm also einen Spitznamen ;) ). Zum Beispiel \(e^{2x}+e^x+1=0\ =>\ u^2+u+1=0\text{, für }u=e^x\) \footnote{Vergesst dann bitte nicht u zurück zu substituieren, wenn ihr einen Wert für u habt und dann nach x umzustellen.}
