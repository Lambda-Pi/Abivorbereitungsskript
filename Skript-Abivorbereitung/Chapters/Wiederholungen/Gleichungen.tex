\section{Gleichungen auflösen}
	\todo[color=green]{Kasten für Formel hinzufügen}
	\todo[color=purple]{Zusammenfassung-Kasten hinzufügen}
	\todo[color=purple]{Signalwörter-Kasten hinzufügen}
	\todo[color=green]{Strukturierter neu schreiben}
	Dieses Kapitel ist Fundamental für das komplette Mathe-Abitur. Egal ob wir
	Funktionen aufstellen, ein lineares Gleichungssystem (kurz LGS)\footnote{Was
	allerdings so Umfangreich ist, das es in einem eigenen Abschnitt erklärt wird.}
	im Geometrieteil lösen oder auch bei einigen anderen Aufgaben im Pflichtteil
	(ansonsten Übernimmt unser CAS-Taschenrechner diese Aufgaben) muss man
	Gleichungen auflösen. Die Palette an Anwendungen sollte also Motivation genug
	sein, sich hier noch einmal ausgiebig damit zu beschäftigen. Das Problem beim
	Gleichungen auflösen ist das es nicht den einen richtigen Weg gibt und auch
	sehr von der Art der Gleichung abhängig ist.\\
	Grundsätzlich gilt jedoch bei allen Gleichungen, wenn auf einer Seite was
	verändert wird, so muss dies auch auf der anderen Seite passieren. Und zwar
	immer mit dem kompletten Term! Als Analogie hierzu: Wenn auf einer
	ausgeglichenen Waage auf der linken Seite alles verdoppelt werden soll, so muss
	auch auf der rechten Seite \underline{alles} verdoppelt werden damit sie wieder
	im Gleichgewicht ist.\\
	
	\(\star\) \textbf{Lineare Gleichungen: } Diese bilden die einfachsten
	Gleichungen und man versucht auch oft andere Arten von Gleichungen auf diese
	umzuformen. Grundsätzlich geht man beim auflösen wie folgt vor: Zunächst bringt
	man mit addieren und subtrahieren alle Summanden mit der Unbekannten auf eine
	Seite und alle anderen auf die andere Seite. Falls nötig klammert man dann die
	unbekannte noch aus. Danach wird durch den Term der vor der unbekannten steht
	geteilt.\\
	
	\(\star\) \textbf{Quadratische Gleichungen: } Kommt die gesuchte Variable nur
	quadratisch vor, so kann man erstmal wie bei den linearen Gleichungen vorgehen.
	Wurde entsprechen (nach z.B. \(x^2\)) aufgelöst, so muss man nur noch die
	Wurzel ziehen. Bitte achtet darauf, dass man 2 Ergebnisse hat, einmal
	\(+\sqrt{\ldots}\) und einmal \(-\sqrt{\ldots}\).\\
	Hat man die Variable quadratisch und linear im Term, so muss man die
	p-q-Formel, bzw Mitternachtsformel benutzen.\\
	
	\(\star\) \textbf{Gebrochenrationale Gleichungen: } Manchmal steht die
	Unbekannte unterm Bruchstrich. Dann muss die Gleichung mit dem Nenner
	multipliziert werden, und man hat dann meist eine lineare oder quadratische
	Funktion.\\
	
	\(\star\) \textbf{Wurzelgleichungen: } Zunächst muss die Wurzel mit der
	Unbekannten darin isolieren. Das bedeutet das diese alleine auf einer Seite
	steht. Anschließend Quadriert man beide Seiten und kann wie gewohnt weiter
	rechnen.\\
	
	\(\star\) \textbf{Exponential Gleichungen: } Hier geht man ähnlich vor wie bei
	den Wurzelgleichungen. Erst wird das \(e^{\ldots}\) isoliert. Anschließend
	nimmt man auf beiden Seiten den ln, wodurch das e verschwindet.\\
	
	\(\star\) \textbf{Gleichungen die mit Substitution zu lösen sind: } Das ist die
	schwierigste Art von Gleichungen die im Abi dran kommen können. Das Problem
	besteht aber meist darin, zu erkennen das es eine solche Gleichung ist. Zwei
	Varianten können vorkommen: Entweder mit ganzrationalen Funktionen oder mit
	e-Funktionen. Bei beiden Fällen sind die Exponenten der einen Variablen doppelt
	so groß wie bei der anderen (gegebenfalls auch erst beim Umformen). Zum
	Beispiel könnte einmal (\( x^2,\ x^4 \)) oder (\( x^3,\ x^6 \)) oder (\( e^x,\
	e^{2x} \)) oder \ldots vorkommen.\\
	Dann geht man wie folgt vor: Zunächst substituiert man, das bedeutet das man
	den Teil mit dem kleineren Exponenten durch einen beliebigen Buchstaben ersetzt
	(z.B. \( u:=x^2 \) ). Der andere Teil muss dann entsprechend mit substituiert
	werden (in unserem Beispiel \( x^4=u^2 \)). Anschließend haben wir eine
	quadratische Funktion die mit der p-q-Formel aufgelöst werden kann.
	Anschließend haben wir zwei Ergebnisse für u, wollen aber natürlich die x-Werte
	wissen die einzusetzen sind. Das heißt wir müssen zurücksubstituieren, also in
	die Substitution einsetzen und nach x auflösen (in unserem Beispiel \( u=x^2
	\)).