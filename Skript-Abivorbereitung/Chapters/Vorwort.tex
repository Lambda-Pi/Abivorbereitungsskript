\chapter*{Vorwort}
	\todo[color=green]{Vorwort nach Umstrukturierung des Skripts erneuern
	(pt)} Dieses Skript soll unterstützend zu unserem Abi-Vorbereitungskurs in Mathematik
	in Baden-Württemberg gelten. Wir haben versucht den Stoff für das Abitur auf
	das Wesentliche zu konzentrieren und hier als Nachschlagewerk zusammenzufassen.
	Wir erheben keinen Anspruch auf Vollständigkeit, aber es sollte alles vorhanden
	sein was ihr für das Mathe-Abi braucht. Sollte dem nicht so sein oder euch
	Fehler auffallen, dann gebt uns bitte Bescheid, damit wir diese beheben können
	;).\\
	Wir möchten noch erwähnen, dass wir hier an einigen Stellen bewusst Sachen
	nicht 100\%ig mathematisch korrekt benannt haben. Das hat aber den Hintergrund,
	dass uns die Didaktik und das Verständnis des Stoffes wichtiger sind, als die
	mathematisch korrekten Definitionen. Wir haben aber versucht, alles - für das
	Abitur ausreichend korrekt - zu beschreiben. Ansonsten hoffen wir, dass euch
	der Kurs ein paar Notenpunkte extra bringt und ihr auch sonst ein erfolgreiches
	Abitur schreibt :).\\
	Das Skript steht unter der '\textbf{GNU Free Documentation License}, Version
	1.3', das heißt, das Skript steht allen frei zur Verfügung und darf unter
	Nennung der Autoren (bzw. unserer Internetseite \url{http://www.lambda-pi.de})
	verbreitet und verändert werden, sofern diese Lizenz beibehalten wird. Das
	heißt ihr dürft das Skript nicht nur kostenlos weitergeben sondern auch
	verbessern und so verändern wie ihr wollt. Die Lizenz ist unter anderem hier
	veröffentlicht:\\
	\url{http://www.gnu.org/licenses/fdl-1.3.html}\\
	Mit der Version für das Jahr 2016 haben wir einige Neuerungen eingebracht. So
	gibt es, zumindest für komplexere Themen, Zusammenfassungen und Schlagwörter.
	Diese wurden, ebenso wie die Formeln, in Kästchen hervorgehoben, sodass ihr die
	wichtigsten Dinge sofort zur Hand habt. Außerdem bieten wir euch einige Tipps
	wie ihr an Aufgaben herangehen und die Prüfungen überstehen könnt.
