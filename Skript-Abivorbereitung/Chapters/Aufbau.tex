\chapter*{Aufbau des Skriptes}
	An dieser Stelle erfahrt ihr, wie unser Skript, bzw. die einzelnen Kapitel und
	Unterkapitel aufgebaut sind.
	\section*{Aufbau des Inhalts}
	Die Kapitel sind wie folgt angeordnet:
		\begin{enumerate}
		  \item \textbf{Wiederholungen \& Vorgeplänkel:} Hier werden die wichtigsten
		  Sachen aus der Mittelstufe wiederholt und einige Grundlagen aus
		  dem Oberstufenstoff vorgezogen.
		  \item \textbf{Analysis:} In diesem Kapitel beschäftigen wir uns mit allem,
		  was mit Funktionen zu tun hat. Insbesondere Ableiten, Integrieren und deren
		  Anwendung. Ebenso werden die Themen Kurvendiskussion,
		  Differentialgleichungen und Funktionsscharen behandelt.
		  \item \textbf{Analytische Geometrie (Vektoren):} Hier beschäftigen wir uns
		  mit der 3-Dimensionalen Geometrie, also mit Vektoren im Raum. Zunächst
		  werden die Grundlagen geschaffen und die Darstellung von Geraden und
		  Ebenen erklärt. Anschließend wird erklärt, wie die Lagebeziehungen zwischen
		  den unterschiedlichen Elementen rechnerisch bestimmt und eventuelle
		  Abstände berechnet werden können.
		  \item \textbf{Stochastik:} Alles was mit Wahrscheinlichkeitsrechnung zu tun
		  hat, findet hier seinen Platz. Dabei gehen wir auf die Grundlagen ein, die
		  im Pflichtteil abgefragt werden können, um anschließend den einseitigen
		  Signifikanztest zu behandeln.
		  \item \textbf{Strukturiertes Lösen von Aufgaben:} Hier wird vor allem darauf
		  eingegangen, wie man eine komplexere Aufgabe angeht. Zwar können wir hier
		  keine Lösung für alle Aufgaben geben, aber Tipps und Anregungen geben, wie
		  solche Probleme zu lösen sind.
		  \item \textbf{Auf was ihr beim Abi achten solltet:} In diesem Kapitel findet
		  ihr noch Tipps und Tricks, die für das Abi hilfreich sein können.
		\end{enumerate}
	
	\newpage %Nur für diese Version notwendig. Bei änderung des Textes prüfen, ob
	% das weg gelassen werden kann
	\section*{Aufbau der einzelnen Abschnitte}
		Um die wichtigen Dinge hervorzuheben, haben wir, wenn sie sinnvoll sind,
		Kästchen eingefügt.
		Davon gibt es drei Arten:\\
		Die Wichtigen Formeln sind so dargestellt:
		\\ \\
		\formel{\[ e^{i\phi}+1=0 \]}
		\\ \\
		Bei manchen Abschnitten findet sich am Ende eine Zusammenfassung, die ohne
		Erklärungen nochmal das wichtigste aus dem Kapitel auf einen Blick darstellt:
		\summary{
			\begin{enumerate}
			  \item 1 ist eine Zahl.
			  \item 1+1=2.
			  \item 42 ist die Antwort auf alles.
			  \item \ldots
			\end{enumerate}
		}
		Bei mehreren Themen können auch Signalwörter hilfreich sein, um zu
		erkennen, dass genau diese Formeln / Anleitungen gebraucht werden:
		\tags{Sid, Jessi, Wheezy, Buzz, Rex, Potato}
