\chapter*{Aufbau des Skriptes}
	\section*{Aufbau des Inhalts}
		Die Kaptel sind wie folgt angeordnet:
		\begin{enumerate}
		  \item \textbf{Wiederholungen \& Vorgeplänkel:} Hier werden die wichtigsten
		  Sachen aus der Mittelstufe wiederholt und einige Grundlagen vorgezogen.
		  \item \textbf{Analysis:} In diesem Kapitel beschäftigen wir uns mit allem
		  was mit Funktionen zu tun hat. Insbesondere Ableiten, Integrieren und deren
		  Anwendung. Ebenso werden die Themen Kurvendiskussion,
		  Differentialgleichungen und Funktionsscharen behandelt.
		  \item \textbf{Analytische Geometrie (Vektoren):} Hier beschäftigen wir uns
		  mit der 3-Dimensionalen Geometrie, also mit Vektoren im Raum. Zunächst
		  werden die Grundlagen geschaffen und die Darstellungen von Geraden und
		  Ebenen erklärt. Anschließend wird erklärt wie die Lagebeziehungen zwischen
		  den verschiedenen Elementen rechnerisch bestimmt werden können.
		  \item \textbf{Stochastik:} Alles was mit Wahrscheinlichkeitsrechnung zu tun
		  hat findet hier ihren Platz. Dabei gehen wir auf die Grundlagen ein, die im
		  Pflichtteil abgefragt werden können, um anschließend den einseitigen
		  Signifikanztest zu behandeln.
		  \item \textbf{Strukturiertes Lösen von Aufgaben:} Hier wird vor allem darauf
		  eingegangen wie man an komplexere Aufgaben rangeht. Zwar können wir hier
		  keine Lösung für alle Aufgaben geben, aber Tipps und Anregungen geben wie
		  solche Probleme zu lösen sind.
		  \item \textbf{Auf was man beim Abi achten sollte:} In diesem Kapitel geben
		  wir noch Tipps und Tricks die für das Abi hilfreich sind.
		\end{enumerate}
	
	\section*{Aufbau der einzelnen Abschnitte}
		Um die wichtigen Dinge hervorzuheben haben wir, wenn sie sinnvoll sind,
		Kästchen eingefügt.
		Dabei gibt es 3 Arten:\\
		Die wichtigen Formeln sind so dargestellt:
		\formel{\[ e^{i\phi}+1=0 \]}
		Am Ende finden sich bei manchen Abschnitten Zusammenfassungen, die ohne
		Erklärungen nochmal das wichtigste auf einen Blick bieten:
		\summary{
			\begin{enumerate}
			  \item 1 ist eine Zahl.
			  \item 1+1=2.
			  \item 42 ist die Antwort auf alles.
			  \item \ldots
			\end{enumerate}
		}
		Bei verschiedenen Themen können auch Signalwörter hilfreich sein, um zu
		erkennen das genau diese Formeln / Anleitungen gebraucht werden:
		\tags{Sid, Jessi, Wheezy, Buzz, Rex, Potato}
