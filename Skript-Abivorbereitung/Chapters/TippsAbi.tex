\chapter{Auf was man beim Abi achten sollte}
	Die Prüfungen kommen und der Stress wächst. Ob es die Angst ist zu versagen
	oder man sich einfach unvorbereitet fühlt, wir wollen euch hier ein paar Tipps
	geben die euch vor und während dem Abi helfen können.
	
	\section{Vorbereitung}
		Eine gute Vorbereitung ist alles :) Hier ein paar Tipps die wir für sinnvoll
		erachten:
		\begin{itemize}
		  \item \textbf{Lernt frühzeitig!} Hört sich einfacher an als es ist aber so
		  spart ihr euch unnötig Stress!
		  \item \textbf{Lernt in Gruppen!} Das motivierst wenn mal die "`kein Bock"'
		  Phase kommt und Fragen können gegenseitig beantwortet werden.
		  \item \textbf{Hört rechtzeitig auf mit dem Lernen!} Ein paar Tage bevor man
		  die Prüfungen schreibt sollte man es lassen. Was man bis dahin nicht konnte
		  lernt man auch nicht in der Nacht vor der Prüfung. Das Einzige was passieren
		  kann ist, dass ihr durch den Stress bei einer Aufgabe nicht weiter kommt und
		  letztendlich komplett demotiviert seid. Mal nochmal die Zusammenfassung
		  durchzulesen ist aber OK.
		  \item \textbf{Bereitet euer Zeug für die Prüfung vor!} Richtet euch das
		  Essen, Stifte, Geodreieck etc. hin und schaut das die Batterien im Taschenrechner
		  durch neue ersetzt worden sind. Je weniger ihr am Morgen der Prüfung
		  zusammen sammeln müsst, desto mehr sinkt der Stressfaktor.
		  \item \textbf{Seid ausgeschlafen!} Bringt euch am Abend vor der Prüfung
		  etwas runter und versucht früh schlafen zu gehen. Je Ausgeschlafener ihr seid desto
		  besser laufen die Klausuren. Außerdem: Der frühe Wurm wird gefressen\ldots
		  oder iwie sowas.
		\end{itemize}
		
	\section{Während der Prüfung}
		\begin{itemize}
		  \item \textbf{Überspringt Aufgaben die ihr nicht sofort lösen könnt!} So
		  verhindert ihr das ihr euch verzettelt. Oftmals hilft es später nochmal die
		  Aufgabe zu bearbeiten. Außerdem verhindert ihr so, dass ihr Aufgaben die ihr
		  könnt aus Zeitgründen nicht mehr lösen könnt.
		  \item \textbf{Wenn ihr Zeit habt, probiert auch die Aufgaben die ihr nicht
		  könnt!} Raten ist besser als nichts hinzuschreiben. Mit Glück gibt es
		  Teilpunkte.
		  \item \textbf{Streicht nur Sachen durch, wenn ihr die Aufgabe anschließend
		  richtig löst!} Ärgerlich für jeden Korrektor ist es, wenn richtige, oder
		  fast richtige, Lösungen durchgestrichen wurden. Erst recht wenn unten kein
		  Lösungswert mehr kommt. So können gar keine Punkte gegeben werden, ansonsten
		  kann es immerhin Teilpunkte geben.
		  \item \textbf{Schreibt so, dass man auch versteht was ihr wollt!} Mathematik
		  bedeutet nicht das NUR Formeln und Zahlen da stehen. Ein kurzes Stichwort
		  was ihr da gerade macht hilft zu verstehen was ihr wollt. So können, auch
		  wenn man sich in der Formel vergriffen hat aber die richtige Idee hatte,
		  Teilpunkte geben werden. Vor allem da der Lösungsweg mehr Punkte gibt als
		  die Lösung!
		\end{itemize}
		
	\section{Nach den Prüfungen}
		Nach gelungener Arbeit muss man sich auch gelungen Belohnen. Geht Feiern,
		macht Urlaub, geht euren Hobbies nach. Auch das gehört zu einer Prüfung ;) Und
		die Vorfreude darauf lässt einen auch über die Lernphasen kommen ;)
